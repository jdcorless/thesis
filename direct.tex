% Chapter 1:  Direct Excitation: Theory


\begin{singlespace}

\chapter{Direct Excitation: Theory}
\label{direct}
\begpagestyle
%\pagenumbering{arabic}

\end{singlespace}

In this chapter, we consider the problem of complete
population transfer from an initially populated ground state to a desired
Rydberg state by direct single-photon excitation. The close proximity of
neighboring Rydberg states to the desired Rydberg state can cause the ground
state to be coupled to multiple Rydberg levels, and we examine under what
conditions this interaction can be reduced to the usual two-level problem
\cite{Allen:87}.  This leads us to the realization that angular momentum
mixing of degenerate Rydberg states in the optical field can be a critical
limitation to the selectivity of the optical excitation.  We formulate a
model describing this interaction and explore its consequences.

\section{Limitations to Two-Level Model}

\hspace{\parindent} The most straightforward technique to excite efficiently
a single Rydberg eigenstate is to use direct optical excitation from the
ground state to the desired Rydberg state.  Because of the parity selection
rules, this method is limited to low angular momentum quantum numbers.
However, several techniques of dressing the Rydberg $n$-manifold with dc or
low-frequency fields enable excitation to higher angular momentum quantum
numbers \cite{Hulet:83,Molander:86}.  In each case the dressing field splits
the (nearly) degenerate states in the manifold so that laser excitation can
couple to a particular dressed state.  Subsequent turn-off of the external
field determines the resultant distribution of angular momentum quantum
numbers.  This suggests that the excitation of an arbitrary Rydberg
eigenstate can be reduced to the problem of exciting the upper level of a
two-level atom.  Standard excitation techniques can then be employed,
including the use of $\pi$-pulses \cite{Allen:87}, or adiabatic following
using a slowly varying sweep field (which can either be a dc field
\cite{Loy:74} or a chirped optical field \cite{Melinger:92}).

When is this two-level approximation invalid?  Of course, the pulse duration
must be long enough so that the Fourier bandwidth of the pulse only overlaps
a single $n$-manifold.  This implies that the pulse duration must be greater
than the Kepler period, $T_k = 2 \pi n^3$ (atomic units are used unless
otherwise specified), for the desired
level $n$.  For pulses shorter than $T_k$ a coherent superposition of
Rydberg states will be formed which has been shown (under weak field
excitation conditions) to create a wave packet state that has many
interesting classical and quantum properties
\cite{Parker:86,tenWolde:88,Yeazell:89}.  Furthermore, we expect that as the
peak electric field becomes large, neighboring Rydberg states can become
excited as well.  For instance, if the ground-to-Rydberg Rabi frequency,
$\Omega_{gn}$, exceeds the local level spacing, $\delta_n = 1/n^3$, then a
wave packet may still be created due to power broadening of the off-resonant
Rydberg transitions.  This type of wave packet has been referred to as a
``depletion wave packet'' \cite{Alber:91} because the rapid depletion of the
ground state gives rise to a coherent excitation of off-resonant Rydberg
states.  The large Rabi frequency therefore gives rise to $n$-mixing of the
Rydberg states.

Strong electric fields can also give rise to another type of Rydberg state
mixing.  It is well known that Rydberg states are easily perturbed by dc or
low-frequency radiation fields.  This is due to the fact that the dipole moment
connecting states with the same $n$ but differing $\ell$ grows as $n^2$ for
$\ell \ll n$.  These large dipole moments give rise to the coupling of states
of differing $\ell$ within an $n$-manifold.  For the moment we are considering
an atom in an optical frequency field and are ignoring any possibilities of the
presence of dc or low-frequency fields.  The question at this point then is, can
electric fields created by the optical frequency field that is inducing the
ground-to-Rydberg transition be large enough to induce this type of
$\ell$-mixing?

The coupling of these nearest neighbor degenerate states is characterized by
a detuning given by the optical frequency.  Therefore, we can expect that
these transitions are negligible for field strengths that produce Rabi
frequencies much less than the optical frequency.  Now, optical frequencies
are on the order of $10^{15}$ Hz.  Rabi frequencies vary greatly depending
on dipole moments and field strengths, however, we can elucidate the physics
by considering the Rabi frequency associated with the familiar doublet
transition, $3S_{1/2} \rightarrow 3P_{1/2}$ in sodium, with wavelength
$\lambda = 589.76$ nm.  This transition has a dipole moment of $1.5$ a.u
\cite{Wiese:69}.  In order for this transition to have a Rabi frequency
equal to the applied optical frequency that resonantly drives the transition
requires a peak intensity of 9.3 $\times 10^{13}$ W/cm$^2$.  With a field of
such enormous intensity, perhaps it is not surprising that the Rabi
frequencies would be large.  However, in the case of degenerate Rydberg
transitions, dipole moments become asymptotically large with
increasing $n$.  For example, the $nS
\rightarrow nP$ transition in hydrogen has a dipole moment (for linear
polarization) given by
\begin{equation}
d_{nS \rightarrow nP} = \frac{\sqrt{3}}{2} n \sqrt{n^2 -1}.
\label{dnSnP}
\end{equation}
For $n = 30$ this is  $d_{30S \rightarrow 30P} \approx 779$ a.u.
If the same optical field with $\lambda = 589.76$~nm is used to couple the
$30S$ state of hydrogen to the $30P$ state, an intensity of 3.5~$\times
10^8$~W/cm$^2$ is enough to make the associated Rabi frequency equal to the
optical frequency.  We therefore see that modest optical fields can give rise
to mixing of degenerate Rydberg states.

\begin{figure}[tbp]
\postfull{thesisfigs/direct/inten.eps}
\bigskip
\ncap{Critical intensities to induce Rydberg state mixing.} {Critical
intensities to induce Rydberg state mixing. The dashed line is the intensity
for which the $1S \rightarrow nP$ Rabi frequency equals the local level spacing
and the solid line is the intensity for which the $nS \rightarrow nP$ Rabi
frequency equals the optical frequency.
\label{inten}}
\end{figure}

We have identified two types of Rydberg state mixing which can occur
as the result of a large electric field.  The first is  $n$-mixing due to
strong ground-to-Rydberg coupling, and the other is $\ell$-mixing due to
strong Rydberg-to-Rydberg coupling. To characterize these two types of
couplings,
\Fig{inten} shows the critical intensity required to make the $1S
\rightarrow nP$ Rabi frequency equal the local level spacing, and the
critical intensity needed to make the $nS
\rightarrow nP$ Rabi frequency equal to the $1S \rightarrow nP$ transition
frequency.  These intensities therefore identify threshold values above which
the corresponding type of Rydberg mixing ($n$ or $\ell$) can occur.  It is
interesting to note that for fixed $n$, as the intensity is increased, $\ell$
levels are mixed before the $n$ levels.  Furthermore, for an energy level
with $n \approx 30$ (a relatively small $n$ value given that a recent
experiment in potassium excited $n \approx 520$ \cite{Frey:96}),
the required $\ell$ mixing field is only $10^{10}$~W/cm$^2$.

Now that we have shown that the $\ell$-mixing of Rydberg states is the
strongest interaction, it is clear that if the goal is to preferentially excite
a single Rydberg eigenstate, then the peak electric field must be kept as small
as possible to avoid $\ell$-mixing in the final state.   The threshold for
$\ell$-mixing of Rydberg states was identified to be the intensity for which the
$nS \rightarrow nP$ Rabi frequency equals the $1S \rightarrow nP$ transition
frequency, or in atomic units
\begin{equation}
\Omega_{nS \rightarrow nP} = \frac{1}{2}.
\label{mix}
\end{equation}
For the electric field this becomes
\begin{equation}
E_0 = \frac{1}{2 d_{nS \rightarrow nP}}.
\end{equation}
We can turn this expression into a threshold $n$ value rather than a threshold
electric field value by taking $d_{nS \rightarrow nP} \approx n^2$ using
\Eq{dnSnP} as
\begin{equation}
n_{th} = \sqrt{\frac{1}{2 E_0}}.
\label{nth}
\end{equation}
This expression identifies the minimum $n$ value for a given electric field
above which $\ell$-mixing occurs.

For complete population transfer to the final state, the pulse must have an area
of at least $\pi$.  This implies a relationship between the peak electric field
and the pulse duration.  If we assume an electric field of the form $E(t) = E_0
f(t) ({\rm e}^{-i\omega_L t} + c.c.)$ where $f(t)$ is Gaussian with a FWHM of
$\tau_p$, then for a $\pi$-pulse, this relationship takes the form (in a.u.)
\begin{equation}
E_0 = \frac{\sqrt{\pi \log_e(2)}}{d_{gn} \tau_p}.
\label{pi_pulse}
\end{equation} 
In this expression, $d_{gn}$ is the dipole moment connecting the ground state 
to the $n$ Rydberg state.   Clearly, for very long pulses, the peak electric
field will be very small and we may expect that $n_{th}$ will be very
large.  In general, however, because of the $n^{-3/2}$ scaling of the
ground-to-Rydberg dipole moment, the electric field must increase for
increasing $n$ (for fixed pulse duration).  Then, because the threshold field for
$\ell$-mixing decreases with increasing $n$, there will always be an $n_{th}$
above which $\ell$-mixing occurs.  Note also that since $d_{gn}$ depends on $n$,
\Eq{nth} is only implicitly solved for $n_{th}$.

The particular pulse duration used in a given experiment will be determined by
several factors.  Because this pulse duration is one of the parameters that
determines $n_{th}$, we shall consider three different pulse
durations and evaluate the corresponding $n_{th}$.  This will determine, for
fixed pulse duration and desired $n$ level, whether $\ell$-mixing will affect
the interaction.  The three time scales we consider for the pulse duration are 
the spontaneous emission lifetime, the Stark period in a typical stray dc
electric field, and the Kepler period.  The spontaneous emission lifetime places
an upper limit on the pulse duration consistent with complete population
transfer.  The Stark period in a stray dc electric field imposes a practical
upper limit, given that stray electric fields persist in every experimental
configuration.  And finally, a pulse whose duration is the Kepler period is the
shortest pulse consistent with exciting a single $n$ state.

Typical radiative lifetimes for low-$\ell$ Rydberg states are 10~$\mu$s at $n
\sim 30$ with $n$-scaling of $n^3$ (see Table~\ref{table1}).  The
ground-to-Rydberg dipole moment in hydrogen can be evaluated analytically to be
$d_{gn} \approx 1.3/n^{3/2}$ \cite{Bethe_Salpeter}.  For a pulse with
duration given by this radiative lifetime, \Eq{pi_pulse} gives the threshold
electric field to be $E_0 \approx 7.4 \times 10^{-8} n^{-3/2}$.  Notice that this
electric field decreases with $n$. The threshold $\ell$-mixing state decreases
as $n^{-2}$ so we still predict a threshold $n$ value for $\ell$-mixing.  Using
this result and \Eq{nth} we obtain
$n_{th} \approx 10^{14}$.  Clearly, $\ell$-mixing will not be an important
interaction for pulses with this maximum possible duration.

We now consider a shorter pulse duration which is derived from more practical
considerations.  Though we have considered that the atom is perturbed
only by an optical frequency field, in any real experimental situation, stray dc
electric fields persist. Even when care is taken to ground interaction region
metal plates, stray electric fields of $\sim 1$ mV/cm are typically present
\cite{Frey:93}.  These electric fields can cause mixing of the nearly degenerate
manifold of Rydberg states.  In the energy domain, the requirement that the
angular momentum states are not mixed by the electric field corresponds to
having a pulse whose Fourier bandwidth overlaps the entire Stark-split manifold.
Then the optical pulse samples a short time interval during which the effect of
the stray field cannot be resolved.  In hydrogen, the total energy spread is
given by $\omega_s = 3 n (n-1) F$ with $F$ the stray electric field
amplitude.  Therefore the pulse must be shorter than the mixing time, defined as
\begin{eqnarray}
T_s & = & \frac{2 \pi}{\omega_s} \\
    & = & \frac{2 \pi}{3 n (n-1) F}. \label{Ts}
\end{eqnarray}
A stray field amplitude of $1$ mV/cm gives a mixing period of
$T_s = 0.3$ $\mu s$ at $n = 30$ and has an $n$-scaling of $n^{-2}$.

If the pulse duration is constrained to be much less than $T_s$, say $(1/10)
T_s$ to be concrete, then the effect of the stray dc field can be ignored.
If we use this pulse duration in \Eq{pi_pulse} we obtain an electric field of
$E_0 \approx 1.0 \times 10^{-12} n^{7/2}$.  This electric field increases with
$n$ and hence the threshold $n$ should be lower.  Using \Eq{nth} we calculate
$n_{th} \approx 130$.  We see then that if the practical consideration of
avoiding the dc Stark mixing of Rydberg states is considered, the threshold for
$\ell$-mixing is dramatically reduced.

The final pulse duration we consider is the Kepler period, $T_K = 2\pi n^3$. 
From \Eq{pi_pulse} this gives an electric field of $0.18 n^{-3/2}$.  Again, this
decreases with $n$ at a rate which is slower than the decrease in the threshold
field for $\ell$-mixing, and hence a threshold $n$ will exist.  From \Eq{nth}
this threshold is $n_{th} \approx 8$.  This shortest pulse consistent with
exciting a single $n$ level then causes $\ell$-mixing of Rydberg states with $n
> 8$.  It must be emphasized that these threshold $n$ values assume the pulse
has an area of $\pi$.  If efficient transfer of population is not required, the
threshold $n$ values can be increased significantly.

\section{Theoretical Model}
\label{theory}
\hspace{\parindent}  We have seen that over a broad range of parameters,
$\ell$-mixing of Rydberg states can be induced by optical fields.  This mixing
clearly degrades the selectivity of the excitation process. If a single
\{$n,\ell,m$\} state is desired, then coupling between $\ell$ and $\ell \pm
1$ states will cause a range of angular momenta to be populated.  In this
section we shall examine this coupling to quantify the extent to which it
can degrade the Rydberg selection process.  But from a broader standpoint,
we shall examine the nature of this interaction independent of any
particular excitation goals and we shall show that some rather interesting
dynamics can occur in these strongly driven Rydberg systems.

\begin{figure}[tbp]
\postfull{thesisfigs/direct/mix_levels.eps}
\bigskip
\ncap{Model of $\ell$-mixing interaction.} {Model of $\ell$-mixing interaction.
\label{mix_levels}}
\end{figure}

The model we consider is shown in \Fig{mix_levels}.  A ground state is
near resonantly coupled to a Rydberg manifold by a laser frequency
$\omega_L$.  The detuning of the ground-to-Rydberg resonance is given by
$\delta \ll \omega_L$ (we also assume that $\delta < \delta_n$ so that
we are considering the manifold closest to resonance).  The detuning for
transitions within the Rydberg manifold is
$\Delta = \omega_L$.  We showed in \Fig{inten} that as the intensity is increased, the
$\ell$-mixing coupling becomes important before $n$-mixing.  Furthermore,
the critical intensities for $\ell$- and $n$-mixing have different
asymptotic scalings.  For the ground-to-Rydberg Rabi frequency, $\Omega_{gn}
= (d_0 E_0)/n^{3/2}$, to equal the local Rydberg level spacing, $\delta_n =
1/n^3$, requires an electric field scaling of $E_0 \sim n^{-3/2}$.  For the
case of $\ell$-mixing interactions, for which the Rabi frequency must equal the
laser frequency, the $n^2$ scaling of the Rydberg dipole moments gives $E_0
\sim n^{-2}$.  Therefore, for sufficiently large $n$, the
coupling of the ground-to-Rydberg state is perturbative, while the
Rydberg-to-Rydberg states are strongly mixed.

The next consideration is the presence of neighboring $n$-manifolds not
shown in the figure.  The influence of these states can be estimated by making
use of Heisenberg's correspondence principle \cite{Picart:78}.  This principle
relates quantum mechanical matrix elements to Fourier components of the
corresponding classical dynamical variable.  If we consider $\Delta n = n -
n'$ transitions with $\ell \ll n$, then the correspondence principal estimates
the ratio of these dipole moments to the $\Delta n = 0$ transitions as
\begin{equation}
\frac{d(\Delta n)}{d(\Delta n = 0)} = \frac{2}{3 \Delta n} J_{\Delta
n}^{'}(\Delta n)
\end{equation}
with $J_{\Delta n}^{'}(\Delta n)$ the derivative of the Bessel function of the
first kind, order $\Delta n$.  For $\Delta n = 1$ this number is $\approx 0.2$
and for $\Delta n = 2$ it is $\approx 0.07$.  The ratio continues to fall for
increasing $\Delta n$.  We therefore estimate the coupling between
neighboring manifolds to be significantly smaller than the degenerate
couplings, and thus not critical in our model of these degenerate
transitions.

Another way to estimate the influence of these neighboring levels comes from
the ac Stark effect.  In the presence of the optical field, level
\{$n,\ell$\} will have its energy shifted due to its non-resonant
interaction with all levels of the atom.  The contributions to this shift
from states degenerate in energy with the \{$n,\ell$\} state cancel.  This
is because the ``resonant'' and ``anti-resonant'' terms have equal and opposite
contributions because of the degeneracy of the levels.  Therefore only
non-degenerate levels contribute to this shift.  However, the magnitude of
the contribution from level $n^{'}$ decreases rapidly as $n^{'}$ moves far
away in energy from $n$ (provided $n \gg 1$).  This is due to the fact that
the oscillator strength of level \{$n,\ell$\} is concentrated near
\{$n,\ell$\} for $n \gg 1$.  Therefore, the ac Stark shift of
\{$n,\ell$\} Rydberg states is predominantly determined by nearby manifolds. 
However, the ac Stark shift of Rydberg states is also known to be closely
approximated by the ponderomotive energy of a free electron
\cite{Avan:76,Obrian:94}.  This is due to the quasi-free nature of Rydberg
electrons.  Using the ponderomotive energy to estimate the ac Stark shift of
the Rydberg states is then an estimate of the influence of neighboring
$n$-manifolds on the \{$n,\ell$\} state.  The ponderomotive energy is given
by $ U_p = (1/4)E_0^2/\omega_L^2$ independent of $n$ \cite{Avan:76}.  If we
use the threshold field strength for $\ell$-mixing in this expression, we
find that
\begin{eqnarray}
U_p &\approx & 1/(4n^4)  \\
    & \sim   & \frac{1}{n} \delta_n.
\end{eqnarray}
Since this shift is asymptotically small in terms of the local level spacing,
we can again conclude that the contribution from neighboring $n$-manifolds
may be ignored.

The time-dependent wavefunction of our atom expanded on these essential
states is
\begin{equation}
\ket{\Psi(t)} = c_g(t)\ket{g} + \sum_{\ell} c_{\ell}(t)\ket{n,\ell}.
\end{equation}
The Schr\"{o}dinger equation for the amplitudes of the Rydberg states,
$c_\ell$, is then given by
\begin{equation}
i\dot{c}_\ell = \omega_n c_\ell + f(t)\cos (\omega_L t)
\left[\delta_{\ell1}\Omega_g c_g +
\sum_{\ell^{'}=0}^{n-1} \Omega_{\ell \ell^{'}} c_{\ell^{'}}\right]
\label{schrodinger_eqn}
\end{equation} 
where $c_g$ is the ground state amplitude, $\omega_n$ is the
ground-to-Rydberg transition frequency, $\delta_{\ell1}$ is the Kronecker
delta indicating coupling to the $P (\ell=1)$ state only, and
$\Omega_{\ell\ell^{'}}$ is the Rydberg-to-Rydberg Rabi frequency. The applied
optical field is taken to be linearly polarized in the $z$-direction and given
by $E(t) = E_0 f(t) \cos(\omega_L t)$, with $f(t)$ the laser pulse envelope
and $E_0$ the amplitude.  $\Omega_g$ is the ground-to-Rydberg Rabi frequency.
As described previously, the ground-to-Rydberg coupling can be treated
perturbatively in regimes in which the Rydberg states are strongly mixed.
Hence, at this stage we abandon the goal of complete population transfer to
the Rydberg state.  We do this so that we may investigate the nature of the
Rydberg $\ell$-mixing, with the knowledge that if it is important when a
small amount of population is transferred, it will certainly be important
when the intensity is increased to produce complete population transfer.
So at this stage, we assume that the ground state is weakly perturbed
and set $c_g \approx 1$ (taking the ground state as the zero of
energy). Because the Rabi frequencies, $\Omega_{\ell\ell^{'}}$, are greater
than $\omega_L$ we can not make the rotating-wave approximation
\cite{Piraux:89}.  We can however, diagonalize the Rydberg-to-Rydberg coupling
by transforming to the parabolic state basis,
\begin{equation}
b_k = \sum_{\ell=0}^{n-1} S_{k\ell} c_\ell.
\end{equation}
These states are labeled by the electric quantum number
$k\ \epsilon\ \{-n+1,-n+3,\ldots,n-3,n-1\}$.  $S_{k\ell}$
is the unitary transformation matrix whose components are proportional to the
Wigner-$3j$ symbol \cite{Gallagher_book} and is given by
\begin{eqnarray}
S_{k\ell} &=& \braket{n,k,m}{n,\ell,m} \\
              &=& (-1)^{(1-n+m+k)/2 + \ell} \sqrt{2\ell+1}
\left( 
\begin{array}{ccc}
\frac{n-1}{2} & \frac{n-1}{2} & \ell  \\
\frac{m+k}{2} & \frac{m-k}{2} & -m
\end{array} \right)  \\
&=& S_{\ell k}
\end{eqnarray}
where the term in large parentheses is the Wigner-$3j$ symbol.
The last equality follows from the fact that the Wigner-$3j$ symbols are
real.  The expense of this transformation is that every parabolic state is
coupled to the ground state and has a time-dependent homogeneous term,
\begin{equation}
i \dot{b}_k = \left[\omega_n + \Delta_k f(t)
\cos (\omega_L t)\right] b_k  
+S_{k1} \Omega_g f(t) \cos (\omega_L t) \label{eq:parabolic}.
\end{equation}
The term $\Delta_k = (3/2) n k E_0$ is the Stark shift of state $k$
that a dc field of amplitude $E_0$ would generate (within the
single $n$-manifold approximation).  $S_{k1}$ represents the
projection of the $k$-th parabolic state onto the $P$ state.
 
Assuming that $\delta = \omega_n - \omega_L$ is much less than the laser
frequency, $\omega_L$, and that the pulse contains many optical cycles
so that $\omega_L \tau_p \gg 1$ then \Eq{eq:parabolic} can be integrated
yielding the solution
\begin{eqnarray}
b_k(t)&=&-i \Omega_g S_{k1} \exp \left[-i\omega_n t -i
\frac{\Delta_k}{\omega_L} f(t)
\sin (\omega_L t)\right] \nonumber \\ 
&& \times \int_{-\infty}^{t} dt^{'} \frac{\omega_L}{\Delta_k}
\exp \left(i \delta t^{'}\right) J_{1} \left[\frac{\Delta_k}{\omega_L}
f\left(t^{'}\right)
\right] \label{eq:harmonics}
\end{eqnarray}
where $J_1$ is the Bessel function of the
first kind, order one.  For hyperbolic secant pulses, $f(t) =
\mathrm{sech}(t/\tau_p)$, in the long-time limit, the solution can be expressed
in terms of confluent hypergeometric functions \cite{Gradshteyn_Ryzhik},
\begin{eqnarray}
\lim_{t \rightarrow \infty} b_k(t) e^{i \omega_n t} & = &
\frac{-i \pi \Omega_g \tau_p e^{i \Delta_k/\omega_L}}
{2 \cosh(\pi \delta \tau_p/2)} S_{k1}
\Biggl\{
{}_1F_1\left(1/2-i\delta \tau_p/2,1,-i\Delta_k/\omega_L\right) 
\Biggr. \nonumber \\ 
&&\times {}_1F_1\left(1/2+i\delta \tau_p/2,1,-i\Delta_k/\omega_L\right) +
\left(\frac{\Delta_k}{\omega_L}\right)^2
\frac{\left[1+(\delta\tau_p)^2\right]}{2^4} \nonumber \\
&& \times {}_1F_1\left(3/2-i\delta \tau_p/2,3,-i\Delta_k/\omega_L\right)
\nonumber \\
&& \Biggl. \times {}_1F_1\left(3/2+i\delta
\tau_p/2,3,-i\Delta_k/\omega_L\right)
\Biggr\} \label{bigequation}
\end{eqnarray}
where $_1F_1$ is the confluent hypergeometric function.
For $\delta = 0$ this solution simplifies considerably as
\begin{equation}
\lim_{t \rightarrow \infty} b_k(t) e^{i \omega_n t}  = 
\frac{ -i\pi \Omega_g \tau_p}{2} S_{k1}
\left\{\left[J_0\left(\frac{\Delta_k}{2 \omega_L}\right)\right]^2+
       \left[J_1\left(\frac{\Delta_k}{2 \omega_L}\right)\right]^2 \right\}
\label{delta0}
\end{equation}
where $J_0$ and $J_1$ are Bessel functions of the first kind.
The amplitudes in the spherical basis can be obtained by unitary
transformation as 
\begin{equation}
c_\ell = \sum_{k=-n+1}^{n-1} S_{\ell k} b_k
\label{unitary}
\end{equation}
with the sum being over $k$ odd if $n-1$ is odd and over $k$ even otherwise. 
In what follows, we examine some of the predictions of this model.  Unless
otherwise stated we will use the results for the hyperbolic secant pulse.  We
have, however, verified that the qualitative results are insensitive to the
particular (smooth) pulse shape.


\section{Amplitudes}
\hspace{\parindent}  The first question to ask is to what extent is population
redistributed among higher angular momentum states?  In order to answer this
question we introduce a field strength parameter,
\begin{equation}
\beta = \frac{3 n (n-1) E_0}{2 \omega_L},
\end{equation}
which is the maximum value of the Stark shift, $\Delta_k$, measured in terms
of the laser frequency.  To verify the theory, we first consider the small
$\beta$ limit.  Then the term in \Eq{delta0} in curly brackets becomes equal
to one.  The amplitude of each parabolic state is then only determined by its
projection onto the $P$-state.  If we transform back to the spherical basis,
we find
\begin{eqnarray}
\lim_{t \rightarrow \infty} c_\ell e^{i \omega_n t} &=&
\frac{-i \pi \Omega_g \tau_p}{2} \sum_{k} S_{\ell k} S_{k1} \nonumber \\
& = &  \frac{-i \pi \Omega_g \tau_p}{2} \delta_{l1}.
\end{eqnarray}
The total population in the Rydberg manifold is in the
$P$-state as expected and furthermore, since
${\displaystyle \int}_{-\infty}^{\infty}\mathrm{sech}(t/\tau_p) dt = \pi \tau_p$,
we see that the total population transferred is consistent with the perturbative
ground-to-Rydberg coupling.  This small-$\beta$ limit provides a convenient
normalization for subsequent analysis.  Amplitudes will be normalized by $\pi
\Omega_g \tau_p/2$,
\begin{eqnarray}
\tilde{b}_k &=& \frac{b_k}{\left(\pi \Omega_g \tau_p/2\right)} \\
\tilde{c}_{\ell} &=& \frac{c_{\ell}}{\left(\pi \Omega_g \tau_p/2\right)},
\label{normalize}
\end{eqnarray}
so that in the small-$\beta$ limit, the $P$-state would have normalized
population equal to one.

\begin{figure}[tbp]
\postfull{thesisfigs/direct/ampk.eps}
\bigskip
\wcap{Amplitudes of the parabolic basis states.} {Amplitudes of the parabolic
basis states for various values of the strength parameter $\beta$.
\label{ampk}}
\end{figure}

\begin{figure}[tbp]
\postfull{thesisfigs/direct/ampl.eps}
\bigskip
\wcap{Amplitudes of the spherical basis states.} {Amplitudes of the spherical
basis states for various values of the strength parameter $\beta$.
\label{ampl}}
\end{figure}

Now we consider larger $\beta$ distributions and focus first on the amplitudes
of the parabolic basis states.  \Figure{ampk} shows the amplitudes of the
states in the parabolic basis at the end of the pulse.  The distributions are
symmetric in the electric quantum number, $k$, and hence only positive $k$ is
shown.  The strength parameter $\beta$ is varied from 2 to 16 and is doubled at
each step.  For this calculation we chose $n = 30$.  The amplitudes of the
states are normalized as mentioned previously.  For $\beta=0$, the amplitudes
are determined entirely by the transformation matrix, $S_{k1}$.  We see
that for $\beta = 2$ the distribution is highly linear in $k$
with only a slight curvature seen near the middle of the range in $k$. For
larger values of $\beta$ the overall magnitude of the states starts to decrease
and some oscillation as a function of $k$ begins to develop.  The amplitude
remains small for $k=0$ even for the maximum value of $\beta$, where the other
states have very similar amplitude (ignoring the small oscillation).
 
\Figure{ampl} shows the amplitude of the spherical basis states obtained by
transformation using \Eq{unitary} of the parabolic state distributions shown
in \Fig{ampk}.  Several interesting features emerge upon examination of
\Fig{ampl}.  One is that only states with the same parity as the $P$ state are
populated at the end of the pulse.  Furthermore, the maximum $\ell$ with
significant amplitude is given approximately by $\beta$.  Given that a single
parity has population at the end of the pulse, one might predict that the
physical mechanism giving rise to the angular momentum mixing is
$\Delta \ell = 2$ Raman transitions \cite{Grobe:86,Ivanov:94}.  That is, the
$P$-state is populated by direct excitation from the ground state and its
population flows to higher angular momentum states via resonant two-photon Raman
transitions to states with the same parity as the $P$-state.  However, we have
already argued that the threshold for {\it neighboring} state transitions would
be when the Rabi frequency is equal to the detuning, or laser frequency.  Given
that these states have a non-zero detuning tells us that even for enormous Rabi
frequencies, if the pulse is turned on and off adiabatically (in this case
meaning over many optical cycles) then the detuned states will not be
populated in the long-time limit.  However, they can receive significant
transient population, as shown in \Fig{ampl_t}.  In this figure, the
amplitude of the Rydberg states is shown near the peak of the laser pulse,
which occurs at $t=0$.  The $t=0$ distribution looks qualitatively similar to
the distribution at the end of the pulse.  But this is not true for
the remaining distributions shown, which are snapshots every quarter of an
optical cycle after $t=0$.  Both parities have significant population during the
pulse. Indeed, the $t = \frac{1}{4}\left(\frac{2\pi}{\omega_L}\right)$
distribution has the $S$-state maximally populated and the $t =
\frac{1}{2}\left(\frac{2\pi}{\omega_L}\right)$ distribution is
generally smooth showing a gradual decay towards larger angular momentum. 
These figures emphasize that the nearest neighbor coupling is the dominant
coupling and not a $\Delta\ell = 2$ Raman coupling via other states (continuum
or lower-lying bound).

  
\begin{figure}[tbp]
\postfull{thesisfigs/direct/ampl_t.eps}
\bigskip
\wcap{Amplitudes of the spherical basis states at various times.} {Amplitudes of
the spherical basis states for $\beta = 8$ and for various times near the peak
of the pulse at $t=0$.
\label{ampl_t}}
\end{figure}


\section{Angular Distributions}
\hspace{\parindent}

Given that population is spread to higher angular momentum states, it is
natural to investigate the resultant angular distributions.  The
electronic wavefunction can be written
\begin{equation}
\psi(\vec{r},t) = \sum_{\ell} c_{\ell}(t) R_{n \ell}(r)
Y_{\ell}^{m}(\theta,\phi).
\end{equation}
Because linear polarization is assumed we take $m=0$.  Since the radial part
of the wavefunction depends on $\ell$, strictly speaking the angular
distribution will depend upon what radius we are interested in.  This
dependence, however, can be shown to be very simple for a large part of the
$r$-range.  The WKB radial wavefunctions are given by \cite{Bethe_Salpeter}
\begin{equation}
R_{n \ell}(r) = \sqrt{\frac{2}{\pi n^3}}\left(\frac{r}{2}\right)^{1/4} \cos
\left[\sqrt{8 r} - (2 \ell +1)\frac{\pi}{2} - \frac{\pi}{4}\right].
\end{equation}
This expression is valid between the inner and outer turning points of the
corresponding classical electron's motion, given by
\begin{equation}
r_{1,2} = n^2 \pm n\sqrt{n^2 - \ell (\ell+1)}
\end{equation}
with $r_1$ the perihelion of the orbit, and $r_2$ the aphelion.  As long as
$\ell \ll n$ this range essentially encompasses the entire orbit.  While we are
considering $\ell$ larger than this in our problem, these wavefunctions still
give valid results for a large fraction of the orbit.  Furthermore,
outside of the turning points the wavefunctions are exponentially decaying
and thus very little probability resides beyond the turning points.  Under this
approximation then the $\ell$-dependence reduces to
$R_{n \ell} \propto (-1)^{\ell}$.  The wavefunction then becomes
\begin{eqnarray}
\psi(r,\theta,t) &\approx& f(r) \sum_{\ell} c_{\ell}(t) (-1)^{\ell}
Y_{\ell}^{0}(\theta,\phi) \\
& = & f(r) \chi(\theta,t)
\end{eqnarray}
with $f(r)$ containing the radial dependence and $\chi(\theta,t)$ the desired
angular distribution.

\begin{figure}[tbp]
\postfull{thesisfigs/direct/angular.eps}
\bigskip
\wcap{Angular distributions.} {Angular distributions of the Rydberg
population for various values of the strength parameter $\beta$.  The linear
polarization direction is shown in each with the polar angle indicated in the
first.
\label{angular}}
\end{figure}

\Figure{angular} shows the probability distribution,
$\left|\chi(\theta,t)\right|^2$, for various values of the strength parameter,
$\beta$, at the end of the pulse, $t \rightarrow \infty$.  For $\beta=2$ we see
an essentially undistorted
$P$-wave angular distribution.  This is to be expected, for when $\beta=2$,
very little amplitude is transferred to higher angular momentum states and the
$P$-state is the only state receiving population.  As $\beta$ increases, the
distribution starts to become less extended along the laser polarization axis,
and more spread out orthogonal to this direction.  This is in distinction with
photoelectron angular distributions obtained with very intense lasers in
non-Rydberg systems.  There, for increasing laser field strength, the
distribution becomes more peaked along the polarization direction.  This is
consistent with the tunneling theory of ionization, which becomes more valid
for increasing intensity and decreasing laser frequency.  In this model, the
electrons escape along the polarization direction because the atomic potential
is suppressed by the oscillating electric field such that the electron can
escape through a narrow barrier.

However, we are considering bound state angular distributions here, at
much lower intensities.  To date there is no method for accurately measuring a
bound state angular distribution, but the ionization from these distorted
states should show evidence of this distribution.  Since photoionization from
Rydberg states strongly favors small-$\ell$ states (for optical frequency
fields), the photoelectron distribution will appear more as a $P$-state than
as the bound electron distribution.  However, the strong tendency of the bound
state to spread orthogonal to the polarization direction should cause the
photoelectron distribution to bulge in this direction.  This type of
distribution was observed experimentally by Leuchs \etal \cite{Leuchs:85}. In
that experiment, the angular distributions of photoionized electrons were
measured as an intermediate resonance was tuned through the Rydberg series. For
small $n$ the distribution was peaked along the polarization direction. As
$n$ was increased, the distribution, while still peaked along the
polarization, began to bulge in the orthogonal direction.  As $n$ increases
for a fixed field strength, $\beta$ increases and we predict that the bound
state distribution becomes elongated perpendicular to the polarization axis,
but since photoionization cross-sections decrease rapidly with increasing
$\ell$, mostly the small $\ell$-component of the wavefunction will be
photoionized.  This leads to a photoion distribution which remains peaked
along the polarization direction but begins to show increased probability in
the orthogonal direction, as seen in \cite{Leuchs:85}. A very similar
distribution was observed in a high intensity experiment in xenon
\cite{Yang:93}.  Evidence for this type of distribution was also seen
in a theoretical study of the classical hydrogen atom in a strong
electromagnetic field \cite{Kyrala:87}.

The distributions shown are at the end of the pulse.  Since we showed
previously that the coefficients $c_{\ell}(t)$ change on the time scale of the
optical frequency, it may appear that the angular distribution must change on
this time scale as well.  While this is true for the wavefunction, it is not
true for the probability distribution.  That is to say that the changes in the
wavefunction are in its phase and not in its amplitude.  The probability
changes on the much longer time scale of the pulse, looking at early times
like the $P$-state orbital and smoothly changing into the distributions shown
in \Fig{angular} at the end of the pulse.

\section{Stabilization}
\hspace{\parindent}

\begin{figure}[tbp]
\postfull{thesisfigs/direct/pop_v_beta.eps}
\bigskip
\ncap{Population versus $\beta$ (fixed ground coupling).} {Total normalized
population in Rydberg manifold versus $\beta$.  The ground-to-Rydberg coupling
is kept constant for this figure.
\label{popvbeta}}
\end{figure}

\begin{figure}[tbp]
\postfull{thesisfigs/direct/pop_v_I.eps}
\bigskip
\ncap{Population versus $\beta$.} {Total population
in Rydberg manifold versus $\beta$.  The ground-to-Rydberg coupling increases
as the electric field ($\beta$) increases.
\label{popvI}}
\end{figure}

Now we consider the total population transferred to the Rydberg manifold.  The
strong mixing of Rydberg states leads to a change in the effective coupling of
the (dressed) manifold with the ground state.  In \Fig{popvbeta} we show the
total normalized population (normalization as in \Eq{normalize}) transferred to
the Rydberg manifold as a function of
$\beta$.  In this plot the coupling of the ground state to the $P$-state is
kept constant.  This plot shows that the dressed manifold has
decreasing overlap with the $P$-state as $\beta$ increases.

This variation in $\beta$ is somewhat artificial in that an increase in
$\beta$ can occur either by increasing $n$ or by increasing electric field.
In the former case, the ground-to-$P$-state coupling decreases and in the
latter it increases.  However, the figure shows that the dressed manifold has
decreased coupling with the ground state than the ``undressed'' manifold.  In
\Fig{popvI} we consider the effect of increasing the electric field.  This
causes $\beta$ to increase and the ground-to-$P$-state coupling to increase
as well.  The figure shows that for small $\beta$, the population goes to zero as
expected. As $\beta$ increases initially, the population grows, until $\beta
\sim 4$ at which the slope changes sign and the population begins to decrease
with increasing $\beta$.  This is a bound state analog of dynamic laser
induced stabilization \cite{Burnett:91,Huens:93} (as opposed to adiabatic, or
high frequency, stabilization \cite{Pont:88}).  This theory predicts that as
the laser intensity is increased (under certain circumstances) the
photoionization can become a decreasing function of laser intensity.  The
conditions for achieving this stabilization can depend on the laser frequency
and the initial state.  Stabilization in Rydberg systems has been
predicted \cite{Fedorov:90}.  In this model, a strong, low frequency field causes
$n$-mixing of neighboring $n$-manifolds via the continuum which can give rise
to a stabilized wave function, due in large part to the formation of a dark
state wave packet.  The low frequency condition of this model causes the
$n$-mixing via the continuum to be an important interaction.  In optical
frequency fields, however, this interaction is negligible compared to the
$l$-mixing because the photoionization cross-sections are so small.

This bound state stabilization should give rise to traditional photoionization
stabilization for the following reasons.  The photoionization signal from the
Rydberg states depends on the population in the Rydberg states.  Thus if the
population decreases the ion signal should decrease as well.  Also, since the
population which is in the Rydberg manifold is moving towards larger angular
momentum states that have smaller ionization cross sections, the signal should
also decrease due to the reduced effective cross-section.  These two effects
combined should give rise to a more dramatic stabilization figure than that
shown in \Fig{popvI}.

\section{Lineshape}
\hspace{\parindent}

The solution obtained in \Eq{bigequation} is for arbitrary detuning of the
laser field from the Rydberg manifold.  Given that the strong field mixing of
Rydberg states can change the strength of the coupling to the ground state, we
might expect the absorption lineshape to also change.  The validation of this
expectation is shown in \Fig{lineshpe} which depicts the absorption lineshape
for various values of $\beta$.  When $\beta$ is small, the lineshape is
completely determined by the Fourier transform of the hyperbolic secant pulse
which is shown as the dashed line in each graph.  For $\beta=4$ we see a
slightly narrower distribution with essentially the same shape as the weak
field case. For $\beta=8$ and $\beta=16$, however, the lineshape shows
modulation structure but generally occupies a width in frequency space
given by the weak field envelope.

\begin{figure}[tbp]
\postfull{thesisfigs/direct/lineshpe.eps}
\bigskip
\wcap{Rydberg manifold absorption lineshape.} {Absorption lineshape of
Rydberg manifold for various values of strength parameter $\beta$.  The
dashed line shows the Fourier transform of hyperbolic secant pulse.  The
dimensionless frequency units are the detuning, $\delta$, multiplied by the
pulse duration, $\tau_p$.
\label{lineshpe}}
\end{figure}

\section{High Harmonic Generation}
\hspace{\parindent}

Another interesting consequence of this model can be seen by examination of
\Eq{eq:harmonics}.  If the exponential in front of the integral is expanded in a
Fourier series, we obtain the result
\begin{eqnarray}
b_k(t) &=& -i \Omega_g S_{k1} \exp(-i \omega_n t) \sum_{q= -\infty}^{\infty}
J_q\left[\frac{\Delta_k}{\omega_L} f(t)\right] \exp(iq\omega_L t)
\nonumber \\
&& \times \int_{-\infty}^{t} dt^{'} \frac{\omega_L}{\Delta_k}
\exp \left(i \delta t^{'}\right) J_{1} \left[\frac{\Delta_k}{\omega_L}
f\left(t^{'}\right)
\right] \nonumber \\
& =& g_k(t) \exp(-i \omega_n t) \sum_{q=-\infty}^{\infty}
J_q\left[\frac{\Delta_k}{\omega_L} f(t)\right] \exp(iq\omega_L t)
\end{eqnarray}
with $g_k(t)$ a slowly varying function of time.
The parabolic state amplitudes therefore oscillate at harmonics of the driving
field, with amplitudes determined by Bessel functions.  We may therefore expect
that the time dependent dipole moment, which is a bilinear product of the ground
and excited state amplitudes, will also show oscillation at these harmonics and
hence the Rydberg system may emit radiation at the harmonics of the driving
field.  If we write the electron wavefunction as
\begin{equation}
\ket{\psi(t)} = c_g(t)\ket{g} + \sum_{\ell}c_{\ell}(t)\ket{\ell},
\end{equation}
then the time dependent dipole moment is
\begin{eqnarray}
d(t) &=& \bra{\psi(t)}\hat{d}\ket{\psi(t)} \nonumber \\
&=& c_g(t) \sum_{\ell}c_{\ell}^{*}(t)\bra{\ell}\hat{d}\ket{g} + \sum_{\ell
\ell^{'}} c_{\ell}(t)c_{\ell^{'}}^{*}(t) \bra{\ell^{'}}\hat{d}\ket{\ell}
+ c.c.
\label{dipole}
\end{eqnarray}
Since the ground-to-Rydberg coupling is treated perturbatively, the ground
state amplitude is near one and thus the Rydberg amplitudes are small.  The
second term in \Eq{dipole} is therefore much smaller than the first and can be
dropped. Furthermore, the dipole selection rules allow us to reduce the sum to
\begin{eqnarray}
d(t) &=& c_g(t) c_1(t)d_{1g} + c.c. \nonumber \\
&=& c_1(t)d_{1g} + c.c.
\end{eqnarray}
with $d_{1g}$ the ground-to-$P$-state dipole moment, and
where we have used $c_g(t) \approx 1$, consistent with the perturbative
assumption (and the zero of energy taken to be the ground state).  Then our
time-dependent dipole moment is (for $\delta =0$)
\begin{eqnarray}
d(t) &=& \sum_k S_{1k} b_k(t) \nonumber \\
     &=& \sum_k S_{1k} (-i \Omega_g) S_{k1} \exp \left[-i\omega_n t 
\right] \int_{-\infty}^{t} dt^{'} \frac{\omega_L}{\Delta_k}
\exp \left(i \delta t^{'}\right) J_{1} \left[\frac{\Delta_k}{\omega_L}
f\left(t^{'}\right)
\right] \nonumber \\
&& \times \sum_{q= -\infty}^{\infty}
J_q\left[\frac{\Delta_k}{\omega_L} f(t)\right] \exp(iq\omega_L t)
\nonumber \\
&=& \sum_{q=-\infty}^{\infty}\Biggl\{ \sum_k (-i) \Omega_g S_{1k}^2
J_q\left[\frac{\Delta_k}{\omega_L} f(t)\right]
\int_{-\infty}^{t} dt^{'} \frac{\omega_L}{\Delta_k}
 J_{1} \left[\frac{\Delta_k}{\omega_L}
f\left(t^{'}\right) \right] \Biggr\} \nonumber \\
&& \times  \exp\left[i(q\omega_L-\omega_n)t\right]. \label{eq:harmonics2}
\end{eqnarray}
Before turning to numerical evaluation of this expression, we can make some
qualitative predictions.  The symmetry of the parabolic states
enables us to analyze the amplitude of each harmonic term in the $q$
expansion.  The parabolic terms naturally separate into pairs of equal
$\left|k\right|$.
$S_{1k}^2$ and the time integral are even functions of $k$,
because $\Delta_{-k}=-\Delta_{k}$, and from the parity of the Bessel functions
\begin{equation}
J_q\left[\frac{\Delta_{-k}}{\omega_L} f(t)\right] =
(-1)^{q} J_q\left[\frac{\Delta_{k}}{\omega_L} f(t)\right].
\end{equation}
Therefore, all contributions to the $q^{th}$ amplitude cancel if $q$ is odd. 
Since the exponential contains $\omega_n = \omega_L$ (for $\delta=0$), this
means that only odd harmonics are present.  The case $k=0$ (which only occurs
when $n$ is odd) does not contribute because $S_{1k}=0$ for $k=0$.  Therefore
the dipole moment contains only odd harmonic contributions.  Also, given that the
$q^{th}$ harmonic has amplitude proportional to
$J_q\left[\frac{\Delta_k}{\omega_L} f(t)\right]$, this amplitude will be small
for all $k$ unless $\beta \geq q$.  We therefore predict odd harmonics with
maximal spectral content $\omega \approx \beta \omega_L$.

\begin{figure}[tbp]
\postfull{thesisfigs/direct/harmonics.eps}
\bigskip
\wcap{Atomic dipole moment spectrum.} {Fourier spectrum of the time dependent
atomic dipole moment for several values of strength parameter
$\beta$.
\label{fig:harmonics}}
\end{figure}

Numerical results of evaluating the Fourier transform of \Eq{eq:harmonics2} are
shown in \Fig{fig:harmonics}.  They confirm our expectations that only odd
harmonics are present and that the majority of the spectral content is
contained for $\omega < \beta \omega_L$.  The shape of the spectra displays what
might be called the typical high harmonic generation curve \cite{Gavrila:92}. 
The harmonics have a rapid decay from the fundamental, then a relatively flat
plateau region extends to a cut-off, above which the harmonic amplitudes fall
rapidly to the background level.  High harmonic generation has been observed
when very intense laser fields ($I \sim 10^{13}$ W/cm$^2$) are incident on noble
gas atoms and has been predicted in several models, ranging from full three
dimensional calculations of Schr\"{o}dinger's equation \cite{Kulander:89} to an
analysis of a two-level atom in a strong field \cite{Sundaram:90,Burlon:96}. 
Here, we see the possibility for high harmonic generation from these strongly
driven Rydberg systems.  However, the most interesting feature of this
interaction, from the standpoint of short wavelength generation, is that the
required intensity is much lower than that needed in the noble gas experiments. 
For example, the harmonic generation spectrum shown for $\beta = 32$ is very
similar to an experiment with argon atoms illuminated by 3 $\times 10^{13}$
W/cm$^2$ Nd:YAG light \cite{Li:89}.  To achieve the $\beta=32$ spectrum for
$n=65$ requires a factor of 50 lower intensity. If higher $n$ could be used the
required intensity would be even lower. 

\section{Landau-Zener Model}
\hspace{\parindent}
In this section, we attempt to elucidate some of the previous
results by modeling the interaction using Landau-Zener (LZ) theory
\cite{Zener:32,Landau:32,Rubbmark:81}.
LZ theory describes the evolution of a two-state quantum system for which the
energy separation between the two levels varies with an external parameter (e.g.
time).  In the original form of the theory, the two levels were taken to have an
energy separation of $\omega = +\infty$ at $t=-\infty$ and $\omega = - \infty$ at
$t=\infty$. The levels are degenerate at $t=0$.  If there is no
coupling between the levels, then the levels cross without any effects of one
impacting the evolution of the other.  However, when the levels are coupled,
there is an avoided crossing in which the coupling serves to prevent exact
degeneracy during the time evolution.  These avoided crossings are such that if
the system begins in one state and evolves adiabatically (defined
precisely below) then the state moves through the avoided crossing so that at the
end of the interaction, the system is in the opposite state.  This interaction
is characterized by a single parameter given by
\begin{equation}
\Lambda =\frac{|\left\langle 1 \right| \hat{V} \left|2\right\rangle|^2}{d \omega/dt},
\end{equation}
where $\hat{V}$ is the interaction term coupling state
$\left|1\right\rangle$ to $\left|2\right\rangle$ and $\omega(t)$ is the energy
difference between the two states \cite{Rubbmark:81}.  The probability of an
adiabatic transition can be shown to be $P_a = 1 - {\rm e}^{-2 \pi
\Lambda}$.  The condition for adiabaticity is therefore $2 \pi \Lambda \gg 1$. 
LZ theory has previously been applied in Rydberg systems when the energy
shift is due to slowly varying dc electric fields \cite{Rubbmark:81} and also
in the case of avoided crossings of Rydberg Floquet levels due to large ground
state Stark shifts \cite{Story:93,Story:94}.

\begin{figure}[tbp]
\postfull{thesisfigs/direct/lz1.eps}
\bigskip
\ncap{Landau-Zener model of $\ell$-mixing.} {Landau-Zener model of the
$\ell$-mixing interaction.  The energy of the parabolic states shift linearly
with the applied optical electric field such that they experience avoided
crossings with the ground state twice during one optical cycle.
\label{lz1}}
\end{figure}

This simple two-level model can be applied to our $\ell$-mixing interaction by
considering each parabolic state coupled to the ground state as individual
two-level problems, as shown in \Fig{lz1}.  This figure is analogous to
the Stark maps of Rydberg atoms in a dc electric field \cite{Zimmerman:79}.   In
that problem, a dc electric field (of given polarity) causes the energy levels of
the Rydberg states to shift by an amount that increases with the electric field. 
Here, the shifting is due to an optical frequency electric field, and hence both
polarities are present.  The $x$-axis in the figure is therefore the optical
electric field.  The parabolic states have their energies shifted by this field,
with the sign of the shift determined by the sign of the electric field.  The
ground state is unshifted. The energy of the $k^{th}$ parabolic state
(within the single $n$-manifold approximation used here) is given by the
homogeneous term in \Eq{eq:parabolic},
\begin{equation}
\omega(t) = \omega_n + \Delta_k f(t) \cos(\omega_L t) .
\end{equation} This function oscillates between $\omega_n - \Delta_k f(t)$ and
$\omega_n + \Delta_k f(t)$ during the optical cycle.  The function $f(t)$ is a
slowly varying envelope function and so is nearly constant for several optical
cycles.  Notice that this term can equal zero twice during one optical cycle.
At these times the ground and parabolic states are degenerate (in the absence
of coupling).  The presence of the coupling term in \Eq{eq:parabolic} causes
these levels to repel and leads to an avoided crossing, shown in \Fig{lz1}.
Therefore we may formulate the LZ theory of our problem for each optical cycle,
allowing for free evolution, then avoided-crossing, then free evolution followed
by a second avoided crossing.  In this picture, population can be transferred
from the ground state to the parabolic state by two separate routes in one
optical cycle.  \Figure{lz2} depicts these two transfer routes. The population
may first make a diabatic traversal followed by an adiabatic transition, or
an adiabatic transition followed by a diabatic traversal.  The fact that two
independent routes can lead to the same final state is the hallmark of quantum
interference and we may expect to find such effects here.  This interference is
due to the accumulation of phase between avoided crossings and is the analogue of
St\"{u}ckelberg oscillations in inelastic atomic collisions
\cite{Stuckelberg:32}.  It has been recently observed in microwave
multiphoton transitions \cite{Gatzke:95}.


\begin{figure}[tbp]
\postfull{thesisfigs/direct/lz2.eps}
\bigskip
\ncap{Two Landau-Zener population transfer routes.} {The two possible
population transfer routes per optical cycle are shown for a given parabolic
state interacting with the ground state.
\label{lz2}}
\end{figure}

For simplicity, we formulate our model for square pulses, though recent studies
have shown how LZ theory can be extended to include the effects of a smoothly
varying pulse \cite{Vitanov:96}.  The crossing that occurs closest to
$t=0$ is given by
\begin{equation}
t_c = \frac{1}{\omega_L}\cos^{-1}(\frac{-\omega_n}{\Delta_k}).
\label{tc}
\end{equation}
The next crossing occurs within the same optical cycle at time $2
\pi/\omega_L- t_c$. By adding $\frac{2\pi}{\omega_L}$ to these two numbers we
obtain the crossing times for the next optical cycle, and so on for subsequent
cycles.  Clearly, for values of $k$ such that $\Delta_k <
\omega_n$ there will be no crossings and our model will predict no
population in those parabolic states.

The evolution from time $t_1 \rightarrow t_2$, during which no avoided crossings
occur, can be described using the time evolution operator
\begin{equation}
\hat{U}(t_2,t_1) = \left[\begin{array}{cc}
               \exp(-i\int_{t_1}^{t_2}\omega(t)dt) & 0 \\
	      0 & 0
	     \end{array}\right]
\end{equation}
where we have taken the ground state to be the zero of energy.
The transition matrix for the avoided crossing is \cite{Gatzke:95}
\begin{equation}
\hat{T}_1 = \left[\begin{array}{cc}
               \sqrt{1 - P_a} & -\sqrt{P_a} e^{i\phi} \\
	       \sqrt{P_a} e^{-i\phi} & \sqrt{1 - P_a}
	    \end{array} \right]
\end{equation}
where $\phi$ is the phase associated with the LZ transition and is given by
\cite{Garraway:92}
\begin{equation}
\phi = \frac{-\pi}{4} + \Lambda \log_e \Lambda -\Lambda -
{\rm arg}\left[\Gamma(\Lambda)\right]
\label{phi}
\end{equation}
with ${\rm arg}\left[\Gamma(\Lambda)\right]$ the phase of the gamma function
evaluated at $\Lambda$.
Now, this transition matrix assumes that the states cross in a
particular order (i.e. one from above, the other from below).  If they cross
in the opposite order, the transition matrix is slightly different and is given
by
\begin{equation}
\hat{T}_2 = \left[\begin{array}{cc}
               \sqrt{1 - P_a} & \sqrt{P_a} e^{-i\phi} \\
	       -\sqrt{P_a} e^{i\phi} & \sqrt{1 - P_a}
	    \end{array} \right].
\end{equation}
This change from $\hat{T}_1 \rightarrow \hat{T}_2$ can be effected by letting
the phase change as $\phi \rightarrow -\phi \pm \pi$.

We now calculate the evolution for one optical cycle with the
time evolution from $t=0 \rightarrow 2\pi/\omega_L$.  The single cycle
evolution operator is
\begin{equation}
\hat{U}_1 =
\hat{U}\left(\frac{2\pi}{\omega_L},\frac{2\pi}{\omega_L}-t_c\right)\hat{T}_2
\hat{U}\left(\frac{2\pi}{\omega_L} -t_c,t_c\right)\hat{T}_1\hat{U}(t_c,0).
\end{equation}

The analysis thus far has been for an arbitrary two-level system that exhibits
periodic level crossings mediated by an ac modulated detuning.  However, at this
point we specialize to our particular problem and use our knowledge of our
problem's adiabaticity.  The matrix element of the interaction term coupling
our two states is
\begin{equation}
V = S_{k1} \Omega_g \cos(\omega_L t).
\end{equation}
This term is modulated at the optical frequency but we will only need
its value at the crossing times.  Also the time rate of change of
the frequency separation is
\begin{equation}
\frac{d\omega}{dt} = \Delta_k \omega_L \sin(\omega_L t),
\end{equation}
which also will be evaluated at the crossing times. Therefore we may evaluate
$\Lambda$ as
\begin{equation}
\Lambda_k = \frac{S_{k1}^2 \Omega_g^2 \cos^2(\omega_L t_c)}{\left|\Delta_k
\omega_L \sin(\omega_L t_c)\right| }.
\end{equation}
$\Omega_g$ is a small perturbative quantity and $\Delta_k \approx \omega_L$.
Therefore, $\Lambda_k \ll 1$ and our transition is overwhelmingly diabatic.
That is, $P_a \approx 2 \pi \Lambda_k \ll 1$.  Keeping to lowest order in $P_a$
the single cycle transition matrix is
\begin{equation}
\hat{U}_1 \approx \left[\begin{array}{cc}
               e^{-i 2 \pi x} & \Psi_k \sqrt{2 \pi \Lambda_k} \\
	       \Psi_k \sqrt{2 \pi \Lambda_k} & 1
	     \end{array}\right]
\end{equation}
with
\begin{eqnarray}
x &=& \frac{\omega_n}{\omega_L}, \\
\Psi_k &=& \exp\left[-i(\phi_k -\alpha_k+2\pi x)\right] - \exp\left[i
(\phi_k -\alpha_k)\right],
\end{eqnarray}
where $\phi_k$ is defined as in \Eq{phi} and
\begin{equation}
\alpha_k = \omega_n t_c + \frac{\Delta_k}{\omega_L}\sin(\omega_L t_c).
\end{equation}
The $N$-cycle evolution matrix can be calculated, again to lowest order in
$P_a$, as
\begin{equation}
\hat{U}_N \approx \left[\begin{array}{cc}
    e^{-i2\pi N x} & \Psi_k \sqrt{2 \pi \Lambda_k}
    {\displaystyle \sum_{j=0}^{N-1}e^{-i j 2 \pi x}}  \\
    \Psi_k \sqrt{2 \pi \Lambda_k}{\displaystyle \sum_{j=0}^{N-1}e^{-i j
    2 \pi x}} & 1
    \end{array}\right].
\end{equation}
If we assume $\delta = \omega_n - \omega_L \ll \omega_L$, then $ x \approx 1$
and we can write the $N$-cycle time evolution matrix as
\begin{equation}
\hat{U}_N \approx
\left[\begin{array}{cc}
 e^{-i\delta \tau_p} & N \Psi_k \sqrt{2 \pi \Lambda_k}
 {\displaystyle \frac{\sin(\delta \tau_p/2)}{(\delta \tau_p/2)}}  \\
 N \Psi_k \sqrt{2 \pi \Lambda_k}{\displaystyle \frac{\sin(\delta
 \tau_p/2)}{(\delta\tau_p/2)}} & 1
\end{array}\right]
\end{equation}
with
\begin{equation}
\Psi_k \approx 2 i \sin(\alpha_k - \phi_k)
\end{equation}
for $x \approx 1$. The probability of a transition, $k \leftarrow g$,
is then given by
\begin{equation}
P_{k \leftarrow g} = N^2 8 \pi \Lambda_k \sin^2(\alpha_k-\phi_k)
\frac{\sin^2(\delta \tau_p/2)}{(\delta \tau_p/2)^2}
\label{lzprob}
\end{equation}

This last equation represents the main result of our LZ calculation.  It has
several interesting features that deserve elaboration.  The first is the
factor $N^2$.  While each transition probability is small, each small one
adds coherently to produce a significant population transfer, depending on the
number of optical cycles.  This many cycle periodic avoided crossing has
recently been examined in the microwave domain \cite{Watkins:96}.  The second
feature to notice is the presence of the detuning term.  This is the traditional
weak field absorption response due to a square pulse.  This shows that the
coherent-many-cycle addition of the small transition probabilities will only add
up to an appreciable value if the detuning is within a reciprocal pulse width of
resonance.  This helps to justify ignoring the neighboring $n$ manifolds.
One might wonder if our LZ theory is flawed because if level $n$ shifts
through the ground state and gets populated, then the shift of level $n
\pm 1$ is approximately the same and hence they too should be populated.  The LZ
theory, however, shows that even though these states have avoided crossings
with the ground state, they accumulate phase at a slightly different rate so
that many small amplitudes add to give a still small transition probability,
with the smallness determined by the detuning and the pulse width.  This
theory does not predict the distorted lineshape that was seen for hyperbolic
secant pulses previously.  However, we saw there that the width of the
absorption feature was not significantly modified from its weak field value,
and this feature is reproduced in the LZ square pulse model.
A second feature of this result is the presence of the $\sin(\alpha_k -
\phi_k)$ term.  This represents the interference of the two population
transfer routes during one cycle depicted in \Fig{lz2}.

A quantitative comparison of the LZ theory (dashed line) with the predictions
of \Eq{eq:harmonics} for a square pulse (solid line) are shown in \Fig{poplz}.
In this case, the population in the parabolic states are shown versus
the electric quantum number $k$ for increasing values of $\beta$. 
The distribution is symmetric in $k$ and hence only positive $k$ values are
shown.  For $\beta=8$ we see excellent agreement between the two models for $k
>17$.  For this value of $\beta$, we see from \Eq{tc} that the crossing time
is imaginary for $\left| k \right| < 8$.  That is, these states do not have
avoided crossings with the ground state.  Our LZ theory tacitly assumes that each
state has avoided crossings with the ground state, and hence, this model
predicts zero population for these states, as shown in the figure. The LZ
theory also assumes that the evolution time can be well separated into
crossing and free evolution periods.  This will not be true for those states
that have avoided crossings with the ground state near the extreme of their
motion.  Therefore, we see that  until $k \sim 17$ the model does not predict
quantitatively the population transfer.  Above this value, the agreement is
excellent.

\begin{figure}[tbp]
\postfull{thesisfigs/direct/pop_lz.eps}
\bigskip
\wcap{Landau-Zener predictions of parabolic state populations} {The
predictions of the Landau-Zener theory (dashed line) is compared to the
predictions of \Eq{eq:harmonics} (solid line) for a square pulse and various
values of $\beta$. The principal quantum number is taken to be $n=60$.
\label{poplz}}
\end{figure}


For $\beta=16$, the no-crossing cut-off is for $k < 4$ and the agreement
becomes quantitatively accurate for $k > 9$.  And finally, for $\beta=32$ the
no crossing cut-off is $k < 2$ and here the quantitative agreement
essentially extends over the entire $k$ range. 
Typical applications of LZ theory include Rydberg atoms in dc or
microwave electric fields, and recently, in strong optical fields where the ac
Stark shift gives rise to avoided crossings between Floquet states.  In this
last example, the time scale of the state shifting is the optical pulse
length.  In our case, it is the optical period.  It is perhaps surprising then
that this model should work so accurately in such an unorthodox application.
Yet the conditions for applicability of the LZ theory are straightforward.  The
levels must be separated by $\infty$ (infinite energy) at early times, the shift
must vary linearly with the time (or another parameter), and
they must be separated by $-\infty$ at large positive times.  These conditions
are closely approximated in our problem.  The levels are separated by an
amount $\sim \omega_L$ at early times and experience an avoided crossing of
width $\sim \Omega_g \ll
\omega_L$ so that on the scale of the avoided crossing, the levels appear to
be separated by $\infty$ at early times.  The levels also shift linearly with
the optical electric field.  And finally, provided the avoided crossing
does not occur near the extreme of the motion, the levels appear to be
separated by $-\infty$ for long times.  And indeed, the agreement shown in
\Fig{poplz} is excellent for those states that meet these requirements.

\section{Conclusions}
\hspace{\parindent}

In this chapter, we delineated regimes under which efficient excitation of
Rydberg states can be achieved by single-photon coupling from the ground
state. We have found that one of the main limitations is that the strong
electric field needed to transfer the population to the Rydberg state is also
strong enough to induce $\ell$-mixing of the Rydberg states.  The interaction
degrades the selectivity of the excitation process, but from a broader point
of view, the interaction gives rise to some exciting predictions for these
Rydberg atoms in strong optical fields.

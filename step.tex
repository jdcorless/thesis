\begin{singlespace}

\chapter{Direct Excitation: Experiment}
\label{step}
\begpagestyle
%\pagenumbering{arabic}

\end{singlespace}


In this chapter, we shall discuss the results of our experimental
investigation of the optical mixing of Rydberg angular momenta.  The
experiment consisted of tuning a picosecond dye laser through two-photon
resonance with the Rydberg series in potassium.  A third photon from this
laser ionizes these Rydberg states.  We investigate the ionization signal
for varying laser intensities, thus controlling the amount of angular momentum
mixing of the Rydberg states.  We compare these experimental results with two
theoretical calculations.  The first models many of the subtleties of the
potassium atom using a traditional approach from multiphoton physics
\cite{Lambropoulos:92}, but neglects the Rydberg-Rydberg coupling.  The second
approach ignores the details of the potassium atom, but models the
Rydberg-Rydberg mixing with the theory presented in \rChapter{direct}.  We
find that the main features observed in the experiment are reproduced only
when Rydberg-Rydberg mixing is included.


\section{Experimental System}
\hspace{\parindent}
The laser and atomic beam system used in the experiments described below
are based on the designs of Yeazell \cite{Yeazell_thesis} and were used by Noel
\cite{Noel_thesis}.  A schematic diagram of the set-up is shown in
\Fig{gensetup}. The experiment utilized a synchronously-pumped modelocked
dye laser \cite{Wisoff:85}.  This laser was pumped by the frequency doubled
output of a mode-locked and Q-switched Nd:YAG laser \cite{Dawson:84}
manufactured by Quantronix, Model 416.  The laser produced a Q-switched burst
of mode-locked pulses.  Approximately 25 mode-locked pulses were contained
within each Q-switch envelope.  The pulses have duration $\approx 130$~ps and
were separated by the cavity round trip time of 13~ns.  The Q-switch
repetition rate was 500 Hz and the average frequency doubled power was
630~mW.

\Figure{ps_laser} shows the cavity design of the mode-locked dye laser.  The
main cavity is formed by a flat high-reflector, M1, and a grating, G.  The
cavity used has the ``Z'' configuration, utilizing folding mirrors M2 and
M3.  By varying the separation between the folding mirrors, the size of the
beam near the gain medium could be adjusted.  This affects the minimum
pulse length attainable with this laser \cite{Wisoff:85}.  The grating was
mounted on a rotation stage that was driven by a stepper motor so that
rotating the grating with the stepper motor changed the laser wavelength.  The
laser dye was a flowing solution of $4 \times 10^{-4}$~M Rhodamine 6G in
methanol.  The dye laser's wavelength range was 560-580~nm with a peak near the
two-photon ionization limit of potassium at 571.2~nm.  The prism pair (P1 and
P2) enabled the size of the dye beam at the grating to be adjusted so that
sufficient dispersion could be added to the cavity to obtain near
transform-limited performance.  Note that too much dispersion, caused by
over-filling the grating, leads to a reduction in the laser bandwidth and a
subsequent increase in the laser pulse duration.

\begin{figure}[tbp]
\postfull{thesisfigs/step/gensetup.eps}
\bigskip
\wcap{Experimental setup for Rydberg interaction.}
{Experimental setup for Rydberg interaction.  The synchronously-pumped
picosecond dye laser was pumped by the frequency-doubled output of a modelocked
and Q-switched (MLQS) Nd:YAG laser.  The dye laser interacted with an atomic
beam and propagated to a spectrometer.
\label{gensetup}}
\end{figure}

The dye laser produces a burst of mode-locked pulses much like the pump
beam.  The dye laser does not reach threshold, however, until
after approximately 10 pump pulses.  Consequently, the dye laser output consists
of a train of about 15 modelocked pulses.  To select a single pulse for the
experiment, we utilized a cavity-dumper consisting of a Pockels cell and a
polarizing beam splitter.  When a certain optical threshold is reached within
the dye cavity, a high voltage pulse is sent to the Pockels cell which produces a
$\lambda/4$ retardation for the oscillating dye beam.  The double-pass
polarization rotation causes the pulse to be rejected out of the cavity after
reflecting off the polarizing beam splitter.  By matching the high voltage
pulse duration to the cavity round-trip time, a single
optical pulse can be switched out of the cavity.  This technique also
efficiently extracts the energy circulating in the cavity so that the brightest
pulse can be obtained from this system.  The dye pulse output has an energy of
$\approx 30~\mu$J and a pulse duration of 25~ps, producing a peak power of
$\approx 1$~MW.

\begin{figure}[tbp]
\postfull{thesisfigs/step/pslaser.eps}
\bigskip
\ncap{Picosecond dye laser.}
{Picosecond dye laser.  The cavity is formed by mirror M1 and grating G.  The
spot size at the dye cell DC could be adjusted by changing the separation
between the 10~cm folding mirrors M2 and M3.  The pump beam was focused onto
the dye cell with a 40~cm lens (L).  A single pulse could be cavity dumped from
the laser with a Pockels cell (PC) and polarizing beam splitter (BS). The
grating was mounted on a rotation stage which was controlled by a stepper motor
(SM).
\label{ps_laser}}
\end{figure}

The laser beam was focussed with a $f=250$~mm lens into a vacuum chamber that
contained a potassium atomic beam.  The vacuum chamber was pumped by an
Alcatel Model 5400 turbomolecular pump with a Welch DuoSeal Model 1402
two-stage rotary forepump.  Potassium metal, contained in a stainless steel
oven, was heated to $\sim 300^{\circ}$C.  The potassium vapor effused out
of the oven and into a heated nozzle, placed near the interaction region.  This
essentially produced a ``fire hose'' of potassium atoms instead of a well
collimated atomic beam.  The advantage of this design is that the density of
potassium atoms could be made larger than for the collimated beam.  The laser
pulses used have a bandwidth of 40~GHz, so Doppler broadening is not
important in this interaction and a high degree of collimation is not required. 
The reason a vapor cell was not used is that the electron multiplier tube (EMT)
used to count photoions requires a clean vacuum environment, so the vapor must
be confined within the chamber.

The intersection of the laser and atomic beams was contained between two
grounded metal plates.  Following the laser-atom interaction, a negative
voltage was applied to the top metal plate.  This accelerated any photoions
generated by the laser pulse to the EMT mounted above the interaction
region.  In addition to accelerating any photoions, the electric field could
also dc-field ionize bound Rydberg states provided their binding energy was
below a certain level.  We used an accelerating field of 2.1~V/cm, which was
sufficient to ionize Rydberg states of potassium with $n^* > 110$.

The signal from the EMT was amplified and sent to a Stanford Research
Systems SR400 Gated Photon Counter, which in turn was connected to a personal
computer via a Hewlett-Packard HPIB (IEEE-488 compatible) interface bus.  This
computer, through Data Translation DT2801A digital I/O ports, also controlled
the stepper motor which was mounted to the diffraction grating in the dye
laser.  A single step of the stepper motor gave a wavelength change of
0.1~\AA.  This way, the photoion signal could be measured as the dye laser's
wavelength was tuned through the Rydberg series.  The dye laser light propagated
out of the vacuum chamber and into a Jarrell-Ash one-meter Czerny-Turner
spectrometer \cite{Sawyer_book}.  This device was calibrated with a sodium lamp
to an absolute frequency accuracy of approximately 3~cm$^{-1}$, and has a
resolution of 0.4~cm$^{-1}$.

\section{Experimental Results}
\hspace{\parindent}  The relevant energy structure of potassium is shown in
\Fig{K_levels}.  The ground state is brought into two-photon resonance with the
$n^*S$ and $n^*D$ Rydberg series with an applied laser field of
wavelength $\lambda \approx$ 571~nm.  The energy from a single photon
falls between the $\ket{4P}$ and $\ket{5P}$ states, with detunings that are a
large fraction of the photon energy.  Consequently, these intermediate $P$
states receive very little population.  The population in the Rydberg states is
photoionized by the applied laser field into the $\varepsilon P$ and
$\varepsilon F$ continua.  While we have only shown two intermediate $P$
states, other $P$ states contribute to the multiphoton couplings.  However, in
evaluating the multiphoton parameters, it can be shown that accurate results
are obtained despite only including these low-lying $P$ states \cite{Allen:82}.
This being the case, the states shown in \Fig{K_levels} represent the basis of
essential states that typically are included in a model of this
interaction
\cite{Fedorov:89a,Gratl:89,Yeazell:90,Raczynski:93,MIvanov:94,Wojcik:95}.

\begin{figure}[tbp]
\postfull{thesisfigs/step/K_levels.eps}
\bigskip
\ncap{Potassium energy levels.}
{Potassium energy levels coupled by picosecond dye laser.
\label{K_levels}}
\end{figure}

We showed in \rChapter{direct}, however, that the coupling between degenerate
Rydberg states can give rise to mixing of the angular momentum $\ell$.  The
laser intensity for which the $nS \rightarrow nP$ Rabi frequency is equal to
the ground $\rightarrow nP$ transition frequency was termed the
critical intensity in \rChapter{direct}.  This critical intensity for our
two-photon interaction in potassium is shown in \Fig{inten_K} as a function of
the principal quantum number.  For this plot, the Rydberg-Rydberg dipole moments
of potassium were calculated using the Coulomb approximation method of Bates and
Damgaard \cite{Bates:49} which was applied to Rydberg transitions by
Zimmerman \etal \cite{Zimmerman:79}.  We see in this figure that field strengths
$> 10^9$ W/cm$^2$ are sufficient to mix the angular momentum states of potassium with
$n^* > 40$.  These intensities are easily achieved with our focused picosecond
dye laser.  So we may expect that the standard essential states shown in
\Fig{K_levels} are not complete, and that the degenerate coupling of Rydberg
states should also be included.

\begin{figure}[tbp]
\postfull{thesisfigs/step/inten_K.eps}
\bigskip
\ncap{Critical intensity for $\ell$-mixing in potassium.}
{Critical intensity for $\ell$-mixing in potassium.
\label{inten_K}}
\end{figure}

We therefore sought to examine the ways in which this Rydberg-Rydberg coupling
could manifest itself in the three-photon ionization signal.  Of course, there
are some added complications intrinsic to modeling the experiment in
potassium that were not present in the theory for hydrogen.  Most
importantly, the two-photon coupling gives rise to an ac Stark shift of the
ground state, due to its off-resonant coupling with the low-lying $P$ states. 
This is an aspect not considered in the previous chapter.  However, it should be
stressed that the Rabi frequency coupling the Rydberg states exceeds the optical
frequency, and so we expect it to be a very important parameter in the
interaction of these Rydberg states with the applied optical field.

\begin{figure}[tbp]
\postfull{thesisfigs/step/weak.eps}
\bigskip
\ncap{Three-photon ionization signal as a function of two-photon detuning from
threshold for weak field.} {Three-photon ionization signal as a function of
two-photon detuning from threshold for weak field.
\label{weak}}
\end{figure}

The signal obtained under weak field excitation conditions is shown in
\Fig{weak}.  The abscissa is the two-photon detuning from the ionization
threshold, $\delta(\omega_L) = 2\omega_L -I_P$, where the ionization potential
is given by $I_P = 35009.78$~cm$^{-1}$ \cite{Moore:71}.  The corresponding
effective principal quantum number, $n^*$, can be obtained with the equation
\begin{equation}
\delta(\omega_L) = -\frac{R_K}{(n^*)^2}
\end{equation}
where $R_K = 109735.774$~cm$^{-1}$ is the Rydberg constant for potassium
\cite{Gallagher_book}.  The signal was obtained by averaging 8000 laser shots
(at the 500 Hz repetition rate) for each of 255 laser frequencies.  The general
shape of the curve is consistent with other weak field studies of two-photon
absorption in potassium \cite{Harper:77,Sharma:90}.  For frequencies far enough
below threshold, individual Rydberg states give rise to $2+1$ resonantly
enhanced photoionization.  The heights of successive peaks decrease with
increasing laser frequency,  because the two-photon ground to
Rydberg coupling scales as $(n^*)^{-3/2}$ as does the Rydberg photoionization
dipole moment.  This $n^*$-dependence is due to the scaling of the Rydberg
wavefunctions near the core.  The ground-to-Rydberg interaction couples a state
localized near the core to a spatially extended Rydberg state and hence the
coupling is determined by the scaling of the Rydberg state for small radius,
which is $(n^*)^{-3/2}$ \cite{Gallagher_book}.  The photoionization process also
predominantly takes place near the nucleus \cite{Alber:88} and therefore has
the same scaling.  The amount of population transferred in each step varies as
the square of these couplings for weak fields, and hence the resonances show
decreasing amplitudes that fall as $(n^*)^{-6}$ \cite{Gratl:89}.

As the laser frequency is tuned closer to threshold, the contrast between
neighboring Rydberg states is decreased as the coherent bandwidth of the 25~ps
laser pulse is large enough to overlap several Rydberg levels.  In this region,
a flat ionization signal is observed.  This can be understood by generalizing
the previous arguments to a density of final states.  When the levels can't be
resolved, the Rydberg excitation step, scaling as $(n^*)^{-3}$, gets multiplied
by the density of states, given by $(n^*)^{3}$, which is one over the frequency
separation.  The photoionization step is treated similarly and hence no
dependence on $n^*$ is expected, and to the level of the noise an essentially
flat signal is observed between -80~cm$^{-1}$ and -40~cm$^{-1}$.  A slight
negative slope is seen for frequencies greater than -40 cm$^{-1}$.

As the laser frequency closely approaches the two-photon threshold, the signal
is seen to rise rapidly off the scale chosen for the graph.  This is because in
this region, even though we are still below the ionization threshold, the
populated Rydberg states are dc-field ionized by the accelerating voltage.  For
the 2.1 V/cm field used, frequencies above -9 cm$^{-1}$ can be field ionized. 
This dc-field ionized signal rises by a factor of 150 relative to the
photoionized signal to approximately 4.5 counts/shot.  This shows that the
photoionization step is perturbative, ionizing approximately 1/150 $\approx$
0.7\% of the Rydberg population.  This weak coupling is consistent with the
$(n^*)^{-3}$ scaling.

\begin{figure}[tbp]
\postfull{thesisfigs/step/mix_expt.eps}
\bigskip
\wcap{Three-photon ionization signal as a function of two-photon detuning from
threshold for strong field.}
{Three-photon ionization signal as a function of two-photon detuning from
threshold for strong field.  In (a) the peak intensity is 2~GW/cm$^2$ and in
(b) 25~GW/cm$^{2}$.
\label{mix_expt}}
\end{figure}

This experiment was performed with a peak laser intensity of $\approx
0.25$~GW/cm$^2$.  The first resonance peak shown at -138~cm$^{-1}$ is located at
$n^* \approx$ 28.  By looking at \Fig{inten_K} we see that this intensity will
not mix Rydberg states until $n^* \approx 55$, which corresponds to a
-36~cm$^{-1}$ two-photon detuning.  This is where the slope  of the signal is
seen to become negative.  Therefore, even for this weak field case, we see some
evidence that the Rydberg mixing may be affecting the three-photon ionization.

\Figure{mix_expt} shows the three-photon ionization signal for larger
intensities.  \Fig{mix_expt}(a) is for a peak intensity of 2~GW/cm$^2$.  At
this intensity, Rydberg angular momentum mixing should occur for $n^* > 32$, or
frequencies above -107~cm$^{-1}$.  In this case we see that in the region where
Rydberg states are not resolved, the shape of the curve is not flat.  Rather,
a dramatic step-like fall-off is seen in the signal between -80~cm$^{-1}$ and
-40~cm$^{-1}$. Finally, when the field becomes very large, as in
\Fig{mix_expt}(b), where the peak intensity is 25~GW/cm$^2$, the ionization
signal shows a smooth decrease for increasing frequency.  This curve has a
completely different shape to the weak field result.  So the general tendency
we observe for increasing intensity is a flat signal with a slight negative
slope which turns into a step-like drop for larger intensity, which finally
turns into a large negative slope for the largest intensities.

\section{Theoretical Modeling}
\hspace{\parindent}  In this section we will argue that the
intensity-dependent shape changes of the three-photon ionization signal are
signatures of angular momentum mixing of Rydberg states.  We do so by comparing
our ionization signal to the predicted results from two theoretical models.  The
first, referred to as the multiphoton model, is
based on the traditional approach from multiphoton physics that ignores the
Rydberg-Rydberg coupling
\cite{Fedorov:89a,Gratl:89,Yeazell:90,Raczynski:93,MIvanov:94,Wojcik:95}.  It 
models the atom with the essential states shown in
\Fig{K_levels} and consists in solving the Schr\"{o}dinger equation for the
coupled state amplitudes in the presence of a laser pulse with parameters
determined by the experiment.  A standard approximation in this approach is to
adiabatically eliminate the low-lying $P$ states because of their rather large
detuning \cite{Allen:87}.  This has the effect of producing two-photon coupling
terms between the ground state and the Rydberg states and introducing the ac
Stark shift, a time-dependent shift of the ground state energy.

The multiphoton parameters needed for this model can be calculated using
tabulated experimental values of energy levels \cite{Moore:71} and transition
strengths \cite{Wiese:69}.  To calculate these parameters, we can limit
ourselves to the two intermediate $P$ states shown in \Fig{K_levels}, since the
contribution from higher $P$ states can be shown to be small \cite{Allen:82}. 
The Stark shift of the ground state is
\begin{eqnarray}
\Delta_s & = & - \sum_{j}|d_{4S \rightarrow j}|^2
\left(\frac{1}{\omega_{j}-\omega_L} + \frac{1}{\omega_{j}+ \omega_L}\right)
E_0^2  \label{stark_shift} \\ & \approx & - \sum_{j=4P,5P}|d_{4S
\rightarrow j}|^2
\left(\frac{1}{\omega_{j}-\omega_L} + \frac{1}{\omega_{j}+ \omega_L}\right)
E_0^2 \\ & = & 356 E_0^2 \ {\rm (a.u.)}.
\end{eqnarray}
In this expression, $\omega_{j}$ is the energy of the intermediate states.  For
$\lambda = 571$~nm, $\omega_L = 0.079$~a.u., and the dipole moments are
$d_{4S \rightarrow 4P} = 2.93$~a.u.\ and $d_{4S \rightarrow 5P} = 0.201$~a.u. 
For an intensity of 2~GW/cm$^2$, $\Delta_s/2\pi = 133$~GHz $= 4.4$~cm$^{-1}$.

The Rydberg states also experience a Stark shift which is due primarily to the
interaction between a given Rydberg state and nearby Rydberg states
\cite{Fedorov:89b}.  Consequently, in this treatment of the problem, the effect
of neighboring Rydberg levels is to produce an ac Stark shift of the Rydberg
energies.  The Stark shift of the $n^*S$ or $n^*D$ can be calculated
using \Eq{stark_shift} with $4S$ replaced by $n^*S$ or $n^*D$ respectively. 
This expression is derived using second order perturbation theory. 
Consequently, this treatment implicitly assumes that the Rydberg states weakly
interact with each other.  It has been shown \cite{Avan:76} that the Stark
shift of \Eq{stark_shift} for Rydberg states has a value closely approximated by
the ponderomotive energy of a free electron.  This ponderomotive shift of
Rydberg states has also been experimentally verified
\cite{Liberman:83,Obrian:94}. The expression for this ponderomotive shift (in
atomic units) is 
\begin{equation}
\Delta_p = \frac{E_0^2}{\omega_L^2}
\end{equation}
for a laser field of the form $E(t) = 2 E_0 \cos(\omega_L t)$.  For our field,
with $\omega_L =0.079$~a.u., this shift is $160E_0^2$, which for an
intensity of 2~GW/cm$^2$ is 60~GHz (or 2.0~cm$^{-1}$).
Rather than describing Stark shifts for both the ground state and Rydberg
state, we can combine both effects into an effective Stark shift for
the ground state only.  The effective ground state Stark shift will then be
given by the difference of the two shifts,
\begin{equation}
\overline{\Delta}_s = \Delta_s -\Delta_p = 196 E_0^2.
\end{equation}

The two-photon Rabi frequencies from the ground state are similarly calculated
according to the expression
\begin{eqnarray}
\Omega_{n^*D}^{(2)}& =
&-\sum_{j}\frac{d_{4S \rightarrow j}d_{j \rightarrow n^*D}}{\omega_j -
\omega_L}E_0^2 \\
& \approx & -\sum_{j=4P,5P}\frac{d_{4S \rightarrow j}d_{j
\rightarrow n^*D}}{\omega_j -
\omega_L}E_0^2 \\
& = & \frac{307}{(n^*)^{3/2}}E_0^2.
\end{eqnarray}
To arrive at this final expression, we have used the experimental values of the
dipole moments for the largest $n^*$ tabulated, and then assumed the
semiclassical dipole moment scaling of $(n^*)^{-3/2}$ for larger values
of $n^*$ \cite{Gallagher_book}.  The corresponding two-photon Rabi frequency
from the ground state to the $n^*S$ Rydberg series is
\begin{eqnarray}
\Omega_{n^*S}^{(2)}& =
&-\sum_{j}\frac{d_{4S \rightarrow j}d_{j \rightarrow n^*S}}{\omega_j -
\omega_L}E_0^2 \\
& \approx & -\sum_{j=4P,5P}\frac{d_{4S \rightarrow j}d_{j
\rightarrow n^*S}}{\omega_j -
\omega_L}E_0^2 \\
& = & \frac{296}{(n^*)^{3/2}}E_0^2.
\end{eqnarray}

The ionization cross-sections for the Rydberg series are derived from the work
of Aymar \etal \cite{Aymar:76}.  The $n^*S$ cross-section was shown in this
work to be $\approx 36$ times smaller in potassium than in hydrogen.  The $n^*D$
cross-section was shown to be determined predominantly by the $D \rightarrow F$
component, and to be approximately hydrogenic.  We calculate the hydrogenic
values using the exact expression for bound-bound dipole moments
\cite{Bethe_Salpeter}, and then extend these to bound-free dipole
moments by letting the principal quantum number $n \rightarrow -i/k$.  The
energy then changes as $(-1/2n^2) \rightarrow (k^2/2)$, which is the
positive energy of a free state.  This transformation gives us the correct
dipole moment because when the change $n \rightarrow -i/k$ is made to the bound
state wavefunction, the new wavefunction corresponds to a free state
wavefunction with energy $k^2/2$.  If the free state wavefunction is to be
normalized on the energy scale, however, the result of the $n \rightarrow
-i/k$ transformation to the wavefunction must be multiplied by an energy
dependent constant.  That is, if $B_{n\ell}$ is the bound state radial
wavefuntion, then the free state wavefunction, $F_{E\ell}$, with energy $E =
k^2/2$, is given by
\cite{Bethe_Salpeter}
\begin{equation}
F_{E\ell} = \frac{B_{(-i/k)\ell}}{(-1)^{\ell}(ik)^{3/2}\sqrt{1 -\exp(-2\pi/k)}}.
\end{equation}
Given the bound-bound dipole moments $d_{n\ell,n^{'}\ell^{'}}$, which can be
expressed in terms of hypergeometric functions
\cite{Bethe_Salpeter}, the bound-free dipole moment then is given by
\begin{equation}
d_{n\ell\rightarrow k\ell^{'}} = \frac{1}{(-1)^{\ell^{'}}(ik)^{3/2}\sqrt{1
-\exp(-2\pi/k)}} d_{n\ell \rightarrow \left(\frac{-i}{k}\right)\ell^{'}}
\end{equation}
for a final state with an energy of $k^2/2$ (a.u.) normalized on the energy
scale.  The Fermi decay rate is \cite{Gottfried_book}
\begin{equation}
\Gamma_{n\ell} = \pi \left(d_{n\ell \rightarrow
k\ell^{'}}\right)^2 E_0^2
\label{H_ion}
\end{equation}
and we calculate for potassium (in atomic units)
\begin{eqnarray}
\Gamma_{n^*S} &=& \frac{46.9}{(n^*)^3} E_0^2 \\
\Gamma_{n^*D} &=& \frac{729}{(n^*)^3}  E_0^2
\end{eqnarray}
where we have taken the final state energy to be equal to $ \omega_L
-1/2(n^*)^2$, which is the energy of the $n^*$ Rydberg state plus one $\omega_L
= 0.079$~a.u.\ photon energy.

The amplitudes associated with the continuum can be formally integrated and
substituted back into the equations of motion for the bound state amplitudes. 
If the dipole moments connecting bound states to free states are independent
of the energy of the free state (flat continuum approximation), then the effect
of this procedure is that the bound state amplitudes decay, which is just the
ionization process.  The resulting equations of motion become (in the rotating
wave approximation)
\begin{eqnarray}
i\dot{C}_{4S}& = &f^2(t)\overline{\Delta}_s C_{4S} +
f^2(t)\sum_{n^*}\left(\Omega^{(2)}_{n^*S}C_{n^*S} +
\Omega^{(2)}_{n^*D}C_{n^*D}\right) \\
i\dot{C}_{n^*S}& = & \left[\Delta_{n^*}-if^2(t)\Gamma_{n^*S}\right] C_{n^*S} +
f^2(t)\Omega^{(2)}_{n^*S} C_{4S} \\
i\dot{C}_{n^*D}& = & \left[\Delta_{n^*}-if^2(t)\Gamma_{n^*D}\right] C_{n^*D} +
f^2(t)\Omega^{(2)}_{n^*D} C_{4S}.
\end{eqnarray}
In these equations, $C_{j}$ is the amplitude of state $j$, $\Delta_{n^*} =
\omega_{n^*} - 2\omega_L$ with $\omega_{n^*}$ the energy of state $n^*$, and
$f(t)$ is the electric field pulse envelope taken to be $f(t) = \sin^2(\pi
t/\tau_p)$.  This is a convenient smooth pulse shape because it is zero at
$t=0$ and $t=\tau_p$ and hence the integration domain is finite.

\begin{figure}[tbp]
\postfull{thesisfigs/step/mix_thy1.eps}
\bigskip
\wcap{Calculated three-photon ionization signal as a function of two-photon
detuning from threshold using the multiphoton model.}
{Calculated three-photon ionization signal as a function of two-photon detuning
from threshold using the multiphoton model.  In (a) the peak intensity is
2~GW/cm$^2$ and in (b) 25~GW/cm$^{2}$.
\label{mix_thy1}}
\end{figure}

A consequence of integrating the equations of motion for the free state
amplitudes and making the flat continuum approximation is that the
normalization of the remaining amplitudes is not constant in time.  This
decaying normalization is due to the ionized population.  We therefore
calculate the total ionization at the end of the pulse as
\begin{equation}
{\rm ion} = 1 - \left|C_{4S}\right|^2 - \sum_{n^*} \left(
\left|C_{n^*S}\right|^2 + \left|C_{n^*D}\right|^2 \right).
\end{equation}
\Figure{mix_thy1} shows the results of this calculation for the two intensities
corresponding to the experimental parameters presented in \Fig{mix_expt}.  In
this theoretical calculation, we do not attempt to model the dc-field ionization
of populated Rydberg states because we are interested in the shape of the
three-photon ionization signal below this dc-field ionization feature.  The
agreement between the multiphoton theory and the experimental results is not
very strong.  The resonance peaks are similar but the shape of the curve in the
smooth, unresolved Rydberg state region is essentially flat in the theoretical
results.  The step-like fall-off seen in the experiment for $I=2$~GW/cm$^2$ is
not seen at all in the theoretical calculation shown in \Fig{mix_thy1}(a).  And
while the shape of the curve in
\Fig{mix_thy1}(b) shows a negative slope for increasing frequency, the slope is
not nearly as large as that seen in the corresponding experimental curve of
\Fig{mix_expt}(b).  Furthermore, the larger intensity curve of
\Fig{mix_thy1}(b) is in fact very similar to the lower intensity curve of
\Fig{mix_thy1}(a), which is not true of the experimental results of
\Fig{mix_expt}.

Because of these discrepancies, we sought to determine if the ion signals
seen in the experiment were affected by the optical mixing of Rydberg
states, which is neglected in the multiphoton model.  We now will present a
second model (which we will refer to as the mixing model) which includes
Rydberg mixing.  It will be built upon the analytic results of
\rChapter{direct}.  Since we are interested in the experimental results in the
regime where the pulse bandwidth overlaps many
$n$ levels, we cannot use the single $n$-manifold theory presented in
\rChapter{direct}.  However, we will make the simplest possible multiple-manifold
extension to this theory by assuming that the manifolds are not coupled to each
other.  This was one of the main approximations discussed in \rChapter{direct}. 
Then, to model the interaction when the laser pulse bandwidth overlaps many
$n$ levels, we simply multiply the single manifold result by the local density
of states, $(n^*)^3$.  As mentioned previously, this has the effect of cancelling
the $n$-dependence of the squared ground-to-Rydberg Rabi frequency in the
expression for the Rydberg population.

Since the ionized population is very small we will assume that the dynamics are
not modified by the ionization process.  Therefore, the time-dependent
coefficients of the Rydberg states can be calculated in the absence of
ionization.  Then, the ionization signal can be calculated from these
time-dependent amplitudes.  Since the theory presented in \rChapter{direct} is
implicitly hydrogenic, we will not attempt to modify it to potassium, but rather
assume that the Rydberg mixing phenomenon is insensitive to the particular
atomic structure.  The largest difference between the Rydberg states of
potassium and the Rydberg states of hydrogen is the quantum defect,
$\delta_{\ell}$, which prevents the angular momentum states from being
degenerate for small $\ell$.  In potassium,
$\delta_0 = 2.18, \delta_1 = 1.71$, and $\delta_2 = 0.277$
\cite{Gallagher_book}.  Therefore, the assumption of degenerate Rydberg states
will not be valid for potassium.  However, when we are considering Rabi
frequencies on the scale of the optical frequency, a lack of degeneracy on the
order of the Rydberg spacing should be entirely negligible.  We will therefore
use the hydrogen results calculated in \rChapter{direct} and use hydrogenic
ionization cross-sections calculated using \Eq{H_ion}.

We shall further simplify the model by neglecting the two-photon nature of the
coupling of the ground state to the Rydberg series and model it as a
single-photon coupling to the $P$ state.  The motivation here is that this
coupling is assumed weak and therefore will not dominate the dynamics.  Of
course, the Stark shift can impact the dynamics, but again, the largest
coupling is the Rydberg-to-Rydberg coupling.  Our motivation in using this
theory, that essentially only models the Rydberg mixing, is to see if this effect
predicts ionization curves similar to those observed experimentally.

The differential equations for the Rydberg state amplitudes get modified in
the presence of ionization to
\begin{equation}
i\dot{c}_{n \ell} = -if^2(t)\Gamma_{n \ell} c_{n \ell} + \left\{\cdots \right\}
\end{equation}
where the term in curly brackets, given by \Eq{schrodinger_eqn}, represents the
coupling among Rydberg states and the ground state, and was solved in
\rChapter{direct}. We calculate the ionization signal as follows.  The
Rydberg amplitudes, assumed to be essentially unperturbed by ionization,  can be
written (assuming resonance) using \Eq{unitary} and \Eq{eq:harmonics} as
\begin{equation}
c_{n\ell} = \sum_{k=-n+1}^{n-1} S_{\ell k} b_{nk}
\end{equation}
with
\begin{eqnarray}
b_{nk}(t)&=&-i \Omega_{gn} S_{k1} \exp \left[-i\omega_n t -i
\frac{\Delta_k}{\omega_L} f(t)
\sin (\omega_L t)\right] \nonumber \\ 
&& \times \int_{-\infty}^{t} dt^{'} \left(\frac{\omega_L}{\Delta_k}\right)
 J_{1} \left[\frac{\Delta_k}{\omega_L}
f\left(t^{'}\right) \right].
\end{eqnarray}
The ionization is then calculated as
\begin{equation}
{\rm ion}(E_n \ell) =(n)^6\left| \int_{-\infty}^{\infty} f^2(t)\Gamma_{n
\ell} c_{n \ell}(t) dt \right|^2.
\label{ion_nl}
\end{equation}
This integral represents the time-integrated change to the Rydberg amplitude
coefficient due to the presence of ionization, which is the desired ionization
signal.  The factor of $(n)^6$ in this expression is the square of the local
density of states.  As mentioned previously, to generalize the
bound-bound excitation step from a single Rydberg state to a continuum of
Rydberg states entails multiplying the single state result by the density of
states, $(n)^3$.  This is also true for the ionization step.  While we
calculate the ionization from a single Rydberg state, we generalize this to the
ionization from a continuum of Rydberg states by multiplying by the density of
states.  The total ion signal for a given $n$-manifold is then
\begin{equation}
{\rm ion}(E_n) = \sum_{\ell} {\rm ion}(E_n \ell).
\end{equation}

The integral in \Eq{ion_nl} has no known analytical evaluation.  We can
simplify it by analyzing the time scales of the integrand.  The Rydberg
coefficients vary in time on the scale of the optical frequency, whereas the
pulse envelope varies on the much longer time scale of the pulse duration. 
We can write the Rydberg coefficients as
\begin{equation}
c_{n\ell}(t) = \overline{c}_{n\ell}(t) + \Delta c_{n\ell}(t)
\end{equation}
where $\overline{c}_{n\ell}(t)$ is the slowly-varying component that is
time-averaged over the rapid optical frequency oscillation and
$\Delta c_{n\ell}(t)$ is the rapid optical frequency variation that has a zero
integrated value.  If we use this decomposition in \Eq{ion_nl} then the terms
in $\Delta c_{n\ell}(t)$ will be approximately zero because of the much longer
time variation of $f^2(t)$.  Therefore, the ionization signal will be
determined by the slow variation of the Rydberg amplitudes.  It can be shown that
the Rydberg states that have parity opposite to the $P$ state have zero
time-averaged amplitude, and hence do not contribute significantly to the
ionization signal.  We therefore predict that the dominant ionization signal
will come from states with the same parity as the $P$ state that do not have
zero time-averaged amplitudes.  Furthermore, we estimate the magnitude of the
time-averaged signal for these states to be equal to their value at the end of
the pulse.  We then have an approximate expression for the total ionization
signal as
\begin{eqnarray}
{\rm ion}(E_n) &=& n^6 \sum_{\ell\ {\rm
odd}} \left| \int_{-\infty}^{\infty}f^2(t) \Gamma_{n\ell}\overline{c}_{n\ell}(t)
\right|^2 \\
&\approx& n^6 (\tau_p^2) \sum_{\ell\ {\rm odd}}\left| \Gamma_{n\ell}
\overline{c}_{n\ell}(t \rightarrow \infty) \right|^2 
\end{eqnarray}
where $\overline{c}_{n\ell}(t \rightarrow \infty)$ is the
the population at the end of the pulse.  So that we may use the
analytic results obtained in \rChapter{direct} we will assume a hyperbolic secant
pulse and use the amplitude expressions from \Eq{delta0} and \Eq{unitary}.
This equation represents the result of our mixing model.  In many ways it is a
crude estimate of the full dynamics of potassium in an intense picosecond dye
laser field.  However, it is an accurate model of the Rydberg mixing in the
presence of weak ionization.  It is worth noting that the $n^6$ factor due to
the density of states exactly cancels the $n^{-3}$ dependence of the squared
ground-to-Rydberg Rabi frequency and the $n^{-3}$ dependence of the squared
Fermi decay rate.  The only $n$ dependence in this expression then is the optical
mixing which is implicit in $\overline{c}_{n\ell}$.  As one tunes higher into
the Rydberg series, the Rydberg-Rydberg coupling grows, and therefore, any
variation in this predicted ion signal will be due to the changing degree of
optical mixing.

\begin{figure}[tbp]
\postfull{thesisfigs/step/mix_thy2.eps}
\bigskip
\wcap{Calculated three-photon ionization signal as a function of two-photon
detuning from threshold using the mixing model.}
{Calculated three-photon ionization signal as a function of two-photon detuning
from threshold using the mixing model.  In (a) the peak intensity is
2~GW/cm$^2$ and in (b) 25~GW/cm$^{2}$.  The dashed line is the experimental
results from \Fig{mix_expt} and the solid line is the theoretical prediction.
\label{mix_thy2}}
\end{figure}

The predictions of this model are shown in \Fig{mix_thy2}.  The only remnant of
the potassium atom in this model is the conversion of the intensity to the
mixing parameter, $\beta$.  This parameter, as defined in \rChapter{direct}, is
given in hydrogen approximately by the $nS \rightarrow nP$ Rabi frequency
measured in terms of the $1S \rightarrow nP$ transition frequency.  We shall
extend this to potassium by taking the $\beta$ parameter to be equal to the
$n^*S \rightarrow n^*P$ Rabi frequency measured in terms of the
$4S \rightarrow n^*P$ transition frequency.  This is the essence of
\Fig{inten_K}, which uses this definition to define the threshold intensity for
optical mixing.  Therefore, the laser intensity, which is $I=2$~GW/cm$^2$ in
\Fig{mix_thy2}(a) and $I=25$~GW/cm$^2$ in \Fig{mix_thy2}(b), is converted into
a ($n^*$-dependent) $\beta$ value using the potassium dipole moments and twice
the dye laser frequency.

The shapes of these theoretical curves are very different from the predictions
of the multiphoton model. The mixing model has been built assuming that the
Rydberg resonances are not resolved, and hence the predictions of this model
should only be compared with the corresponding portions of the experimental
curve.  \Figure{mix_thy2}(a) shows a dramatic step-like fall-off in the
ionization signal in the same region as in the experimental curve.  The
fall-off, however, continues for frequencies above -40~cm$^{-1}$ in the
theoretical curve, whereas the experimental curve levels off in this region. 
However, the prediction of this feature at this location is a significant
improvement over the predictions of the multiphoton model, which predicted a
featureless ionization signal in this region.

For the higher intensity in \Fig{mix_thy2}, the agreement is very good.  The
model predicts a decreasing ionization signal for increasing laser frequency,
and the slope of this decrease is similar to the experimental value.  As in the
multiphoton model, we did not model the dc-field ionization as we were
interested only in the photoionization signal.

In general, the agreement between the experiment and the mixing model is very
good.  Major features seen in the experimental curves, which are absent from
the predictions of the multiphoton model, are reproduced in the mixing
model.  This should be regarded as strong evidence that indeed the optical
mixing of Rydberg states is the dominant interaction affecting the laser-atom
dynamics in this regime.




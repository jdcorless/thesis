
\begin{singlespace}

\chapter{Two-Photon Excitation: Experiment}
\label{ciexpt}
\begpagestyle
%\pagenumbering{arabic}

\end{singlespace}

In this chapter, we shall discuss the results of our experimental investigation
of population transfer using counterintuitive pulse sequences in a
Doppler-broadened medium, in the presence of effects due to Gaussian spatial
averaging and amplitude fluctuations.  In order to investigate the effects
predicted in \rChapter{citheory}, the atomic system must behave as a three-level
system under our excitation conditions.  Furthermore, since the counterintuitive
population transfer technique is coherent, the lasers must have near ideal
performance with respect to linewidth, spatial beam, and temporal profile.  We
address these issues in the context of a system of sodium vapor excited by
pulsed dye lasers.  We then present population transfer and amplitude
fluctuation results for both weak and strong field excitation. 

\section{Sodium as a Three-Level Atom}

\hspace{\parindent}  The cascade level structure we used in sodium vapor is
$\ket{3S_{1/2}} \rightarrow \ket{3P_{1/2}} \rightarrow \ket{4D_{3/2}}$ with
transition wavelengths of $589.755$~nm (`orange') and $568.421$~nm (`green'),
respectively.  These states are shown schematically in \Fig{Na_decay}. 
Coupling from these levels to others outside of the cascade system via
stimulated or spontaneous processes can have a negative impact on the
counterintuitive transfer technique.  Other states that could be coupled via
stimulated processes include the $\ket{3P_{3/2}}$ and the $\ket{4D_{5/2}}$
states. The separation between the
$\ket{3P_{1/2}}$ and $\ket{3P_{3/2}}$ levels is 17.196~cm$^{-1}
\approx 0.5$~THz. This separation is easy to resolve with our pulsed
dye lasers, which have linewidths of $< 1$~GHz.  The fine structure
splitting of the $\ket{4D}$ states is $0.035$ cm$^{-1} \approx 1$ GHz, which
is close to our resolution limit. By selective excitation of the
$\ket{3P_{1/2}}$ state instead of the $\ket{3P_{3/2}}$, the
dipole selection rules prohibit excitation from $\ket{3P_{1/2}}$ to
$\ket{4D_{5/2}}$.  This way, the upper state will be the single
$\ket{4D_{3/2}}$ state, instead of the two nearly degenerate states,
$\ket{4D_{3/2}}$ and $\ket{4D_{5/2}}$.


\begin{figure}[tbp]
\postfull{thesisfigs/two_photon_expt/Na_decay.eps}
\bigskip
\ncap{Spontaneous emission decay channels of sodium cascade system.}
{Spontaneous emission decay channels of sodium cascade system.
Coupled states of cascade system excited in experiment are shown in bold while
extra decay channels are indicated with fine lines.
\label{Na_decay}}
\end{figure}

The importance of coupling due to spontaneous emission is determined by
the relative magnitudes of the various transition lifetimes.  The spontaneous
emission coupling is shown schematically in \Fig{Na_decay}.  The decay lifetimes
between the cascade system states are 15.9~ns for the $\ket{3P_{1/2}}
\rightarrow \ket{3S_{1/2}}$ transition and 91.7~ns for the $\ket{4D_{3/2}}
\rightarrow \ket{3P_{1/2}}$ transition.  We see from \Fig{Na_decay} that the
upper $\ket{4D_{3/2}}$ state has other available decay channels. 
This coupling takes population out of the three-level system, and
hence could affect the time evolution of the driven system.  However, if
lifetimes associated with these transitions are long enough, they will not
influence the dynamics during the laser pulse.  After the pulses have turned
off, the various lifetimes will only serve to determine branching ratios to
all possible states.  Then, the time-integrated (green) fluorescence will
still be proportional to the population in the $\ket{4D_{3/2}}$, but now
with a proportionality constant determined by the associated branching ratio.

Level $\ket{4D_{3/2}}$ has three other levels to which it can radiatively decay
other than the $\ket{3P_{1/2}}$.  They are the $\ket{4P_{3/2}}$,
$\ket{4P_{1/2}}$, and the $\ket{3P_{3/2}}$, shown in \Fig{Na_decay}.  These
decay channels have lifetimes of 901~ns, 180~ns, and 457~ns, respectively
\cite{Wiese:69}.  Subsequent decay from these levels to the ground
$\ket{3S_{1/2}}$ state occurs with lifetimes of 345~ns, 341~ns, and 15.9~ns. 
Thus, the fastest competing decay from the upper state is $\ket{4D_{3/2}}
\rightarrow \ket{4P_{1/2}}$ with a lifetime of 180~ns.  This lifetime is twice
the 91.7~ns lifetime of the driven $\ket{4D_{3/2}} \rightarrow \ket{3P_{1/2}}$
transition and furthermore is approximately twelve times longer than our
pulsewidth of 15~ns.  The decay channels other than those within the cascade
system can therefore be ignored during the interaction.  Afterwards,
approximately 55\% of the population decays from the $\ket{4D_{3/2}}$
level to the $\ket{3P_{1/2}}$ level.

\begin{figure}[tbp]
\postfull{thesisfigs/two_photon_expt/mj_levels.eps}
\bigskip
\ncap{Magnetic sub-levels of sodium cascade system.}
{Magnetic sub-levels of sodium cascade system.  The solid line corresponds
to choosing the quantization axis along $\omega_1$'s linear polarization
direction, and the dashed lines along $\omega_2$'s linear polarization
direction.  The two laser polarizations are orthogonal.
\label{mJ_levels}}
\end{figure}

The magnetic level degeneracy is another potential complication that could
cause the driven system to stray from the three-level cascade model.  Given the
reduction of the problem to the three fine-structure levels, $\ket{3S_{1/2}}
\rightarrow \ket{3P_{1/2}} \rightarrow \ket{4D_{3/2}}$, the magnetic level
degeneracy is shown in \Fig{mJ_levels}, in which laser frequency $\omega_1$
drives the lower transition and $\omega_2$ the upper transition.  In the
experiment, we use two linearly polarized beams with orthogonal axes of
polarization.  This orthogonality is not required, and as we will
show, is formally equivalent to the parallel linear polarizations case.  This
orthogonality, however, did simplify the experimental layout.  In
\Fig{mJ_levels}, the ground and middle states, with total angular momentum
$J=1/2$, have two levels corresponding to $m_J = \pm 1/2$.  The upper state, with
its total angular momentum of $J=3/2$, has four states, $m_J = \pm 1/2,\pm
3/2$.  In specifying $m_J$ values, we are specifying states with definite $J_z$,
which assumes that we have chosen a direction in space to serve as our
``$z$''-axis of quantization.  While no direction is more ideal than another,
the coupling schemes are simplest if this quantization axis is chosen along one
of the linear polarization directions.  \Figure{mJ_levels} shows the coupling
scheme for the quantization axis chosen along the lower transition laser
polarization axis as solid lines, and for the quantization axis
chosen along the upper transition laser polarization axis as dashed lines.

For two linearly polarized fields with parallel polarizations, the
dipole selection rule $\Delta m_J =0$ causes the excitation
scheme to separate into independent $m_J=\pm1/2$ channels.  While the laser
fields cannot couple the channels spontaneous emission can mix channels
as we will discuss later.  Now if we keep the $\omega_1$ laser's linear
polarization fixed (parallel to the quantization axis), and rotate the
$\omega_2$ laser's linear polarization so that it is orthogonal to
$\omega_1$ (shown as solid lines in \Fig{mJ_levels}), the selection rules
on the lower transition remain $\Delta m_J =0$, while the selection rules
on the upper transition become $\Delta m_J = \pm 1$.  This can be understood by
considering the polarization basis.  We chose to represent the polarization
vector with basis vectors that include the direction of the quantization axis
(parallel to the $\omega_1$ laser polarization axis), and the corresponding
right- and left-handed circular polarizations.  In this basis, the polarization
of the $\omega_2$ laser would be represented as a linear combination of right-
and left-handed circular polarizations with corresponding dipole selection rules
$\Delta m_J = -1$ and $\Delta m_J = + 1$.  This might appear to turn the
three-level problem into a four-level one because the upper state now splits
into two degenerate levels.  However, by examining the equations of motion, it
can be seen that the middle level interacts with a particular linear combination
of upper levels, and furthermore, the dipole moment to this superposition of
upper levels is the same as the dipole moment of the upper transition when both
lasers are parallel.  So while the upper level contains two separate levels,
they behave as a single level with the same dipole moment as the parallel
polarization case, and so long as measurements that could distinguish
$m_J$ are not made, the system remains a three-level system for orthogonal laser
polarizations.  This reasoning can also be applied to the dashed line coupling
shown in \Fig{mJ_levels}, except here the system remains as independent
three-level systems, but the dipole moment on the lower transitions can be
shown to be unaffected by the polarization of the $\omega_1$ laser being
rotated 90$^{\circ}$ relative to the polarization of the $\omega_2$ laser.

The final consideration regarding level structure is whether or not spontaneous
emission can couple these otherwise independent three-level systems together. 
If spontaneous emission can mix the population between
the two three-level systems during the interaction, then our three-level
description may not suffice.  Since the lower transition's spontaneous emission
lifetime is approximately equal to the pulse duration and the upper transition's
spontaneous lifetime is six times longer than the pulse duration, we focus our
attention on the lower transition.

Assume the $\omega_1$ laser beam has its polarization axis parallel to the
quantization axis (solid line in \Fig{mJ_levels}).  Then spontaneous emission
from the middle levels can proceed both along these solid lines, $m_J = \pm 1/2
\rightarrow \pm 1/2$, and between channels along the dashed lines, $m_J = \pm
1/2 \rightarrow \mp 1/2$.  This causes the two channels to be coupled. 
However, there are two factors which conspire to make this coupling
trivial.  The first is the symmetry of the problem: the coupling
parameters are equal for both $m_J = + 1/2 \rightarrow - 1/2$ and $m_J = -1/2
\rightarrow + 1/2$ transitions.  Furthermore, the initial conditions are
symmetric in $m_J$.  The ground states are initially thermally populated
which means that they have equal population (degenerate energies) and there is no
coherence between these two states (thermal preparation).  Therefore, we can
imagine that as the laser drives this transition, equal population on both sides
is excited to the middle level.  Each of these populations is then
redistributed between the ground states.  But as much population decays
along one path as decays along the other path and so the ground states receive
the same population.  The second important point here is that this transfer
process is incoherent.  Therefore, coherence will not be built up between
these channels.  

The interaction we investigate is actually relatively insensitive to the
details of the decay from the middle level.  For the counterintuitive
excitation, the middle level is not populated and so the specific details of
its decay are unimportant for the adiabatic population transfer.  Even for the
intuitive case, when the delay between pulses is large enough to allow
significant decay from the middle level, the population in this level is
necessarily reduced and hence the population ultimately
transferred to the final state will be small.  So by the time this decay process
is important in the intuitive regime, the technique will have become an
inefficient way to transfer population to the final state.

So far, we have discussed issues relating to an individual sodium atom.  Now
we consider important experimental aspects of using sodium vapor.  When
using sodium vapor in a sealed absorption cell, the only physical parameter that
can be controlled is the temperature of the vapor.  The temperature changes the
number density of the sodium atoms, which in turn has an effect on
the way in which the sodium vapor interacts with light.  That is, the density
changes both the Beer's absorption length, and the line broadening.  
The vapor pressure, $p$, of sodium as a function of temperature in Kelvin
($T_K$) is given by \cite{CRC_book}
\begin{equation}
\log_{10}\left[p({\rm mmHg})\right] = 7.55299 - \frac{5395.37}{T_K}
\end{equation}
which can be converted into an expression involving the number density, $N$, by
use of the ideal gas law,
\begin{equation}
N({\rm cm}^{-3}) = \frac{9.6559 \times 10^{18}}{T_K} \exp\left(17.3914 -
\frac{12423.3}{T_K}\right).
\end{equation}
For a temperature of 120$^{\circ}$C, $N \approx 1.7 \times 10^{10}$ cm$^{-3}$.

The Beer's absorption length, $L_{Beer}$, is a parameter that characterizes the
1/e length of power absorption of a weak resonant field \cite{Allen:87},
\begin{equation}
\frac{1}{L_{Beer}} = \frac{1}{4\pi\epsilon_0}\frac{4 \pi^2 N |d_{12}|^2
\omega_0}{3 \hbar c} g(\Delta).
\end{equation}
In this expression, $d_{12}$ is the dipole moment of the driven transition with
angular frequency $\omega_0$, and $g(\Delta)$ is the lineshape function, which
will be Lorentzian for a homogeneously broadened line or Gaussian for a Doppler
broadened line, and $\Delta$ is the detuning from resonance.  For the
$\ket{3S_{1/2}}\rightarrow\ket{3P_{1/2}}$ transition on resonance and at
120$^{\circ}$C (where it is predominantly Doppler broadened), $L_{Beer}
\approx 30$ cm.

The broadening of the transition line is also dependent on temperature.  We
mentioned previously that the Doppler broadening of the
$\ket{3S_{1/2}}\rightarrow\ket{3P_{1/2}}$ transition is approximately $1.6$ GHz
FWHM and changes slowly with temperature ($\sim \sqrt{T}$).  The
radiative broadening was also given previously in terms of the decay parameters
in the density matrix equations of motion.  The $16$ ns lifetime of the
$\ket{3P_{1/2}}$ state gives rise to a Lorentzian linewidth of 10 MHz FWHM.
The other main line broadening mechanism is collisional, or self-broadening (we
do not use a buffer gas in our cell and hence inter-species broadening is not
an issue).  This mechanism was reviewed extensively by Lewis \cite{Lewis:80}. 
It was found that the FWHM linewidth due to self-broadening (for fine structure
levels) is
\begin{equation}
\Delta \nu_{1/2} = K^{'}(J_g,J_e)\sqrt{\frac{2J_e+1}{2J_g+1}} \frac{N
\lambda_{eg}^3}{8 \pi^2}\Gamma_e
\end{equation}
where $J_g,J_e$ are the angular momentum quantum numbers of the ground and
excited levels, $K^{'}(J_g,J_e)$ is a number approximately equal to one
(described in detail in \cite{Lewis:80}),
$\lambda_{eg}$ is the transition wavelength, and $\Gamma_e$ is the natural
linewidth of the excited level.  For a temperature of 120$^{\circ}$C, the
self-broadened linewidth is on the order of 2 kHz which is
negligible compared to the radiative width.  The self-broadened linewidth
becomes approximately equal to the Doppler width at $T \approx 500^{\circ}$C. 
So for temperatures far below this, the line is predominantly inhomogeneously
broadened, and far above it the line is homogeneously broadened.  Near $T
\approx 500^{\circ}$C, the line is determined by the Voigt lineshape
\cite{Corney_book}.
\section{Laser System}

\begin{figure}[tb]
\postfull{thesisfigs/two_photon_expt/ns_dye.eps}
\bigskip
\ncap{Schematic layout of the dye oscillator and amplifier system.}{Schematic
layout of the dye oscillator and amplifier system.  L1 is a 250 mm focal
length lens and L2 is a 150 mm focal length lens.  HR1 and HR2 are high
reflector mirrors.
\label{ns_dye}}
\end{figure}

\hspace{\parindent}  We now examine the pulsed laser system
used to drive our three-level cascade system \cite{Corless:97}.  Given our need
to select particular optical frequencies, the dye laser is the
natural choice for the experiment.  \Figure{ns_dye} shows a schematic layout of
our dye oscillator and amplifier system. The properties of the frequency-doubled
Nd:YAG pump are critical to dye laser performance and so we describe its
design first.  The Nd:YAG used is a Quantel model YG581-10 actively
Q-switched system providing approximately $ 60 $ mJ of $ 532 $ nm energy in a
single-transverse, multi-longitudinal mode output.  A typical temporal
profile produced by this system is shown in \Fig{yag_pulse}(a).  This data was
taken with a Hi~Voltage Components,~Inc. Model PDH-114-P vacuum photodiode and
a Tektronix 7912AD digitizing oscilloscope with a 500~MHz bandwidth.  The
amplitude modulation is indicative of the multi-longitudinal mode oscillation,
and the pulse profile varies significantly from shot to shot. As will be
discussed in the section on the temporal profile, this gives rise to an
unacceptable temporal profile of the dye laser.  To remedy this we modified the
cavity in two ways.  Since Nd:YAG is predominantly homogeneously broadened,
spatial hole burning is one of the main obstacles in obtaining single-mode
oscillation \cite{Draegert:72}. This was eliminated in the linear cavity
geometry by making use of a ``twisted-ribbon mode'' \cite{Evtuhov:65} in the
cavity, achieved by surrounding the gain medium with $ \lambda/4 $
waveplates with the output coupler on one side and a linear polarizer on the
other.  This generates an axially uniform mode energy density inside the gain
medium and hence eliminates spatial hole burning.  The second modification was
to replace the Pockels cell Q-switch with a LiF crystal with F$_2^{-}$ color
centers acting as a slow saturable absorber Q-switch \cite{Sumida:86}.  This,
combined with an intra-cavity etalon, produced single-longitudinal-mode output
with a smooth temporal profile shown in
\Fig{yag_pulse}(b).  The resultant 532 nm energy was limited to $12$ mJ. The
Nd:YAG beam was focussed with a $f=250$ mm lens to a spot before the dye cell
with small translations of the lens along the optical axis enabling variation
of the size of the gain region.  Focusing before the cell is preferable as it
ensures that the energy is not focussed onto the dye laser mirrors.

\begin{figure}[tb]
\postfull{thesisfigs/two_photon_expt/yag_pulse.eps}
\bigskip
\ncap{Temporal profile of the Nd:YAG pump laser.}{Temporal profile of the
Nd:YAG pump laser. In (a) the laser is actively Q-switched and is
multi-longitudinal-mode, and in (b) the laser is passively Q-switched and is
single-longitudinal-mode.
\label{yag_pulse}}
\end{figure}

The dye oscillator is based on the original design of Shoshan \cite{Shoshan:77}
and Littman 
\cite{Littman:78a,Littman:84} which makes use of a single diffraction 
grating used in grazing incidence to provide frequency selectivity.  
This is a convenient design because it involves only a few optical 
components and is easy to work with and build.  The grating used is a
1800 lines/mm holographic grating manufactured by Optometrics 
Corporation.  The choice of grating spacing depends on the operating wavelength 
of the laser.  Higher diffraction orders are formed when either the lasing
wavelength or the number of lines per mm is decreased.  This 
has two negative consequences.  First, sending light into higher diffracted
orders increases the loss of the cavity and hence reduces the 
conversion efficiency.  Second, it is possible that in the grazing 
incidence mode of operation there may be a Littrow diffracted order 
within the dye bandwidth.  This can cause more than one frequency to lase
simultaneously.  By choosing  the grating spacing appropriately, these effects
can be avoided.

The rear high reflector HR1 is a high damage threshold tunable laser mirror
made by CVI Laser Corporation, model TLM1.  This mirror was epoxied to a
piezoelectric transducer (PZT) stack, Thorlabs model AE0505D16, which provides
a displacement of $\approx$ 1 $\mu$m at a drive voltage of 10 V.  This
enabled the use of a standard laboratory power supply to make cavity length
changes on the order of the wavelength, which proved to be very useful in
single-mode operation.  The tuning mirror, HR2, is manufactured by Optometrics
and is matched to the grating length of 5~cm.  Since dispersion in the
cavity depends on the length of the illuminated part of the grating (for fixed
grating spacing), the 5 cm length is preferred over more compact
gratings.  The tuning mirror is epoxied to a mirror mount (Thorlabs KC1-PZ)
which contains a PZT stack similar to the one mentioned above in conjunction
with the standard $ 1/4-80 $ adjustment screw.  The PZT allowed for high
resolution tuning through narrow spectral features.  The quartz  dye cell is
manufactured by NSG Precision cells, Type 48 and has a path length of 1 mm.

% LINEWIDTH

There are many commercial dyes available, enabling dye lasers to tune through
the visible and well into the infrared and ultraviolet regions of the
spectrum with essentially no gaps.  However, when a broadband dye is used
in a very low-Q cavity (typical of the grazing incidence design) the entire
tuning range cannot be realized.  For a given pumping level, if the laser
operates in a single-longitudinal-mode (SLM) on the wings of the tuning curve,
it may be multi-longitudinal-mode at the gain peak.  Consequently, the effective
bandwidth of these dye lasers can be reduced, causing gaps in the spectrum
covered by existing dyes. This can be remedied, however, by modifying the
chemical environment of the dye \cite{Schafer_book}. The gain maximum can be
adjusted by changing the concentration, the solvent, or the pH of the solution. 
A shift of the gain maximum can also be achieved by changing the length of the
gain region \cite{Schafer_book}, although this method is not practical.  The
concentration is typically adjusted to provide good pump absorption while still
providing a near cylindrically symmetric gain region within the dye.  These two
competing requirements are usually balanced to yield acceptable results for both
with an absorption level $\sim 80$\% leaving no room for wavelength tuning via
concentration adjustment.  The solvent, however, is modified easily within the
constraints of solubility.  For example, a simple switch from methanol to
ethanol resulted in a 4~nm blue shift.  Using a 50:50 mixture of H$_2$O and
ethanol solution (with a small addition of the surfactant Ammonyx LO to
prevent dye aggregate formation) gave an additional $\sim 5$~nm blue shift.
And as a final technique that works well with the Rhodamine dyes, the pH of
the solution can be varied.  This is achieved by adding a small quantity
of base (saturated solution of KOH in ethanol) to blue shift by up to 10 nm,
or acid (HCl in water) to red shift by up to 10~nm.  The use of both solvent
and pH tuning are typically not needed for a single dye, but we regularly use
both techniques to shift the dye tuning curve and/or to maximize energy
output into a given wavelength.  For the orange laser, a $ 3 \times
10^{-4}$~M solution of Rhodamine 610 in 2~$\ell$ of methanol with 1/2 $\ell$
of 95:5 H$_2$O:Ammonyx LO gave a  single-mode tuning range of 582-599~nm
with a peak near the $\ket{3S_{1/2}} \rightarrow \ket{3P_{1/2}}$ transition of
$ \approx 589$~nm.  The same dye solution was used in the amplifier as in the
oscillator.  For the green laser, the same dye and concentration were used,
but ethanol was the solvent with a few drops of base (saturated solution of
KOH in ethanol).  The amount of base added was determined experimentally by
observing the output until the peak laser wavelength was near $568$~nm.

Given the particular dye environment found for achieving a specified 
wavelength, the main issue becomes attainment of SLM operation.
Single-mode output has been previously reported with this cavity 
\cite{Littman:84,Littman:78b}.  The most important parameters affecting it are
cavity length and angle of grazing incidence.  By use of longitudinal pumping,
short gain regions can lead to short cavities with lengths $\sim$ 4--5~cm.
The angle of incidence is typically 89--89.5$^{\circ}$.  The size of the
gain region also can affect single-mode performance.  Different
longitudinal modes travel slightly different paths in the dye cavity
\cite{Kangas:89}.  The gain region acts as a slit in the spectrometer
formed by the grating and tuning mirror. The larger the slit, the lower the
spectral resolution. We have verified that a cavity geometry that operates
single-mode for one pump spot size can act multimode for a larger gain
region.  This would seem to imply that very small gain regions should be
used.  However, diffraction can give rise to unacceptably large losses. 
Thus, the size of the gain region can be an important cavity parameter.  We
report performance with a pump beam having a Gaussian waist size of 220~$\mu$m.

Continuous wavelength tuning is the most convenient method to locate and
maintain atomic resonances.  When the laser frequency mode-hops, locating
narrow atomic resonances can be exceedingly difficult with pulsed lasers.
If the axis of rotation of the tuning mirror in our cavity is not chosen 
properly, the act of rotating the mirror will change the cavity length in an
uncontrolled fashion which will lead to mode-hopping.  It has been shown
\cite{Liu:81,Zhang:92}, however, that if the axis of rotation is  chosen
properly, the rotation can cause a change in cavity length that exactly tracks
the frequency change so that the longitudinal modes move in tandem with the
grating passband, thereby preventing mode-hops.  In our initial system, we had a
single-mode tuning range of $< 0.05$ cm$^{-1}$.  After a modest attempt
to align the rotation axis with the special location extended the tuning range
substantially to 1.7~cm$^{-1}$.  This was sufficient for our application. 
Note that, as shown by Zhang \etal \cite{Zhang:92}, tuning ranges of
190~cm$^{-1}$ can be achieved with more stringent alignment procedures.

To investigate the single-shot linewidth, we utilized a modified
plane-parallel Fabry-Perot interferometer (Burleigh RC 110) with a free
spectral range of 250~MHz and finesse $\sim 10$ and observed circular fringes
with a laser beam analyzer (Spiricon model LBA-100).  We digitized
single-shot fringe patterns and used these to measure the linewidth. We
concurrently digitized the temporal profile.  This enabled us to calculate
the theoretical transform-limited linewidth without making any assumption about
the pulse shape. This helps to eliminate any systematic error caused by lack of
knowledge of temporal profile.  For our pulse with a FWHM
$\approx 20$~ns we calculated a transform-limited linewidth of 36~MHz and
measured 58~MHz.  This gives $1.6 \times$ transform-limited performance for a
single-shot.  This compares favorably with performance from pulse amplified CW
sources where 1.4 $\times$ transform-limited performance has been achieved
\cite{Black:94}.  However, the simplicity of this design eliminates the need
for a complicated stabilized CW source.  Of course, an advantage of
a stabilized CW source is the long term frequency stability.  To analyze
this parameter for our system, the same single-shot spectrum was averaged
over 100 shots to produce a time averaged linewidth of 240~MHz.  The
most likely source of this frequency jitter is non-Laminar flow in the dye
cell \cite{Maruyama:91}.

\begin{figure}[tb]
\postfull{thesisfigs/two_photon_expt/tempC.eps}
\bigskip
\ncap{Optogalvanic signal as a function of temperature of the dye
fluid.}{Optogalvanic signal as a function of temperature of the dye fluid (open
circles).  The best fit Gaussian (Doppler) profile is shown as the solid line.
\label{tempC}}
\end{figure}

The main source of frequency drift in our system was temperature drift in
the dye.  We measured the required temperature stability by using a
heat exchanger in the dye flow system coupled to a constant temperature bath
recirculator. This system provided temperature stability of 0.01~$^{\circ}$C
with temperature resolution of 0.1~$^{\circ}$C.  Using a Na/Ne optogalvanic
cell, we monitored a Ne optogalvanic resonance signal as the temperature of
the bath was varied.  The results are shown in \Fig{tempC}.  The Ne resonance
at 588~nm has a (Doppler broadened) width of 3~GHz \cite{King:77}.  The best
fit Doppler profile gives a temperature FWHM $= 0.3$~$^{\circ}$C. This
implies a temperature-induced frequency drift of 10~GHz/$^{\circ}$C.  Using
the temperature-dependent refractive index of ethanol and the length of the
gain region, we can calculate a predicted frequency drift of 9~GHz/$^{\circ}$C,
very close to the measured value.  This shows that the temperature of the dye
must be kept stable to less than a tenth of a degree Celsius if sub-GHz
resolution is required.  Without the use of the constant temperature bath, we
found two modifications to the dye flow system that helped to minimize the
temperature drift.  The first consideration was to use a large dye reservoir. We
found that 2~$\ell$ was a reasonable compromise between stability and
convenience.  The second modification was to arrange the dye pump so that it
pulled fluid through the dye cell rather than forcing it through.  This had a
considerable reduction in temperature drift because the motor is the largest
source of heat in the system.

Slow drifts can be eliminated by active cavity stabilization using the
technique of Raymond \etal \cite{Raymond:89}.  This scheme makes use of
the different intra-cavity paths taken by different longitudinal modes which
creates angular deviations in the output (this can also give rise to transverse
mode structure which will be discussed later).  By using a two-element
detector, an error signal can be derived from this angular separation.  We
have implemented this technique and achieved the short time-averaged linewidth
for much extended periods.

Finally, using a one meter Jarrell-Ash Czerny-Turner spectrometer with a
Princeton Applied Research CCD camera and optical multichannel analyzer we
measured the amplified spontaneous emission (ASE).  The measurement was limited
by short-term baseline drift in the detection system.  We obtained an upper limit
of ASE to be $< 10^{-4}$ in our amplified system.

% SPATIAL PROFILE

\begin{figure}[tb]
\postfull{thesisfigs/two_photon_expt/spatial.eps}
\bigskip
\ncap{Spatial profile of nanosecond dye laser.}{(a) Spatial profile of the
dye laser.  (b) Cross-section of the profile taken horizontally through
the center of the beam showing its near Gaussian spatial quality.
\label{spatial}}
\end{figure}

\begin{figure}[tb]
\postfull{thesisfigs/two_photon_expt/gain_guide.eps}
\bigskip
\ncap{Spatial profiles of pump and dye lasers showing impact of
gain-guiding.}{Spatial profiles of the pump and dye lasers showing the impact
of gain guiding in the oscillator.  In (a) is the Nd:YAG mode incident on
the dye cell.  In (b) is the resultant dye beam.
\label{gain_guide}}
\end{figure}

The bare resonator cavity as described contains no beam shaping optical
elements.  The grating and tuning mirror can be replaced by a single flat
mirror for mode calculations.  The model cavity then contains two flat
mirrors, a marginally stable design with large diffractive losses. 
When the dye is pumped, however, gain guiding serves to shape the output beam
and make the cavity less lossy \cite{Casperson:68}.  Thus the dominant beam
shaping within this cavity is done by the gain region, which in turn depends on
the transverse quality of the Nd:YAG pump beam.  To demonstrate the gain guiding
of this resonator, we first aligned the dye resonator to achieve the best
spatial mode, which is nearly Gaussian and is shown in \Fig{spatial}.  We
then modified the  pump beam (nominally a Gaussian spatial mode) from the Nd:YAG
by placing a thin wire horizontally in the beam after the focusing lens.  This
served to create a beam with horizontal nulls in the focused beam profile at the
cell, shown in \Fig{gain_guide}(a).  Then, without modifying the dye cavity,
we measured the output beam from the dye (shown in
\Fig{gain_guide}(b)).  The correlation in shape is clearly visible.  It is
important to note that the shape of the output dye beam is not caused by
the cavity being misaligned so that a higher-order mode (e.g. 
TEM$_{20}$) is lasing in the bare cavity, or because of a
spatially modulated loss. Rather, the shape of the gain region is determining
the beam shape of the output.  We also verified this gain guiding by using
slightly distorted pump beams and obtained similar (though less dramatic)
results.

We see that the spatial quality of the dye beam is critically linked to
the spatial quality of the pump beam.  Since the dye beam appears nearly
Gaussian, we sought to quantify the near diffraction limited performance by
measuring the $M^2$ beam quality parameter \cite{Ruff:92}.  This entailed
measuring the spot size at the focus of a high quality lens ($f=100$ mm doublet)
and in the far field of that lens.  We measured $M^2=1.2$ which is comparable
to the lowest $M^2$ obtained for a CW argon ion laser in a recent study
\cite{Johnston:92}.  We therefore see that the high quality 
(TEM$_{00}$) mode of the Nd:YAG pump gives rise to a near diffraction
limited mode of the dye beam in this type of cavity.  This is one of the
main benefits of using the longitudinally pumped gain region instead of
the original transversely pumped design \cite{Shoshan:77,Littman:78b}.

\begin{figure}[tb]
\postfull{thesisfigs/two_photon_expt/doub_mode.eps}
\bigskip
\ncap{Spatial profile of the dye beam in a  quasi-single-longitudinal-mode dye
laser.}{Spatial profile of the dye beam in a  quasi-single-longitudinal-mode
dye laser.  The two lobes are present when the laser is running in two
longitudinal modes.
\label{doub_mode}}
\end{figure}

A final issue affecting transverse mode quality is whether the laser
is running in a SLM.  We mentioned above that different longitudinal modes travel
slightly different paths in the oscillator cavity.  This can give rise to
slight angular deviations in the direction of the output beam.  It also can
radically modify transverse mode profiles.  We show in \Fig{doub_mode} a
spatial mode profile obtained when the dye laser was running in two
longitudinal modes.  This cavity was quasi-SLM in the sense that changing the
cavity length by $\lambda/2$ with the PZT could cause the laser to cycle
through SLM operation to double mode operation and back again to SLM. When the
laser was in double mode operation, the different longitudinal modes actually
were separated transversely in space, as seen in the figure.  By tuning the
cavity length so that SLM operation was achieved, the spatial mode returned to
its near Gaussian profile.

% TEMPORAL PROFILE

\begin{figure}[tb]
\postfull{thesisfigs/two_photon_expt/dye_bad.eps}
\bigskip
\ncap{Temporal profile of a single-longitudinal-mode dye laser (solid line)
when pumped by a multi-longitudinal mode Nd:YAG laser.}{Temporal profile of
a single-longitudinal-mode dye laser (solid line) when pumped by
a multi-longitudinal mode Nd:YAG laser.  Shown as a dashed line is the pump
temporal profile (from \Fig{yag_pulse}(a)) that produced the dye laser pulse. 
\label{dye_bad}}
\end{figure}

When a multi-longitudinal mode laser pulse is used to pump a dye laser,
the temporal profile of the dye laser pulse will be similar to the pump pulse,
even if the dye laser is in a SLM.  This behavior is shown in \Fig{dye_bad}
The temporal pulse shape data were taken with a Hi Voltage Components, Inc.
 Model PDH-114-P vacuum photodiode and a Tektronix 7912AD digitizing oscilloscope
with a 500~MHz bandwidth.  If the dye laser is pumped just slightly above
threshold then the pulse may be smooth and significantly shorter than the pump
pulse \cite{Lin:75}.  The amplitude fluctuations typically will be larger in
this configuration, and the energy output will be reduced.  Therefore it is
often desirable to run the laser significantly above threshold with the dye
laser pulse essentially following the pump pulse profile (for typical nanosecond
Nd:YAG, N$_2$ and Excimer pump pulses) as in \Fig{dye_bad}.

\begin{figure}[tb]
\postfull{thesisfigs/two_photon_expt/rough.eps}
\bigskip
\ncap{Temporal profile of a single-longitudinal-mode dye laser when pumped
by a single-mode (smooth) pump laser.}{Temporal profile of a
single-longitudinal-mode dye laser when pumped by a single-mode (smooth) pump
laser.  The rapid rising edge is indicative of the output from the low-Q dye
cavity.
\label{rough}}
\end{figure}

If smooth dye laser pulses are required, then the pump pulse must also be
smooth, as in \Fig{yag_pulse}(b).  This is not enough to determine smooth
output, however.  The dye laser must not exhibit mode-beating itself, and so SLM
operation must be attained.  The SLM output from the dye laser, pumped by the
smooth Nd:YAG pulse, is shown in \Fig{rough}.  Here we see dramatic
improvement in the smoothness of the temporal profile.  There is still a
rather sharp rising edge to the pulse. These sharp features were observed
experimentally and modeled theoretically by Sorokin \etal
\cite{Sorokin:67}. In this work it was found that the sharp rising edge was
indicative of a rapid relaxation oscillation spike characteristic of these
low-Q cavities.  The severity of this rising edge can be modified
somewhat depending on the particular day-to-day cavity alignment
(the feature was not minimized for \Fig{rough}).  Nonetheless, the temporal
profile is generally very smooth and much improved as compared with
\Fig{dye_bad}.

\section{Experimental Layout}

\begin{figure}[tbp]
\postfull{thesisfigs/two_photon_expt/apt_expt.eps}
\bigskip
\wcap{Experimental layout for counterintuitive pulse experiment.}{Experimental
layout for counterintuitive pulse experiment.  The $\omega_1$ dye laser beam is
combined with the $\omega_2$ dye laser beam in a polarizing beam splitter
(PBS).  The $\omega_2$ beam is first collimated in a corrected lens system (L),
then passes through a quarter-wave plate ($\lambda/4$) in a variable delay arm
($\Delta\tau$).  The two beams interact with a sodium vapor cell (Na) where the
fluorescence is imaged through a green spectral filter (SF), onto a pinhole
(PH), and finally into a photomultiplier tube (PMT). Nonabsorbed beams propagate
on to laser diagnostics.
\label{apt_expt}}
\end{figure}

\hspace{\parindent} \Figure{apt_expt} shows the general layout of 
our experiment.  We use two pulsed dye lasers.  The orange laser is labeled
$\omega_1$ and the green laser $\omega_2$.  The green light emitted from the
laser has its plane of polarization in the plane of the paper.  It travels
through a collimator and into a polarizing beam splitter (PBS), where it enters
the variable delay arm.  This consists of a corner cube high reflector, a
quarter-wave plate and a retroreflecting high reflector.  After two passes
through this system, the beam's polarization is rotated through
90$^{\circ}$.  Then, when it enters the PBS, it is ejected out the side and
exits the delay arm.  The orange beam has a static time delay built-in so the
two beams can be made to overlap in time for some predetermined position of
the variable delay arm.  The orange beam has its plane of polarization in the
plane of the paper, so that when the two beams are combined in another PBS, they
have orthogonal polarizations.  The two beams then travel on to the sodium
absorption cell.

The collimation of the green beam is very important in this experiment.  If the
beam is not well-collimated, or if it changes shape upon propagation due to
aberrations, then the size of the beam entering the sodium absorption cell will
vary with the time delay.  The peak Rabi frequency associated with
this laser would then change as a function of delay.  This would be a
systematic error in the experimental system.  To prevent this, we used a
corrected two-element (Galilean) beam expander.  The third-order spherical
aberration can be cancelled in this system by choosing appropriately shaped
lenses.  The third-order longitudinal spherical aberration (LSA) scales as
\begin{equation}
{\rm LSA} \sim \frac{f}{(f/\#)^2}
\end{equation}
with a constant of proportionality determined by the shape of the lens. For
a beam expander of this sort, the $f/\#$ is the same for both lenses. 
Then, since negative and positive focal lengths contribute opposite signs to
the LSA, we can eliminate LSA by choosing the focal lengths so that for
convenient shapes, the two contributions are equal and opposite
\cite{Melles_Griot}.  This can be achieved with $2.7 \times$ beam expansion with
a 25~mm plano-concave lens and a 67~mm bi-convex lens.

The two beams entering the cell have a spot-size of 3~mm.  The cell temperature
was approximately 120$^{\circ}$C giving a Beer's absorption length
$\approx 30$~cm.  This is significantly longer than the cell length of 7.5~cm
so that the absorption is small.  This avoids propagation effects including
self-induced transparency \cite{Allen:87} as well as electromagnetically induced
transparency phenomena \cite{Harris:94,Eberly:94}. The fluorescence is collected
at right angles to the beam's propagation direction.  Both orange and green
fluorescence are present and so a green spectral filter is used to eliminate the
orange fluorescence.  This way, we measure the green fluorescence as a measure
of population transfer to the upper state, as discussed earlier in
\rChapter{citheory}.

The fluorescence is detected with a Hamamatsu R928 photomultiplier tube (PMT)
whose quantum efficiency is $\sim 10$\% at 570~nm.  This signal was filtered
with a Mini-Circuits BLP-21.4 low pass filter with a 3dB frequency of 24.5~MHz,
and then integrated with a gated integrator, Stanford Research Systems Model
SR250.  Since we wish to draw conclusions from the data regarding the amplitude
fluctuations of the fluorescence signal, we must be sure to minimize detection
noise so that the laser amplitude fluctuations represent the largest source of
shot-to-shot signal variation.  The main way we minimized the detection noise
was to operate the PMT in the analog mode \cite{Hamamatsu_book}.  This mode of
operation is realized when there is enough optical power incident on the
photocathode so that the output voltage is proportional to the input intensity. 
The photon counting mode is the other frequently used mode of operation and is
preferable when signals are weak.  

The main sources of noise in a PMT are shot noise in the dark
current and amplification noise \cite{Boyd_book1}.  The R928 has a typical
anode dark current of $ \overline{i_D} = 3$~nA.  The variance of this signal
is given by Schottky's formula to be $\overline{(\Delta i_D)^2} = 2 e
\overline{i_D} \Delta f = (0.14$~nA)$^2$, where $e$ is the electronic charge,
and $\Delta f \approx 20$~MHz is the electrical bandwidth (determined in this
case by the low-pass filter).  The amplification noise in PMT's is due to the
variance in the dynode's secondary electron emission, and is very low when
operated at high gain. The (multiplicative) decrease in signal-to-noise (S/N)
is given by \cite{Boyd_book1}
\begin{equation}
\gamma = \sqrt{\frac{\delta}{\delta -1}}
\end{equation}
where $\delta$ is the secondary electron emission.  For the R928 operated at
500 V, the gain is $10^{5}$. The relationship between gain, $G$, and the
secondary electron emission, $\delta$, is
\begin{equation}
G = \delta^N
\end{equation}
where $N$ is the number of dynode stages.  For this nine stage device,
this gives $\delta \approx 3.6$.  The excess noise factor due to amplification
noise is then $\gamma \approx 1.18$.  Thus, the S/N at the photocathode gets
reduced by 18\% at the anode while the signal experiences a gain of $10^5$.

We now estimate the shot noise in the signal current.  The typical
fluorescence signal had a peak voltage of $\approx 100$~mV in a 100~ns time
interval.  Terminated into a 50~$\Omega$ load this gives an anode signal current
of $\overline{i_S} = 2$~mA. Using Schottky's formula, the signal current
variance is $\overline{(\Delta i_S)^2} \approx (0.11~\mu$A)$^2$.  Note that
$\overline{(\Delta i_D)^2} \ll \overline{(\Delta i_S)^2}$ and so the fluctuation
in the anode dark current can be ignored. This gives S/N $\approx 18,000$.  The
amplification noise will degrade this S/N by $\approx~18$\% to S/N $\approx
15,000$.  Clearly, this will not be the limiting noise in the system.

The gated integration signal is digitized by a Data Translation
DT2801-A A/D board.  This unit provides 12-bit A/D conversion with maximal
bipolar input voltage ranges of $\pm 10$~V.  This gives a minimum resolvable
voltage change of 4.9~mV.  The gain on the gated integrator was adjusted to
provide $\sim 1$~V of output signal.  Thus we expect a dynamic range of
$1/0.0049 \approx 200$.  Furthermore, the amount of fluctuation observed in the
gated integrator was one part in a hundred.  These are more stringent
limitations on the S/N than those attributed to the PMT itself and thus will
determine the dynamic range of our measurement.

The laser diagnostics used included a Na/Ne optogalvanic lamp.  This device
creates a transient voltage signal when the laser field is resonant with one of
the elements in the lamp (Na or Ne in this case).  This was a convenient means to
locate single photon resonance for the orange laser. Although this can also be
done by observing the fluorescence from a hot sodium lamp, this technique does
not work for the green resonance.  The optogalvanic signal for double resonance
is easily detected with a magnitude $\sim 100$~mV with duration $ \sim 1~\mu$s.
The laser linewidths were also monitored with our single-shot Fabry-Perot
spectrometer.  This also enables us to monitor frequency drifts to insure that
the laser central frequency stays fixed, since this system, unlike the
optogalvanic signal, gives a sign to the error signal and hence can be used in a
(human) feedback loop.  Finally, a fast vacuum photodiode measures laser
temporal profile as well as providing for calibration of the time delay.

\section{Population Transfer Results}
\hspace{\parindent}

We now present the results of the experiment.  We begin by examining the
fluorescence signal obtained when the laser is weak and when we expect that the
atomic evolution is not adiabatic.  The energy in the orange laser pulse was
16~$\mu$J and in the green pulse was 14~$\mu$J.  The durations of the pulses were
approximately equal and had a FWHM = 15~ns.  The spot-size of the beams was
3~mm.  For the orange laser this gives a maximum Rabi frequency (on-axis of laser
beam) of $\Omega_1/2\pi \approx 3$~GHz.  The corresponding area of this pulse is
$\approx 200\pi$.  For the green pulse, the peak Rabi frequency is
$\Omega_2/2\pi \approx 1.6$~GHz and the area is $\approx 105\pi$.  To test the
weak field response, we placed neutral density filters in front of these beams
with attenuation N.D. = 4.  This reduces the area of each pulse by a factor
of 100 and we would expect the atomic response not to be adiabatic and
hence the counterintuitive pulse delays should be less efficient.

\begin{figure}[tbp]
\postfull{thesisfigs/two_photon_expt/weak_ave.eps}
\bigskip
\wcap{Integrated fluorescence signal obtained under weak field excitation
conditions.}{Integrated fluorescence signal obtained under weak field excitation
conditions.  In (a) is the experimental curve, and in (b) the theoretical
curve.  Positive time delays correspond to the counterintuitive pulse ordering.
\label{weak_ave}}
\end{figure}

\Figure{weak_ave} shows the integrated fluorescence signal measured due to
these attenuated laser beams.  \Figure{weak_ave}(a) is the measured signal, and
\Fig{weak_ave}(b) is the theoretical prediction, based on the theory
presented in \rChapter{citheory}.  The model used to do the theoretical
calculation includes all the effects studied, including Doppler broadening,
Gaussian spatial beam averaging, and amplitude fluctuations.  The first remark
to make about the data obtained is that the range of time delay that could be
investigated was limited.  Since the pulses have FWHM durations of 15~ns,
actually delaying the two pulses significantly on both the intuitive and
counterintuitive side was not possible in a single data run.  The available
space on our optical table was about 7~ft.  When this is quadruple-passed
in the delay arm, we obtain a maximum time delay of $\approx 28$~ns, which is two
pulsewidths.  If the static delay of the orange beam in \Fig{apt_expt} is set up
so that the delay scans the counterintuitive side, then this static delay would
have to be changed if the intuitive side were to be investigated.  This would
involve complete realignment, and hence, the two data sets would not be
comparable.  As we feel that the physics associated with the intuitive pulse
sequencing is well-known and understood, we focused our attention on the
counterintuitive side (positive time delays in the figures), with some data
taken on the intuitive side to set a reference.

The theoretical curve qualitatively agrees with the measured signal.  In
general, both curves show decreasing fluorescence signal for increasing time
delays, with maxima near zero delay.  The measured signal extends further into
the counterintuitive side than the theory.  This may be due to several factors. 
The first is that when the field is weak, the atom behaves less like a
three-level atom.  The hyperfine splitting of the ground state in our weak field
is partially resolved. Second, with such small laser beam energies, it was not
possible to measure the energy of each attenuated beam to verify the neutrality
of the filters or their rated attenuations.  Higher energies would cause the
theoretical curve to extend further into the counterintuitive side.  But in
general, both curves show similar shapes as a function of time delay.

\begin{figure}[tbp]
\postfull{thesisfigs/two_photon_expt/weak_fluc.eps}
\bigskip
\ncap{Normalized amplitude fluctuations under weak field excitation
conditions.}{Normalized amplitude fluctuations under weak field excitation
conditions.  The experimental curve is the dashed line connecting circles and
the theoretical curve is the solid line.
\label{weak_fluc}}
\end{figure}

The data shown in \Fig{weak_ave}(a) represents the average of 200 laser shots
measured at each time delay.  \Figure{weak_fluc} shows the standard deviation
of the measured signal relative to the (time delay dependent) mean
value (\Fig{weak_ave}(a)).  That is, for fluctuating variable $x$, we plot 
$(\sigma_x/\overline{x})\times 100$\%.  The corresponding theoretical plot
associated with the mean displayed in \Fig{weak_ave}(b) is shown as the solid
line in \Fig{weak_fluc}.  The measured laser intensity fluctuations for each
laser was found to be $\approx 20\% \times \overline{I}$, with $\overline{I}$
the mean of the intensity.  We used these numbers for the theoretical curve.  The
agreement, with no free parameters, is very good for a large range of delays. 
As the delay becomes large, the agreement becomes worse.  This is most likely
due to the fact that as the measured signal becomes small, the dominant source
of noise is no longer from the atomic interaction and instead comes from
detection noise.  Since this is not included in the theoretical curve, we expect
the two curves to separate for large delays (when average signals are small). 
This figure shows that, for weak fields, not only is the population transfer
smaller for counterintuitive delays, it is also more noisy.

\begin{figure}[tbp]
\postfull{thesisfigs/two_photon_expt/strong_ave.eps}
\bigskip
\wcap{Integrated fluorescence signal obtained under strong field excitation
conditions.}{Integrated fluorescence signal obtained under strong field
excitation conditions.  In (a) is the experimental curve, and in (b) the
theoretical curve.
\label{strong_ave}}
\end{figure}

When the neutral density filters are taken out, we expect that the
atomic dynamics should be adiabatic.  \Figure{strong_ave} displays
the measured time-integrated fluorescence signal and the corresponding
theoretical calculation for the strong field case.  Again qualitative agreement
is obtained.  The theoretical prediction of \Fig{strong_ave}(b) assumes
Gaussian laser temporal profiles.  We will improve the agreement between theory
and experiment shortly by using a more realistic model of the laser pulse.  The
most noticeable difference between the two curves in \Fig{strong_ave} is the
presence of a dramatic dip in the theoretical curve near zero delay, whereas
only slight evidence of this structure is seen in the experimental curve.  The
measured curve shows an increase in signal from about 0.7 on the intuitive side
to 1.0 on the counterintuitive, or an enhancement of 43\%.  The theoretical
curve shows enhancement from 0.55 to 1, or 80\% increase.  The shapes of the
two curves, not withstanding the dip near zero, are similar.  The experiment
shows that even in the presence of Doppler broadening and Gaussian spatial
averaging, the counterintuitive delay is more efficient at populating the upper
state.

\begin{figure}[tbp]
\postfull{thesisfigs/two_photon_expt/strong_fluc.eps}
\bigskip
\ncap{Normalized amplitude fluctuations under strong field excitation
conditions.}{Normalized amplitude fluctuations under strong field excitation
conditions.  The experimental curve is the dashed line connecting circles and
the theoretical curve is the solid line.
\label{strong_fluc}}
\end{figure}

The relative amplitude fluctuations for the strong field case are shown in
\Fig{strong_fluc}.  Here again, with no free parameters, we see very good
agreement between experiment and theory.  The most important feature of this
curve is the reduction in the fluctuation on the counterintuitive side.  The
relative fluctuation decreases from 15\% on the intuitive side to a minimum of
about 9\% on the counterintuitive.  Since this is the relative fluctuation,
one might argue that this is due solely to the fact that the signal increased
by 43\%.  However, in general the noise-to-signal ratio (N/S) ratio decreases
as one over the square root of the signal, and hence, a 43\%
increase in signal would be accompanied by a new N/S that is
84\% of the original value.  The reduction from 15\% to 9\%, however, is a
60\% reduction in N/S.  We therefore see that the population transfer process
is more robust to amplitude fluctuations when the pulses are applied in the
counterintuitive order.  The general shape of the two curves are very similar,
and even some of the oscillations in the theory are seen in the experiment.

The agreement between theory and experiment shown in \Fig{strong_ave} can be
improved if we recognize that the experiment utilized a pulse that is not
Gaussian (as was assumed in the previous theory), but actually has the sharp
rising edge shown in \Fig{rough}.  This pulse shape was predicted
theoretically and observed experimentally by Sorokin \etal \cite{Sorokin:67}. 
Using the theory presented in this study, we numerically integrated the
rate equations describing our laser system.  The parameters in these equations
are the cavity lifetime, the dye fluorescence lifetime, and the times above
threshold pumping.  We do not have values for these parameters, nor did we
make a more accurate (not bandwidth limited) measurement of the laser pulse's
risetime.  Consequently, the subsequent analysis of the fluorescence must be
regarded as a free parameter fit.  Given that the grazing incidence
configuration is lossy, we assume that 50\% of the light is coupled out of the
cavity for each pass of the grating.  The resultant cavity lifetime is 330~ps.
The fluorescence lifetimes of the Rhodamine dyes are on the order of 5~ns, and
we take the times threshold pumping to be 1000.  This last parameter is the main
adjustable parameter and we varied it until the predicted pulses, when convolved
with our 500~MHz instrument response, closely resembled the measured pulse.  The
calculated dye laser pulse is shown in \Fig{pulse}.  The rising edge is very
rapid, with a risetime on the order of the cavity lifetime.  The rising edge
rings back to a smaller value and then the pulse essentially follows the smooth
(Gaussian) pump pulse.

\begin{figure}[tbp]
\postfull{thesisfigs/two_photon_expt/pulse.eps}
\bigskip
\ncap{Dye laser pulse with sharp rising edge.}{Dye laser pulse with sharp
rising edge numerically calculated using rate equation model of dye laser.
\label{pulse}}
\end{figure}

The results for the population transfer with this dye laser pulse is given in
\Fig{strong_ave_best}.  In this case the theoretical curve is shown with the
experimental curve, and the agreement is excellent.  The dip in the theoretical
curve near zero delay is no longer present.  Furthermore, the
theoretical population transfer enhancement has been decreased so that it is
very close to the experimental value.  And in general, both curves stay very
close to each other for all delays investigated.  The success of the modified
model is due to the sensitivity of the physical process on the adiabaticity of
the interaction.  The sharp rising edge represents a fast diabatic coupling
that could serve to destroy the adiabaticity, and consequently eliminate the
population transfer enhancement on the counterintuitive side.  However, the
adiabaticity criterion depends both on the time rate of change of the pulses
and on their energies.  Consequently, we are still able to observe adiabatic
evolution, evidenced by the improved transfer efficiency on the counterintuitive
side.

\begin{figure}[tbp]
\postfull{thesisfigs/two_photon_expt/strong_ave_best.eps}
\bigskip
\ncap{Integrated fluorescence signal obtained under strong field excitation
conditions with dye laser pulse with sharp rising edge.}{Integrated fluorescence
signal obtained under strong field excitation conditions.  In (a) is the
experimental curve, and in (b) the theoretical curve, where the dye laser pulses
are modeled as in \Fig{pulse}.
\label{strong_ave_best}}
\end{figure}

\section{Caveat}
\hspace{\parindent}  Having discussed the success of the counterintuitive
technique, we would be remiss not to mention the difficulties in using this
technique on a day-to-day basis with pulsed lasers \cite{Bergmann_caveat}.  The
counterintuitive transfer process requires the highest quality laser pulses.  In
this chapter, we have presented a dye laser with near ideal performance.  The
single-mode linewidth was close to the transform limit.  However, it must be
realized that if there is a second longitudinal mode, say with energy 1/100 of
the main mode, then this can produce modulation in the electric field with an
amplitude of 10\%.  The period of such modulation would be the same as the
cavity round trip time, $\approx 130$~ps.  This rapid modulation could cause the
evolution to be diabatic. Also, the spatial quality of the beams must be high
so that overlapping two beams in the interaction region is effective
\cite{Bergmann_caveat}.  Following this work, we developed a technique to smooth
the rising edge of the dye laser pulse \cite{Corless:97}, so this should improve
the technique.

We found that for all the experimental runs we performed, the intuitive
technique performed as well as the counterintuitive, on average.  And this is
given the high quality of the dye laser we used.  Typical nanosecond dye lasers
do not have the same high quality and performance as the system we used
(certainly not commercial systems).  So while we have demonstrated the improved
efficiency of the counterintuitive population transfer, from a practical
standpoint we would suggest the use of the intuitive technique, unless the
laser system used can meet these stringent requirements.  Furthermore, the
atomic transitions used must have a three-level structure.  Coupling
to other atomic levels can likewise eliminate the counterintuitive enhancement
\cite{Martin:95}.  While the intuitive technique cannot produce the highest
transfer efficiency, it requires lasers of modest quality and is less sensitive
to perturbing atomic levels.


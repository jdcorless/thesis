%   Thesis Abstract



%\documentstyle[12pt]{article}

%\begin{document}

%\begin{center}
%\Large\bf   
%   Efficient Excitation of Atomic Rydberg\\ 
%   States\\
%   John D. Corless
%\end{center}

%\bigskip

%\centerline{\Large\bf Abstract}
%\bigskip


\begin{singlespace}

\chapter*{Abstract\markboth{ABSTRACT}{ABSTRACT}}
\label{abstract}
\begpagestyle

\end{singlespace}

We have investigated the optical excitation of atomic Rydberg states with
emphasis on efficiency and selectivity.   We consider both direct excitation
from the ground to the Rydberg state as well as excitation via an
intermediate resonance.

In the direct excitation case, we find that a major limitation to the
selectivity of the process is the optical mixing of the nearly degenerate
Rydberg states with the same principal quantum number, but differing angular
momentum quantum numbers.  The interaction between these states is
characterized by Rabi frequencies that exceed the optical frequency even for
very modest optical field strengths.  This interaction gives rise to angular
distributions peaked orthogonal to the laser polarization direction,
emission of high harmonics of the laser field, as well as laser induced
stabilization.  We derive analytic results for this interaction as well as
develop a model based on Landau-Zener level-crossing theory.  Experimentally,
we observe this strong interaction between Rydberg states by examining
three-photon ionization in atomic potassium when a picosecond dye laser is tuned
through two-photon resonance with the Rydberg series.  The ionization signal
becomes suppressed when the optical mixing of the Rydberg states becomes large. 
The details of this suppression depend on the peak intensity of the laser field.

We then consider doubly-resonant excitation of Rydberg states and investigate
the dependence of the transfer efficiency on the time delay between the two
resonant laser pulses.  We find that even in the presence of Doppler broadening,
transverse spatial variation of the laser beam, and laser amplitude fluctuations
that the transfer efficiency from the ground state to the Rydberg state is
maximized when the laser pulses are applied in the counterintuitive order.  We
investigate these predictions experimentally in a three-level cascade system in
atomic sodium vapor and verify that the population transfer efficiency is
maximized in the counterintuitive regime.  We further find that in the
presence of laser amplitude fluctuations, the population transfer signal
fluctuates less when the pulses are applied in the counterintuitive order.

%\end{document}


\begin{singlespace}

\chapter{Two-Photon Excitation: Theory}
\label{citheory}
\begpagestyle
%\pagenumbering{arabic}

\end{singlespace}


In this chapter, we examine excitation from the initial ground state to a
final Rydberg state that proceeds through an intermediate resonance, modeling
the atom as an ideal three-level cascade system.  We examine theoretically
this three-level interaction mediated by two laser pulses as a function of
the time delay between the two excitation pulses.  This leads us into the
field of ``counterintuitive'' interactions \cite{Shore:95a} and we examine
this process in detail by including such experimental effects as Doppler
broadening, Gaussian spatial profiles, and laser amplitude fluctuations.  We
find that even in the presence of these effects, the population transfer is
maximized when pulses are applied in the counterintuitive order.  Finally,
extending this ideal three-level model to a Rydberg final state is discussed.

\section{Counterintuitive Pulses}
\label{CI}

\hspace{\parindent} Consider the general three level interaction shown in
\Fig{apt_levels}.  A ground state, $\ket{0}$, is near resonantly coupled to
an intermediate level, $\ket{1}$, by a field of frequency $\omega_1$, which
in turn is coupled by a field of frequency $\omega_2$, to a final state,
$\ket{2}$.  If the goal of the interaction is to transfer maximal population
from the initially populated ground state to the upper state, $\ket{2}$,
then the natural approach would be to use ``intuitive'' pulse ordering.  That
is, first a laser pulse resonant with the $\ket{0} \rightarrow \ket{1}$
transition takes population from the ground state to the middle state.
Then, after a short time delay (shorter than the spontaneous emission
lifetime of the middle state), the second pulse, resonant with the
$\ket{1} \rightarrow \ket{2}$ transition, takes the population from the
middle level to the upper level.  This step-wise approach is
traditionally used in this problem \cite{Stebbings_book}.  Under ideal
conditions (single atom, fully coherent fields), this method can achieve
complete population transfer if odd-$\pi$ pulses are used \cite{Allen:87}.

\begin{figure}[tbp]
\postfull{thesisfigs/two_photon/apt_levels.eps}
\bigskip
\ncap{Level scheme for the three-level interaction.} {Level scheme for the
three-level interaction excited by two laser fields.
\label{apt_levels}}
\end{figure}

The possibility of counterintuitive pulse ordering was suggested by Oreg
\etal \cite{Oreg:84}.  In this work, it was shown (using the group-theoretic
technique of the SU($N$) coherence vector of an $N$-level system
\cite{Hioe:81}) that adiabatic transfer to the final state could be achieved
by applying the driving fields in the opposite order, where the second
transition pulse interacts with the atom first.  This effect was first
experimentally observed by Gaubatz \etal \cite{Gaubatz:88}.  In this
experiment, single-mode CW dye lasers intersected an Na$_2$ molecular beam at
right angles.  By displacing the two laser beams with respect to each other
along the molecular beam axis, a time delay between the two laser beams is
experienced by the molecules in the molecular beam.  This experiment
demonstrated efficient and selective population transfer to highly excited
vibrational levels of the Na$_2$ ground electronic state and was later examined
in greater detail \cite{Gaubatz:90}, with transfer efficiencies approaching
100\%.  Subsequent experimental work using the hyperfine resolved Cs D$_2$ line
showed that when the ideal three-level system is coupled to other levels, the
population transfer efficiency can be reduced \cite{Pillet:93}.  A transfer
efficiency of 50\% was observed due to coupling among magnetic sublevels.  An
extensive study of the effects of magnetic sublevels in a degeneracy-lifting
magnetic field has recently been reported in three parts
\cite{Shore:95b,Martin:95,Martin:96}.  Counterintuitive delays have also been
used to transfer momentum efficiently to the center-of-mass motion of atoms in
laser cooling \cite{Lawall:94,Goldner:94}, which has led to the development of
an atomic interferometer based on adiabatic population transfer
\cite{Weitz:94}.

The use of pulsed lasers rather than spatially displaced CW lasers was first
demonstrated with NO molecules \cite{Schiemann:93} obtaining nearly 100\%
transfer efficiency.  An experiment with SO$_2$ molecules extended the pulsed
laser work by demonstrating efficient transfer in a polyatomic molecule and
the existence of the ``dark'' resonance in the pulsed regime.  Efficient
population transfer in a cascade system has been demonstrated with a chirped
ultrashort laser pulse in Rb where the direction of the chirp produces a
counterintuitive sequence of resonances in a three-state anharmonic
ladder \cite{Broers:92}.

Following the original theoretical work by Oreg \etal \cite{Oreg:84},
Kuklinski \etal \cite{Kuklinski:89} showed the connection between this special
type of time-dependent population transfer and some of the classic results of
three-level atoms in CW laser fields \cite{Arimondo:76,Gray:78}.  Conditions
for adiabatic evolution were also established.  Subsequent theoretical work
included analytic solutions for special classes of pulse shapes
\cite{Carroll:90} and analytic results for multilevel systems
\cite{Coulston:92}.  A relative comparison among the various excitation
techniques (including counterintuitive pulse sequences) has been made
\cite{He:90,Shore:92}.  Kuhn \etal investigated the effects of
non-transform-limited laser performance on the transfer efficiency and found
the efficiency to be degraded strongly for spectrally broad light
\cite{Kuhn:92}.  Counterintuitive pulses have also been suggested to
transfer population efficiently between discrete states using a continuum as
the intermediate state \cite{Carroll:92}.

While we shall consider optically thin media (Beer's absorption length longer
than the propagation length), some interesting results have been predicted and
observed in these three-level systems in optically thick media, most notably
the phenomenon of electromagnetically induced transparency \cite{Harris:89}. 
This effect makes use of a coupling laser that strongly dresses one
transition producing ac-Stark split states that interfere to produce a minimum
in the absorption of a probe laser that (weakly) drives the second transition. 
This coupling laser in effect renders an initially opaque medium transparent
to the probe laser, and has been experimentally observed \cite{Boller:91}. 
Analysis of propagation dynamics has yielded predictions of matched pulse
production \cite{Harris:93}.  Normal modes for this propagation were identified
\cite{Harris:94} and the existence of dressed-field pulses (where the field's
evolution in space is equivalent to the dressed atoms evolution in time) was
demonstrated \cite{Eberly:94}.  Analytic predictions of solitary wave formation
\cite{Hioe:94} and the prediction of a special class of pulses termed
``adiabatons'' \cite{Grobe:94} have been predicted in this vein.  Eberly
\cite{Eberly:95} discussed the importance of counterintuitive pulse ordering for
these propagation problems.  Another on-going research effort in driven
three-level systems is the closely related phenomenon of lasing without
inversion which has been experimentally demonstrated \cite{Zibrov:95}.

The physics behind these counterintuitive pulse sequences can be understood
by examining the Hamiltonian (in the rotating wave approximation) of the
three-level system in
\Fig{apt_levels} driven by two laser fields
% HAMILTONIAN
\begin{equation}
\hat{H} = \left[ 
\begin{array}{ccc}
0 & \Omega_1 & 0  \\
\Omega_1 & 0 & \Omega_2 \\
0 & \Omega_2 & 0
\end{array} \right],
\label{hamil}
\end{equation}
where we assume that we have two-photon and single-photon resonance, $\Delta_1
= \Delta_2 = 0$.  If we diagonalize this Hamiltonian, then one of the three
dressed state solutions has zero eigenvalue and is given by
\begin{equation}
\ket{\lambda_0} = \cos\left(\theta\right)\ket{0} -
\sin\left(\theta\right)\ket{2}
\end{equation}
with
\begin{equation}
\tan\theta = \frac{\Omega_1}{\Omega_2}.
\end{equation}
This is the familiar trapped, or dark state, which is the
source of coherent population trapping \cite{Gray:78}.  One of the most
important features of this state is that it contains no admixture of the
middle state, $\ket{1}$.  If the level structure is in the
$\Lambda$-configuration, or if $\ket{2}$ is a metastable state in
\Fig{apt_levels}, then the fluorescence from this system can be turned off
by driving the atom to the $\ket{\lambda_0}$ state.  That is why this
state is referred to as the dark state, and typically a minimum in the
fluorescence is seen as one tunes through two-photon resonance.  The other
two dressed states contain admixtures of all three levels.

So far, we have not discussed the time-dependence of the fields.  If
$\Omega_1,\Omega_2$ are allowed to vary in time, then
$\ket{\lambda_0}$ becomes an adiabatic solution of the time-dependent
Hamiltonian.  In this case,
\begin{equation}
\ket{\lambda_0} \rightarrow \ket{\lambda_0(t)}= \cos\left[\theta(t)\right]
\ket{0} - \sin\left[\theta(t)\right]\ket{2}
\label{dark}
\end{equation}
and
\begin{equation}
\tan\theta \rightarrow \tan\left[\theta(t)\right] = \frac{\Omega_1(t)}
{\Omega_2(t)}.
\label{theta}
\end{equation}
This adiabatic dressed state can be used to transfer
population to the upper state \cite{Kuklinski:89}.  By examining
\Eq{dark} we see that if $\theta(t) \rightarrow 0$ as $ t \rightarrow -\infty
$ then $\ket{\lambda_0}$ corresponds with the ground state at early times.
Furthermore, if $\theta(t) \rightarrow \pi/2$ as $ t \rightarrow \infty $, then
$\ket{\lambda_0}$ corresponds with the upper state at later times.  Then using
\Eq{theta} we see that these asymptotic limits are realized if $\Omega_2(t)$ is
applied to the atom {\em before} $\Omega_1(t)$.  This sequence of pulses has been
called counterintuitive, being opposite ordering to the
intuitive scheme.  Such adiabatic population transfer schemes in
$\Lambda$-configuration level structures are called STIRAP
(\underline{sti}mulated \underline{R}aman \underline{a}diabatic
\underline{p}assage) processes, but this terminology seems poorly suited for
our cascade geometry.

Thus if the systems time dependent wavefunction satisfies
\begin{equation}
\ket{\Psi(t)} \approx \ket{\lambda_0(t)}
\end{equation}
for all times $t$, and the counterintuitive order of
pulses is applied to the atom, then $\ket{\Psi(t)}$ begins as the ground state
at early times, and evolves into the upper state at later times, transferring
population to the final state without populating the middle state.  We now need
to explore the restrictions on our counterintuitive pulses to ensure an
adiabatic interaction. This has been studied previously \cite{Kuklinski:89}
with the interesting result that the condition for adiabaticity depends not only
on the duration of the applied laser pulses, but also on their energies.  We can
understand this result by making a time-dependent unitary transformation to the
adiabatic basis.  First we must know the time-dependent eigenvalues of the
Hamiltonian.  By diagonalizing
\Eq{hamil}, we obtain the adiabatic dressed state energies
\begin{eqnarray}
\lambda_{+}(t) &=& \sqrt{\Omega_1^2(t) + \Omega_2^2(t)} \\
\lambda_{0}(t) &=& 0 \\
\lambda_{-}(t) &=& -\sqrt{\Omega_1^2(t) + \Omega_2^2(t)}.
\end{eqnarray}
These eigenvalues are plotted in \Fig{lips} for the symmetric case of equal
peak Rabi frequencies. For early and late times, the system (in the rotating
frame) is three-fold degenerate.  As the Rabi frequency of the first pulse
grows, the dressed state energies separate.  The separation between two
neighboring levels at any time is given by $\lambda_+(t)$.

\begin{figure}[tbp]
\postfull{thesisfigs/two_photon/lips.eps}
\bigskip
\ncap{Time-dependent dressed state energies.} {Time-dependent dressed state
energies for three-level system interacting with two delayed pulsed fields.
The two pulses create equal Rabi frequencies, and are separated by
approximately a pulsewidth.
\label{lips}}
\end{figure}

The corresponding eigenvectors associated with these adiabatic dressed state
energies are
\begin{eqnarray}
\ket{\lambda_{+}(t)} &=& \frac{\sin\theta(t)\ket{0} + \ket{1}
+\cos\theta(t)\ket{2}}{\sqrt{2}} \\
\ket{\lambda_{0}(t)} &=& \cos\theta(t)\ket{0} - \sin\theta(t)\ket{2} \\
\ket{\lambda_{-}(t)} &=& \frac{\sin\theta(t)\ket{0} - \ket{1} +
\cos\theta(t)\ket{2}}{\sqrt{2}}.
\end{eqnarray}
Then, if these $\ket{\lambda}$ states are used as the basis states (by
means of a time-dependent unitary transformation), the Hamiltonian
in the adiabatic state basis becomes \cite{Khidekel:95}
\begin{equation}
\hat{H}_{adiabatic}(t) = 
\left[ \begin{array}{ccc}
\lambda_+(t) & 0 & 0  \\
0 & \lambda_0(t) & 0 \\
0 & 0 & \lambda_-(t)
\end{array} \right] -i\frac{\dot{\theta}(t)}{\sqrt{2}} 
\left[ \begin{array}{ccc}
0 & -1 & 0  \\
1 & 0 & 1 \\
0 & -1 & 0
\end{array} \right]
\end{equation}
with
\begin{equation}
\dot{\theta}(t) = \frac{\dot{\Omega}_1(t)\Omega_2(t) -
\Omega_1(t)\dot{\Omega}_2(t)}{\Omega_1^2(t) + \Omega_2^2(t)}.
\end{equation}
If the term in $\dot{\theta}(t)$ can be neglected, then there will be no
coupling between our adiabatic states.  Therefore, $\dot{\theta}$ represents
the strength of the non-adiabaticity.  If $\dot{\theta}$ is much
smaller than the energy separation of the adiabatic dressed state energies,
then the evolution is approximately adiabatic.  Specifically, the condition
for adiabatic evolution is
\begin{equation}
\dot{\theta}(t) \ll \lambda_{+}(t)
\end{equation}
or equivalently
\begin{equation}
\frac{\dot{\Omega}_1\Omega_2 -
\Omega_1\dot{\Omega}_2}{\left(\Omega_1^2 + \Omega_2^2\right)^{3/2}} \ll 1.
\end{equation}
This has been shown \cite{Kuklinski:89} to be equivalent to
\begin{equation}
\left(\sqrt{\Omega_1^2 + \Omega_2^2}\right)\tau_p \gg 1
\label{adiab_criterion}
\end{equation}
with $\tau_p$ the pulse duration.  As mentioned previously, this adiabatic
condition depends on both the duration and the energy of the pulses.
Therefore, a short pulse can still satisfy the adiabaticity criterion
if it has sufficient energy.

\section{Theoretical Formalism}

\hspace{\parindent} In this section, we introduce the general theoretical
formalism that will be used to describe the counterintuitive pulse dynamics. 
We assume the level structure as in
\Fig{apt_levels}.  The equations of motion for the density matrix
are \cite{Allen:82}
\begin{equation}
\dot{\rho_{ll}} = -\Gamma_{ll}\rho_{ll} + \sum_{E_i > E_l} A_{il}\rho_{ii}
+ \frac{i}{\hbar}\sum_{i} \left[d_{li}\rho_{il}-d_{il}\rho_{li}\right]E(t)
\end{equation}
for the populations, and
\begin{equation}
\dot{\rho_{ml}} = \left(-i\omega_{ml}-\Gamma_{ml}\right)\rho_{ml}
+\frac{i}{\hbar}\sum_{i}\left[d_{mi}\rho_{il} - d_{il}\rho_{mi}\right]E(t)
\end{equation}
for the coherences.  Here, $A_{il}$ is the Einstein ``A'' coefficient of the
$i \rightarrow l$ transition,
\begin{equation}
\Gamma_{ll} = \sum_{E_i < E_l}A_{li},
\end{equation}
and
\begin{equation}
\Gamma_{lm} = \frac{1}{2}\left(\Gamma_{ll} + \Gamma_{mm}\right).
\end{equation}

Specializing these equations to our three-level problem, we introduce 
the slowly varying amplitudes
\begin{eqnarray}
x_{ll} & = & \rho_{ll} \\
x_{01} & = & \rho_{01}\exp(-i \omega_1 t) \\
x_{02} & = & \rho_{02}\exp\left[-i\left(\omega_1 + \omega_2\right)t\right] \\
x_{21} & = & \rho_{21}\exp(i\omega_2 t).
\end{eqnarray}
The applied (bichromatic) field is taken to be $E = E_1\left({\rm e}^{-i
\omega_1 t} + c.c.\right) + E_2\left({\rm e}^{-i\omega_2 t} + c.c.\right)$.
The equations of motion in the rotating wave approximation are then given by
\begin{eqnarray}
\dot{x_{00}} & = & A_1 x_{11} + i (x_{10} - x_{01})\Omega_1 \\
\dot{x_{01}} & = & \left(i\Delta_1 -\frac{A_1}{2}\right)x_{01} +
i (x_{11}-x_{00})\Omega_1 -i x_{02} \Omega_2 \\
\dot{x_{02}} & = & \left[i(\Delta_1+\Delta_2) - \frac{A_2}{2}\right]x_{02} +
i x_{12} \Omega_1 -i x_{01} \Omega_2 \\
\dot{x_{11}} & = & -A_1 x_{11} + A_2 x_{22}
+i (x_{01}-x_{10})\Omega_1 +i (x_{21}-x_{12})
\Omega_2 \\
\dot{x_{21}} & = & \left[-i\Delta_2 +\frac{1}{2}(A_1+A_2)\right]x_{21} +
i (x_{11}-x_{22})\Omega_2 - i x_{20}\Omega_1 \\
\dot{x_{22}} & = & -A_2 x_{22} + i (x_{12} - x_{21})\Omega_2
\end{eqnarray}
where $\Omega_1 = d_1 E_1$ and $\Omega_2 = d_2 E_2$ are the transition Rabi
frequencies . The single-photon detunings are $\Delta_1 = \omega_{10} -
\omega_1$ and $\Delta_2 = \omega_{21} - \omega_2$.  The detuning from
two-photon resonance is $\Delta_1 + \Delta_2$.

This theory will enable us to investigate the driven atomic response.  The
presence of the Einstein ``A'' coefficients enables us to correctly model
spontaneous emission.  Furthermore, the density matrix equations of motion can
be readily generalized to include other homogeneous broadening mechanisms by the
addition of phenomenological decay terms.  Doppler broadening can be modeled
with the appropriate detuning terms, and finally, by making the Rabi frequencies
space-dependent, we can model the spatial profile of the exciting lasers.

\section{Single-Atom Response}
\hspace{\parindent} In the next chapter we shall present experimental results
obtained in sodium vapor.  To facilitate comparisons with theory, we will
specialize our formalism here to the $3S_{1/2} \rightarrow
3P_{1/2} \rightarrow 4D_{3/2}$ cascade system in sodium, with laser
wavelengths of $589.755$ nm ($3S_{1/2} \rightarrow 3P_{1/2}$) and $568.421$
nm ($3P_{1/2} \rightarrow 4D_{3/2}$) \cite{Moore:71}, and with Einstein ``A''
coefficients of $A_1 = ($15.9~ns$)^{-1}$ and $A_2 = (91.7$~ns$)^{-1}$
\cite{Wiese:69}.

Since our goal is to maximize population in the final state, we must identify
an experimental parameter to monitor this population information.  There
is a slight subtlety in this choice.  At issue is the fact that our
final state has a finite spontaneous emission lifetime.  The experiment consists
in varying the delay between two resonant laser pulses and measuring the upper
state population for each delay.  Consider that the orange
($589.755$ nm) pulse is fixed with respect to some time axis with its peak at
$t=0$, and the green ($568.421$ nm) pulse is delayed in time with peak occurring
at time $t_G$.  From a very crude standpoint, the population is transferred
to the upper state when the green pulse is ``on'', at $t_G$.  If we choose some
arbitrary time, $t_f$, to measure the upper state population, then
the population can vary depending on the delay between $t_f$ and $t_G$.  So
while the peak transfer of population may be the same at two different values
of $t_G$, the population at $t_f$ may be different.  Notice that if the upper
state had an infinite lifetime, this would not be an issue (and is not an issue
when studying counterintuitive pulses in the $\Lambda$-configuration
\cite{He:90}).  This might be remedied by changing $t_f$ for every value of
$t_G$ so that their difference is kept fixed, fixing the time axis with respect
to the green pulse instead of the orange.  This solves the problem for pulses
with duration much shorter than the decay time of the upper state (91.7 ns).  In
our case, we use pulses with full width at half maximum (FWHM) durations of
$\tau_{\scriptscriptstyle \!F\!W\!H\!M}
\sim 15$ ns and so this does not entirely solve the problem
($\exp[-\tau_{\scriptscriptstyle \!F\!W\!H\!M}A_2] =
\exp[-15/91.7] \approx 0.85$).

To remedy this situation, we chose to integrate the fluorescence from the
$4D_{3/2} \rightarrow 3P_{1/2}$ transition during the entire interaction.  At
any given time the fluorescence is proportional to the population in state
$\ket{4D_{3/2}}$.  The issue then is to demonstrate that the
integrated fluorescence is a meaningful measure of transferred
population.  In order to show this, we first consider the situation where
spontaneous emission is neglected.  Both laser pulses are modeled with a
Gaussian electric field envelope, $f_1(t) = \exp\left[-t^2/2\sigma_p\right]$
for the orange pulse and $f_2(t) = \exp\left[-(t+\Delta
t)^2/2\sigma_p\right]$ for the green. The pulse duration is $\tau_{
\scriptscriptstyle \!F\!W\!H\!M} = 2
\sqrt{\log_e(2)} \sigma_p = 15$ ns.  Note that $\Delta t > 0$ corresponds to a
counterintuitive pulse sequence with these definitions.  The dipole moments
for the transitions are $\bra{3S_{1/2}}\hat{z}\ket{3P_{1/2}} = d_1 = 1.45$
a.u. and $\bra{3P_{1/2}}\hat{z}\ket{4D_{3/2}} = d_2 = 0.813$ a.u.
\cite{Wiese:69}.  Even though the dipole moments are different, we
shall take the associated Rabi frequencies to be equal for simplicity (this does
not have a large impact as the adiabatic criterion is a threshold condition and
therefore, exact values of the Rabi frequency are not as important as their
relationship to the threshold).

Since the fluorescence is proportional to the $\ket{4D_{3/2}}$ state at any
time, we chose to model the fluorescence integration by integrating the
population in the $\ket{4D_{3/2}}$ state.  We shall use the terms
``time-integrated fluorescence'' and ``time-integrated population''
interchangeably because of their proportionality.  Also, to ensure that the
interaction had completed, we performed this integration from a very early
time ($t_i = -1000$ ns) to a very late time ($t_f = 1000$ ns).  The
time-integrated fluorescence is given by
\begin{equation}
F = \int_{t_i}^{t_f} x_{22}(t) dt.
\end{equation}
Note, as defined, this has units of time.

\begin{figure}[tbp]
\postscript{thesisfigs/two_photon/nodecay.eps}{.5}
\bigskip
\ncap{Comparison of instantaneous to time-integrated population in the absence
of decay.} {Comparison of (a) instantaneous to (b) time-integrated population
in the absence of decay, plotted against delay between two resonant
pulses.  Positive time delays (measured in units of pulse standard deviation)
correspond to counterintuitive pulse ordering.  Each pulse area is $ 10 \pi$.
\label{nodecay}}
\end{figure}

In \Figure{nodecay} we show the comparison between the final upper-state
population (as measured at $t_f$) and the time-integrated population for
pulses having areas of $10 \pi$.  The two curves display the characteristic
response of the three-level system to these strong pulses.  On the intuitive
side, the population shows oscillation structure versus delay.  For very early
times, there is no population transfer due to our choice of the area being an
even multiple of $\pi$.  When the pulses are well separated on the intuitive
side, the orange pulse leaves the system in its ground state, and hence the green
pulse cannot transfer any population to the final state.  The counterintuitive
side displays a flat region of population transfer near one.  For large
counterintuitive delays, the population goes back to zero.  This is not an
artifact of the even number of $\pi$ chosen for the interaction, but represents
the fact that if the pulses are not overlapping to some degree, the original
intuition which told us the counterintuitive ordering shouldn't work is
correct.  The magnitude of the integrated population in \Fig{nodecay}(b)
demonstrates that the entire population is transferred to the upper state near
$t=0$ and this population ($=1$) is integrated in time from $t=0$ to
$t=t_f=1000$ ns.  Therefore, the value of the integrated population is nothing
more that the total population transferred times the integration time.

\begin{figure}[tbp]
\postfull{thesisfigs/two_photon/decay.eps}
\bigskip
\ncap{Comparison of instantaneous to time-integrated population in the
presence of decay.}{Comparison of (a \& b) instantaneous to (b \& d)
time-integrated population in the presence of decay, plotted against delay
between two resonant pulses.  Positive time delays (measured in units of pulse
standard deviation) correspond to counterintuitive pulse ordering.  Each pulse
area is $10\pi$.  In (a) and (b) the origin of time (not time delay) is
defined with respect to the orange pulse and in (c) and (d) with respect to
the green pulse.
\label{decay}}
\end{figure}

The general shapes of the two curves are very similar, and this gives good
reason to infer information about the population from the integrated
fluorescence.  However, we must investigate this relationship in the presence of
decay, and we will further explore the issue of the time choice for measuring
the final state population, discussed previously. 
In \Fig{decay}(a) and \Fig{decay}(b), the orange pulse is held fixed and the
population is measured at time $t_f = 1000$ ns after the peak of the orange
pulse.  Notice that the population in \Fig{decay}(a) is very small.  This is
because the population has had $1000/91.7 \approx 10.9$ lifetimes to decay
($\exp(-10.9) \approx 1.8 \times 10^{-5}$).  Such a long delay guarantees that
the atom is no longer interacting with the laser pulses, and that essentially
all the population has been integrated.   The magnitude of $F$ in \Fig{decay}(b)
represents the fact that essentially the entire population is transferred to the
upper state, which then decays to the middle level in a time $1/A_2$.  So, $F
\approx {\displaystyle \int}_0^{t_f} x_{22}(0) \exp(-A_2 t)dt = x_{22}(0)/A_2
=x_{22}(0) \times 91.7$ ns.  Therefore, for $x_{22}(0)\approx 1$, $F$ will have
a maximum signal near $91.7$ ns, which is the value obtained in \Fig{decay}(b).

The first dramatic feature to notice about both \Fig{decay}(a) and
\Fig{decay}(b) is that the population is larger on the counterintuitive side
than on the intuitive.  This is in contrast to \Fig{nodecay} where the
two sides achieved very similar maximum population transfer.  This discrepancy
is due primarily to the decay of the middle state.  In the presence of decay,
pulse area is not a particularly meaningful quantity.  For instance, a CW laser
field may have infinite area, but in the long-time limit, only 50\% of the
population in a two-level system will be in the excited state \cite{Allen:87}. 
Consequently, the population transfer on the intuitive side suffers from this
decay.  The counterintuitive scheme transfers population via the dark state
(\Eq{dark}), which has no admixture of the decaying middle level and thus is
unaffected by the spontaneous emission.

Upon closer examination of \Fig{decay}(a), we notice that what was a flat
constant population on the counterintuitive side in the no-decay case of
\Fig{nodecay}(a) now has an upward rising tendency for increasing pulse delay.
Again, this is due to the choice of $t_f$ being a fixed time with respect to
the orange pulse.  For larger delays, the time difference between the green
pulse and $t_f$ decreases and so the population increases, having had less
time to decay.  The integrated population in \Fig{decay}(b) does not suffer
from this complication.

Given that this slope feature in \Fig{decay}(a) seems like an artifact of the
measurement technique, we propose to measure time with respect to
the green pulse peak, vary the orange pulse in time, and measure the population
at the fixed time $t_f$ (now defined with respect to the green pulse peak). 
This is the approach taken in \Fig{decay}(c) and
\Fig{decay}(d).  This actually overcompensates for the upward rise by
producing a slight downward trend for increasing pulse delay.  This shows that
the actual time ``when'' population is transferred is not a well defined
quantity and is a crude approximation mentioned previously.  Again, if the
pulse duration was much shorter than this decay time, this would not be an
issue, but we do experience this difficulty with our 15 ns pulses. 
However, the integrated population in \Fig{decay}(d) is essentially identical to
that shown in \Fig{decay}(b).  Furthermore, it displays the nice flat region on
the counterintuitive side, and the general shape is similar to the population
curves (save for the flat region).  We have therefore demonstrated that the
time-integrated fluorescence (population) is a more desirable measure of
population transfer than the instantaneous population measured after
the interaction, when there is final state decay.

\begin{figure}[tbp]
\postfull{thesisfigs/two_photon/N10and11.eps}
\bigskip
\ncap{Comparison of time-integrated population for odd and even pulses
areas.}{Comparison of time-integrated population for pulses with area $10 \pi$
(solid line) and $11 \pi$ (dashed line).
\label{N10and11}}
\end{figure}

The response shown in \Fig{decay}(b) is generally indicative of the
single-atom response of this system, for pulse areas even, and greater than
$\sim 6\pi$. Odd area pulses have a slightly modified intuitive side as shown
in \Fig{N10and11}.  Physically this occurs because for large intuitive delays,
whether the first pulse inverts the lower transition or not determines whether
the second pulse will have the opportunity to invert the second
transition.  \Figure{N10and11} shows a longer tail on the intuitive side for the
odd $\pi$ pulse (dashed line).  In general, for larger areas, more interference
oscillations are seen on the intuitive side, while the width of the main
counterintuitive feature grows slightly.  As mentioned before, this is because
the adiabatic criterion is a threshold condition, and hence the dynamics in the
counterintuitive regime is essentially the same for all areas satisfying
\Eq{adiab_criterion}.  We will see later that this causes the counterintuitive
transfer process to be relatively insensitive to laser amplitude fluctuations
compared to the intuitive transfer.  For pulse areas less than $\sim 6\pi$, the
shape is somewhat different.  The evolution stops being adiabatic and the
population transferred on the counterintuitive side decreases until again the
original intuition correctly predicts very little population transfer for the
counterintuitive pulse ordering.  The intuitive side loses evidence of the
interference structure and it too gradually decreases in its efficiency of
population transfer.  The intuitive ordering, however, achieves much greater
transfer efficiency than the counterintuitive ordering when the pulses are weak.

\section{Doppler Broadening}

\hspace{\parindent} The experiments performed so far on this type of adiabatic
population transfer have been carried out in molecular (atomic) beam
geometries.  Recall that in this geometry, CW dye lasers intersect the molecular
beam at right angles, with the two laser beams being separated by an amount that
determines the time delay seen by the molecules.  The laser frequency experienced
by an atom varies depending on its motion relative to the laser, a phenomenon
referred to as the Doppler effect, with the resultant broadening of the
absorption line termed Doppler broadening \cite{Loudon_book}.  Typical atomic
velocities at
$\sim 200^{\circ}$C are 1 mm/$\mu$s = 10$^{-6} c$ (with $c$ the speed of light)
so relativistic descriptions of the Doppler effect need not be
considered.  The shift in frequency seen by the moving atom is given by
\begin{equation}
\delta\omega = \vec{k} \cdot \vec{v}
\end{equation}
with $\vec{k}$ the wavevector of the laser beam, and $\vec{v}$
the atom's velocity, a stochastic variable.  For fixed laser beam
direction, the distribution of frequency shifts will be determined by the
distribution of $ v_k = \hat{k} \cdot \vec{v}$, which is given by the
Maxwell-Boltzmann distribution ($\hat{k}$ is the unit vector in
the direction of $\vec{k}$). The geometry of a laser beam interacting with a
molecular beam at right angles then has the advantage that there is essentially
no atomic motion in the direction of the wavevector of the laser,
and hence no Doppler broadening of the excited transitions.
However, under some circumstances, simple absorption cells are preferable to
molecular beams.  In these cells, there is no preferred direction for atomic
motion, and Doppler broadening will be present.

There is some cause to be concerned about the effect of Doppler broadening on
the counterintuitive population transfer technique.  The theory considered in
\rSection{CI} assumed both one-photon and two-photon
resonance conditions.  The single-photon condition was mainly for mathematical
convenience, because $\ket{\lambda_0(t)}$ is still an adiabatic solution of the
time-dependent Hamiltonian for $\Delta_1 \neq 0$.  However, for
non-zero two-photon detuning, $\Delta_{TP} = \Delta_1 + \Delta_2 \neq 0$, the
state $\ket{\lambda_0(t)}$ is no longer an adiabatic solution and state
$\ket{1}$ gets mixed in to all dressed states.

We must therefore consider the Doppler shift of the two-photon transition. 
For an atom with velocity $\vec{v}$, the effective resonance frequency of the
lower transition is
\begin{equation}
\omega_{10}(\vec{v}) = \omega_{10}(0) + \vec{k}_1 \cdot \vec{v}
\end{equation}
and for the second transition
\begin{equation}
\omega_{21}(\vec{v}) = \omega_{21}(0) + \vec{k}_2 \cdot \vec{v}.
\end{equation}
Since the resonance detunings are
\begin{eqnarray}
\Delta_1(\vec{v}) = \omega_{10}(\vec{v}) - \omega_1 = \Delta_1(0) + \vec{k}_1
\cdot
\vec{v}
\\
\Delta_2(\vec{v}) = \omega_{21}(\vec{v}) - \omega_2= \Delta_2(0) + \vec{k}_2
\cdot
\vec{v}
\end{eqnarray}
the two-photon detuning is
\begin{equation}
\Delta_{TP} = \Delta_1(0) + \Delta_2(0) + (\vec{k}_1+\vec{k}_2) \cdot \vec{v}.
\end{equation}
If $\vec{k}_1$ is chosen to be anti-parallel to $\vec{k}_2$ (and to have nearly
equal magnitudes), then the two-photon Doppler shift can be very small.  This
effect is referred to as Doppler-free two-photon spectroscopy
\cite{Corney_book}.  However, if the laser beams are copropagating so that
$\vec{k}_1 \parallel \vec{k}_2$, then the Doppler shifts add.  This more
troublesome situation is investigated theoretically here and
experimentally in the next chapter.

In order to model Doppler broadening of the cascade transition, we will
use the probability distribution of frequency shifts.  The
Maxwell-Boltzmann distributions of velocities is Gaussian and is given
by 
\begin{equation}
P(v) = \frac{1}{\sqrt{2 \pi \sigma_{T}^2}} \exp\left(-v^2/2 \sigma_T^2 \right)
\end{equation}
where $v = \vec{v} \cdot \vec{k}$, and the standard deviation of this
distribution is
\begin{equation}
\sigma_T = \sqrt{\frac{k_B T}{M}},
\end{equation}
where $k_B$ is Boltzmann's constant, $T$ is the temperature of the atomic
ensemble, and $M$ is the mass of an atom.  From this we define a ``thermal
velocity'',
\begin{equation}
v_T = \sqrt{8 \log_e(2)} \sigma_T = \sqrt{\frac{8 \log_e(2) k_B T}{M}}
\label{thermal_v}
\end{equation}
such that $P(v_T/2) = 1/2$. The probability distribution of frequency shifts then
is also Gaussian and is given by
\cite{Loudon_book}
\begin{equation}
P_D(\delta\omega) =
\frac{1}{\sqrt{2\pi\sigma_D^2}}\exp\left[-\delta\omega^2/2\sigma_D^2\right]
\end{equation}
with
\begin{equation}
\sigma_D = \omega_0 \sqrt{\frac{k_B T}{M c^2}}.
\end{equation}
In this equation, $\omega_0$ is the (angular) transition frequency of atoms with
$\vec{v} =0$.  For the $\ket{3S_{1/2}} \rightarrow
\ket{3P_{1/2}}$ transition, the FWHM Doppler width at $T=200^{\circ}$C is
approximately 1.6~GHz.  The shift scales as $\sqrt{T}$ and thus changes slowly
with temperature.  Since the Doppler width is inversely
proportional to the center wavelength of the transition, we can immediately
infer that for the $\ket{3P_{1/2}} \rightarrow \ket{4D_{3/2}}$ transition, the
Doppler width is $(589.755/568.421)\times1.6$~GHz $\approx 1.038 \times
1.6$~GHz.

Let $\delta\omega$ be the Doppler shift produced on the lower transition
by an atom moving with velocity $v$.  Then the shift on the upper transition is
$1.038 \times \delta\omega$ and so we can average over the single variable
$\delta\omega$.  Let the integrated fluorescence signal from atoms
with a lower transition shift of $\delta\omega$ be written
$F(\delta\omega)$. Then the total signal measured from a vapor cell will be an
ensemble average over the collection of atoms, or equivalently, the
distribution of Doppler shifts.  That is,
\begin{equation}
\langle F \rangle_D = \int_{-\infty}^{\infty}P_D(\delta\omega)
F(\delta\omega) d(\delta\omega)
\label{dopp_eq}
\end{equation}
where $\langle \cdots \rangle_D$ denotes the ensemble average over the
moving atoms (\underline{D}oppler broadening).

\begin{figure}[tbp]
\postfull{thesisfigs/two_photon/N11dopp.eps}
\bigskip
\ncap{Doppler averaged time-integrated population signal as a function of
delay between pulses.}{Doppler averaged time-integrated population signal as a function
of delay between pulses.  Pulse area (on resonance) is $11 \pi$.  Points `A'
and `B' mark reference time delays that will be examined in \Fig{lw}.
\label{N11dopp}}
\end{figure}

In \Fig{N11dopp}, we show the results of the Doppler average of \Eq{dopp_eq}
for pulses whose on resonance area is $11 \pi$.  We see from this curve that
not only does the counterintuitive remain the more efficient population
transfer scheme, the enhancement of transfer efficiency on the
counterintuitive side relative to the intuitive is improved in
the presence of Doppler broadening.  This can be seen by examining
\Fig{N10and11} and \Fig{N11dopp}.  Whereas in the absence of Doppler
broadening, this enhancement is $\sim 300\%$, in the presence of Doppler
broadening, this number is increased to $ \sim 500\%$.

\begin{figure}[tbp]
\postscript{thesisfigs/two_photon/lw.eps}{.5}
\bigskip
\ncap{Time-integrated fluorescence signal as a function of the velocity of the
atom.}{Time-integrated fluorescence signal as a function of the velocity of the
atom.  In (a) is the intuitive time delay labeled `A' in \Fig{N11dopp} and in
(b) is the counterintuitive time-delay labeled `B' in \Fig{N11dopp}.  Pulse
area is $11\pi$.
\label{lw}}
\end{figure}

To further investigate this enhancement, we calculated the integrated
fluorescence signal as a function of velocity of the atoms.  This is a special
sort of linewidth which is not exactly the same as traditional linewidths.  To
achieve the same sort of information by tuning a laser frequency, one would have
to change $\Delta_1$ and $\Delta_2$ in such a way that $\Delta_2 = 1.038 \times
\Delta_1$.  This is the tuning that is realized as one moves through 
velocity in the ensemble of atoms.  We consider two time delays in
\Fig{N11dopp}, one far on the intuitive side (labeled ``A''),
and one near the peak on the counterintuitive side (labeled ``B'').  For
these time delays, we calculated the time-integrated fluorescence as a function
of atomic velocity, and the result is shown in
\Fig{lw}.  \Figure{lw}(a) shows this special linewidth for position ``A'' in
\Fig{N11dopp} and \Fig{lw}(b) shows this linewidth for position ``B'' in
\Fig{N11dopp}.  The velocity is measured in terms of the thermal velocity
defined in \Eq{thermal_v}.  The figure shows that indeed the counterintuitive
delay has a broader linewidth as a function of velocity than does the
intuitive.  Notice that $v$ is much smaller than $v_T$ in these figures and
hence the probability of an atom having a velocity shown within the linewidths
in \Fig{lw} is essentially constant.  The width of these features is similar to
the homogeneous linewidth of a traditional (laser-tuned) lineshape. Nonetheless,
the width of the counterintuitive feature is significantly larger than the
associated intuitive feature.  This is what leads to improved performance of the
counterintuitive technique in the presence of Doppler broadening.  More atoms
significantly contribute to the total signal in the counterintuitive case.

\section{Gaussian Spatial Averaging}

\hspace{\parindent}  Another potential difficulty in using a vapor cell as
opposed to a molecular beam is that the entire spatial profile of the exciting
lasers can contribute to the total signal.  In the case of molecular beams, the
transverse spatial profile serves to produce the temporal pulse seen by the
molecules traveling in the beam.  As long as the transverse dimension of the
laser beam is much larger than that of the molecular beam (typical dimension
of 1 mm), then all molecules in the beam see the same peak laser amplitude. 
When an absorption cell is used, the spatial extent of the vapor is larger than
the transverse dimension of the laser beam, and hence, atoms in the vapor cell
see different peak laser amplitudes depending on their location relative to
the optical axis of the laser beam.  Again, this is cause for concern in terms
of the counterintuitive transfer efficiency.  Since the adiabatic criterion is
a threshold condition, there will be a two-dimensional surface in space inside
which the atomic dynamics are adiabatic and outside which they are not. 
Furthermore, more atoms are outside this surface than inside (since the laser
beam has to fit cleanly through the cell).  We may thus wonder if their
contribution to the total signal will dominate, and conspire to make the
counterintuitive transfer process less efficient than the intuitive.

The well-known Gaussian beam solution to the paraxial wave equation can be
written in cylindrical coordinates \cite{Boyd_book2}
\begin{equation}
E(r,z) = \frac{E_0}{1 + i\xi}\exp\left[\frac{-r^2}{w_0^2(1 + i\xi)}\right]
\end{equation}
with the dimensionless longitudinal coordinate
\begin{equation}
\xi = 2z/b,
\end{equation}
defined in terms of the confocal parameter
\begin{equation}
b = 2 \pi w_0^2/\lambda.
\end{equation}
The meaning of the beam parameters ($w_0$ and $b$) become clear if we examine
the amplitude variation of the beam.  Along the optical axis ($r = 0$), the
amplitude is given by
\begin{equation}
\left|E(0,z)\right| = \frac{E_0}{\sqrt{1 +(2z/b)^2}}.
\end{equation}
The confocal parameter, $b$, therefore represents the depth of focus of the
beam.  In the plane $z=0$, the transverse profile is
\begin{equation}
E(r,0) = E_0 \exp\left(-r^2/w_0^2\right).
\label{gaussian}
\end{equation}
The parameter $w_0$ then represents the minimum transverse dimension of
the Gaussian beam, and is called the spot-size. For $z \neq 0$ the beam size is
\begin{equation}
w^2(z) = w_0^2 \left[ 1 + (2z/b)^2\right].
\end{equation} 

If the spot-size, $w_0$, is much larger than the beam's wavelength, $\lambda$,
then the variation in amplitude along the optical axis is much slower than the
variation transverse to the optical axis.  We will specialize our treatment
allowing only for transverse amplitude variation, which has the simple
functional form given in \Eq{gaussian}.

Then, we must consider the integrated fluorescence to be a function of the
radial dimension through its dependence on the electric field,
$F = F\left[E(r)\right]$.  The signal measured at our detector will be
the sum over all atoms, which are uniformly distributed throughout the
spatial profile of the laser beam, 
\begin{eqnarray}
\langle F \rangle_S &=& \int F\left[E(r)\right] dA \\
&=& \int_0^{2\pi}\!\!\int_{0}^{\infty}F\left[E(r)\right] r dr d\phi
\end{eqnarray}
(the subscript ``S'' refers to the \underline{s}patial average).
While the upper integration limit in $r$ should actually be the edge of the
absorption cell, we assume here that the beam passes cleanly through the cell
so that effectively the beam is zero at the cell and the limits can be
extended to $\infty$.  By changing variables from $r \rightarrow E$ using
\Eq{gaussian}, we obtain the result for the spatially averaged time-integrated
fluorescence
\begin{equation}
\langle F \rangle_S = 2\pi w_0^2 \int_0^{E_0} F(E) \frac{dE}{E}.
\end{equation}
While the probability of a given electric field strength, $E$, increases as $E$
decreases and is divergent at $E=0$, the fluorescence signal goes
to zero more rapidly than $E$, and the average is convergent.  The
$1/E$ divergence is due to the approximation that the absorption cell is
infinite. But even for finite cell size, more atoms ``see'' a small intensity
than ``see'' a near-maximal one.  This is the source of potential difficulty for
the counterintuitive process. 

One complication that has been neglected up to this point is the fact that we
have two laser beams, and hence, two spatial profiles.  The integrated
fluorescence signal depends on both laser amplitudes, $F =
F\left[E_1(r),E_2(r)\right]$.  However, in the experiment, we were careful
to insure that the two beams were spatially coincident and had similar
divergences. Therefore,
$E_2(r)/E_1(r) = $c
(for some constant c ) for all $r$.  Then the average can be written as
\begin{equation}
\langle F \rangle_S = 2\pi w_0^2 \int_0^{E_1} F\left[E, {\rm
c}E\right]
\frac{dE}{E}.
\label{spatial_eq}
\end{equation}

\begin{figure}[tbp]
\postfull{thesisfigs/two_photon/N10spat.eps}
\bigskip
\ncap{Time-integrated fluorescence signal averaged over Gaussian
spatial beam.}{Time-integrated fluorescence signal averaged over Gaussian
spatial beam.  Pulse area (at center of beam) is $10 \pi$.
\label{N10spat}}
\end{figure}

\Figure{N10spat} shows a typical result of the evaluation of \Eq{spatial_eq}. 
In this figure, the pulses have peak area (on-axis) of $10\pi$.  The
counterintuitive pulse sequence remains the most efficient population transfer
process in spite of the spatial averaging.  The efficiency improvement from the
intuitive to the counterintuitive, however, has been decreased, as can be seen
by comparing \Fig{N10and11} with \Fig{N10spat}.  In the absence of spatial
averaging, the increase from intuitive to counterintuitive transfer efficiency
is $\sim 300\%$.  When spatial averaging is included, this enhancement
decreases to $ \sim 50\%$.  The shape of this curve stays essentially the same
for larger peak pulse area, with the enhancement factor improving only slightly.


\section{Amplitude Fluctuations}

\hspace{\parindent} We now examine the impact of amplitude fluctuations on
the population transfer process.  We model the peak intensities driving 
each transition as independent Gaussian random variables.  Then the
probability distribution of intensities (due to amplitude
fluctuations) will be given by
\begin{equation}
P_A(I_j) = \frac{1}{\sqrt{2\pi\sigma_{I_j}^2}}\exp\left[-\frac{(I_j
-\overline{I_j})^2}{2 \sigma_{I_j}^2}\right]
\end{equation}
(the subscript ``A'' refers to \underline{a}mplitude fluctuations).
In this equation, the index $j =\{1,2\}$ represents the intensities driving
the two transition,
$\overline{I_j}$ is the mean of intensity $j$, and $\sigma_{I_j}$
is the standard deviation of $I_j$'s fluctuations.

Since the fluorescence signal depends on the random variables, $I_1$ and $I_2$,
it in turn is a random variable.  We thus compute the average fluorescence
signal,
\begin{equation}
\langle F \rangle_A = \int_0^{\infty}\!\!\int_0^{\infty} F(I_1,I_2)
P_A(I_1) P_A(I_2) dI_1 dI_2.
\label{F_A}
\end{equation}
We also will be interested in examining the fluctuations of the fluorescence
signal.  This is calculated according to
\begin{equation}
\sigma_F = \sqrt{\langle F^2 \rangle_A - \langle F \rangle_A^2}.
\end{equation}
Furthermore, we shall characterize the fluctuations relative to the mean value
by examining the variable, 
\begin{equation}
\sigma_F/\langle F \rangle_A.
\label{relative_fluc}
\end{equation}
For the subsequent
calculations, we have taken the two fields to have equal means, $\overline{I}$,
and standard deviations of $\sigma_I = 20\%\ \overline{I}$.  These choices
model the experimental situation discussed in the next chapter.

\begin{figure}[tbp]
\postfull{thesisfigs/two_photon/fluc_ave.eps}
\bigskip
\ncap{Time-integrated fluorescence signal averaged over amplitude
fluctuations.}{Time-integrated fluorescence signal averaged over amplitude
fluctuations.  Both fields have mean intensities, $\overline{I}$, that produce
$11 \pi$ pulses, and fluctuation standard deviations of
$\sigma_I = 20\%\ \overline{I}$.
\label{fluc_ave}}
\end{figure}

\Figure{fluc_ave} displays the average integrated population, from \Eq{F_A},
versus delay between pulses.  The average pulse area is $11 \pi$.  Generally,
this shape is very similar to the curves without any fluctuations, except that
the interference oscillations on the intuitive side have been washed-out.  The
counterintuitive side still has the flat peak transfer efficiency region.  The
transfer efficiency enhancement in going from intuitive pulse delays to
counterintuitive is approximately the same as in the fluctuation-free case.  We
note that what makes the amplitude fluctuations have such small effect (in terms
of average quantities) is the presence of decay.  In the absence of decay, the
interference terms on the intuitive side have complete modulation.  That is, the
signal cycles from zero to some maximum value and back.  When fluctuations are
present these oscillations will average to half the maximum value.  Further out
in the intuitive regime, where the pulses are not overlapping, the amplitude
fluctuations give rise to an averaged signal of approximately one-quarter.  This
is due to the fact that, if the pulses have fluctuation in area greater than
$\pi$, then the first pulse fluctuates between complete and no population
transfer to the middle state.  Hence, on average, half the population reaches
the middle state.  Then the second pulse similarly redistributes the population
so that on average, one-quarter of the total population reaches the upper state.

\begin{figure}[tbp]
\postfull{thesisfigs/two_photon/fluc.eps}
\bigskip
\ncap{Fluctuations of fluorescence signal in the presence of
laser amplitude fluctuations.}{Fluctuations of fluorescence signal in the presence of
laser amplitude fluctuations.  Both fields have mean intensities,
$\overline{I}$, that produce $11 \pi$ pulses, and fluctuation standard
deviations of $\sigma_I = 20\% \ \overline{I}$.
\label{fluc}}
\end{figure}

The fluctuations of the fluorescence signal are shown in \Fig{fluc}.  Here the
relative fluctuations, as defined in \Eq{relative_fluc}, are shown as a
function of delay between the laser pulses.  This figure shows several
interesting features.  We see that for large intuitive delays, the relative
fluctuations are constant.  We can explain this constancy with a simple
argument.  Assume the first pulse transfers population to the middle state so
that its population is $x_{11} = f_1(\theta_1)$, where $\theta_1$ is the
area of lower transition pulse.  Then, this population decays for a time $\sim
\Delta \tau$ to a value $\exp(-A_1 \Delta \tau) f_1(\theta_1)$. Then the
second pulse cycles the remaining population through Rabi oscillations so that
when it has finished, $f_2(\theta_2) \exp(-A_1 \Delta \tau) f_1(\theta_1)$
population is transferred to the upper state (with $\theta_2$ the area of
the upper transition pulse).  This value will be averaged over the amplitude
fluctuations, but the decay term is not affected.  The calculation
of the variance proceeds in a similar fashion and it is proportional to
$\exp(- 2 A_1 \Delta \tau)$.  When the standard deviation is calculated and
the ratio taken, the decay terms cancel and we predict delay-independent
fluctuations, which is what is seen in \Fig{fluc}.

After this flat feature, there is a local minimum on the intuitive side, where
the fluctuation is reduced to approximately half its previous value, then grows
rapidly in a narrow spike near zero.  Finally, on the counterintuitive side,
there is a deep minimum that nearly reaches zero.  In this regime the atomic
system acts as a noise-eater for the laser $\rightarrow$ fluorescence
input $\rightarrow$ output system.  This happens because the energy dependence
of the adiabatic transfer process is given by a threshold condition.  Once the
laser fields have enough energy to satisfy the adiabaticity criterion in
\Eq{adiab_criterion}, then larger energies will also satisfy it. 
Therefore, as a ``rule of thumb'', if the mean laser intensity minus its
standard deviation is above threshold, then all pulses will satisfy the
adiabaticity criterion, and hence the entire population will be transferred to
the upper state, independent of the fluctuations.

\section{Extension to a Rydberg Final State}

\hspace{\parindent} The ideal three-level results we have discussed are
applicable to a Rydberg final state, provided the final transition behaves as
a two-level system.  In \rChapter{direct}, we outlined the necessary conditions
for the ground-to-Rydberg transition to behave as a two-level system.  The key
requirement was to keep the peak electric field as low as possible, consistent
with complete population transfer.  Then, if we want to use the counterintuitive
excitation technique to excite a Rydberg final state, the electric field driving
the Rydberg transition must be kept small.  There is an advantage in this scheme
in that excited levels have larger dipole moment coupling to Rydberg states than
does the ground state.  Thus the electric field needed to invert the transition
will be smaller.  There also is a disadvantage to this scheme that as
the applied laser frequency decreases, so does the threshold for optical
mixing of the degenerate angular momentum states (see \Eq{mix}). 
So whether this technique is more or less susceptible to angular momentum
mixing than the ground-to-Rydberg transition depends on the relative scaling
of these two quantities.

\begin{figure}[tbp]
\postfull{thesisfigs/two_photon/efield.eps}
\bigskip
\ncap{Influence of optical mixing of angular momentum on counterintuitive
population transfer technique.}{Influence of optical mixing of angular momentum on counterintuitive
population transfer technique.  $E_{\pi}$ is the electric field needed to make
a $\pi$-pulse from state $n=n_i$ to $n=80$.  $E_{\ell}$ is the electric field
needed to induce optical mixing at $n=80$ in a field which resonantly couples
state $n=n_i$ to state $n=80$. Both fields are normalized to unity at $n_i=2$ to
show scaling as a function of $n_i$ and are plotted on a log scale.
\label{efield}}
\end{figure}

The electric field required for a $\pi$-pulse scales as $E_{\pi} \sim
1/d_{n\, n_i}$ (\Eq{pi_pulse}) where $d_{n \, n_i}$ is the dipole moment from the
intermediate state (labeled $n_i$) to the Rydberg state, $n$.  The threshold
electric field for $\ell$-mixing scales as $E_{\ell} \sim \omega_{n \, n_i}$
(\Eq{mix}), with $\omega_{n \, n_i}$ the $n_i \rightarrow n$ transition
frequency.  So both fields are decreasing as $n_i$ increases and approaches
$n$.  To check the scaling of these fields, we consider excitation to the
$n=80,\ell = 2$ Rydberg state via an intermediate $\ell=1$ state that varies from
$n_i =$ 2 to 79 (we will use hydrogen dipole moments and frequencies).  We
normalized the two electric fields to be equal at
$n_i=2$, so that we may just investigate the scaling.  So if $E_{\pi}$ goes to
zero faster than
$E_{\ell}$, then optical mixing will be less important for the case of
coupling of an excited state to a Rydberg state as compared to coupling of the
ground state to a Rydberg state.  On the other hand, if
$E_{\ell}$ goes to zero faster than $E_{\pi}$, then optical mixing will be more
important and we would predict the counterintuitive transfer technique to be
more susceptible to optical mixing than direct excitation from the ground state. 
\Figure{efield} shows the scaling of these two fields as a function of the
intermediate state's principal quantum number, $n_i$.  While there is some
deviation between the two curves, both electric fields have essentially the same
scaling.   This shows that the issue of angular momentum mixing in an ac field is
essentially independent of what lower-lying state is being coupled to  the
Rydberg state (for low angular momentum states).  Consequently, the conditions
which determine when optical mixing is unimportant for ground-to-Rydberg
coupling are essentially the same as in the case of excited state to Rydberg
coupling.  There will then be a range of Rydberg final states for which the
ideal three-level system closely approximates the full dynamics.

%\end{document}

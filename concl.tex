\begin{singlespace}

\chapter{Conclusions}
\label{concl}
\begpagestyle
%\pagenumbering{arabic}

\end{singlespace}

In this thesis, we have examined optical excitation of atomic Rydberg
states for both direct excitation from the ground to the Rydberg state as
well as excitation proceeding through an intermediate resonance.  We have
taken the goal of the interaction to be efficient and selective excitation
of the target Rydberg state.  This work has included theoretical and
experimental examinations of each process.

In the case of direct excitation, we find that the coupling between nearly
degenerate Rydberg states with the same principal quantum number but differing
angular momentum quantum numbers can seriously degrade the selectivity of the
excitation process.  While this is a negative result in terms of achieving the
goals of efficient and selective excitation, we have shown this result to give
rise to very interesting dynamics of these strongly driven Rydberg states.  The
coupling between these Rydberg states is characterized by Rabi frequencies that
exceed the optical frequency, even for modest optical field strengths.  The
angular distribution that results is peaked orthogonal to the laser
polarization direction.  The atom emits high harmonics of the laser driving
field, and the Rydberg population becomes stabilized for increasing laser
intensity, becoming a decreasing and oscillating function of intensity above a
threshold level.

In an effort to understand these results, we developed a model of the
interaction based on Landau-Zener level crossing theory.  In these optical
electric fields, the linear Stark effect causes the Rydberg states to shift
through the ground state.  The coupling between these two states gives rise to
an avoided crossing.  These avoided crossings occur twice every optical cycle. 
We showed that the predictions of this model are in excellent agreement with the
previous theory.

We investigated the three-photon ionization of atomic potassium when a
picosecond dye laser is tuned through two-photon resonance with the Rydberg
series.  As the laser is tuned higher in the Rydberg series, the coupling
between the nearly degenerate Rydberg levels increases, so that the tuning of
the laser controls the degree of the optical mixing of the Rydberg states. 
The shape of the ionization curve is shown to change as the peak laser
intensity is varied.  We developed two models of the interaction.  One is based
on traditional multiphoton physics that ignores the Rydberg-Rydberg
interaction, the other includes the optical mixing of Rydberg states, but
ignores many of the subtleties of the potassium atom.  We find good agreement
with experiment only when the Rydberg-Rydberg interaction is included,
demonstrating that the optical mixing of Rydberg states is the dominant
interaction in this system.

In the case of optical excitation via an intermediate resonance, we investigate
the population transfer efficiency as a function of the time delay between two
laser pulses resonant with the two transitions in a cascade system.  We show
theoretically that even in the presence of Doppler broadening, transverse
variation of the laser beam, and amplitude fluctuations of the laser that the
transfer efficiency is maximized when the pulses are applied in the
counterintuitive order.  We investigate this interaction experimentally in
atomic sodium vapor and verify these predictions.  We also find that the noise
in the population transfer signal is reduced in the counterintuitive regime
relative to the intuitive regime.

Of course, there is much more work that can be done is these areas.  Many of
the theoretical predictions of optical mixing of Rydberg states were not
experimentally observed.  For example, the prediction of high harmonic
generation at these very low intensities could give rise to a more compact
source of DUV and XUV radiation, given that the pump laser requirements would
be significantly reduced compared to current requirements.  In the area of the
counterintuitive excitation scheme, actually extending our initial work to a
Rydberg final state remains to be demonstrated.  Furthermore, we have found
some initial theoretical evidence  that this technique might be particularly
well suited to excitation of a Rydberg wave packet final state.  This would
provide a way to completely transfer the population to the wave packet state,
which could have an impact on strong field coherent control studies.

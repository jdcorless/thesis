% Chapter 1:  Introduction


\begin{singlespace}

\chapter{Introduction}
\label{intro}
\begpagestyle
\pagenumbering{arabic}

\end{singlespace}



\section{Rydberg Atoms}
\hspace{\parindent}  When a valence electron of an atom is excited to a state
with large principal quantum number, then the atom is called a Rydberg atom. 
These highly excited electrons spend most of their time far away from the
nucleus.  Consequently, the dominant force experienced by the electron is the
Coulomb field of the core of unexcited electrons and nucleons.  The behavior
of these Rydberg electrons then, in many ways, is very similar to that of an
excited electron in a hydrogen atom.  This property makes the behavior of
all Rydberg atoms essentially universal, even though the ground states of these
atoms have properties varying across the periodic table.

\begin{table}[t]
\begin{center}
\begin{tabular}{||c|c|c|c||}                                      \hline
{\bf Property}&{\bf Scaling}&{\bf Value $(n\sim 30)$}&{\bf Units}\\ [5pt]
\cline{1-4} 
radius               & $(n^*)^2$        & $10^3$    & \AA \\ [5pt]
\cline{1-4} 
binding energy       & $(n^*)^{-2}$     & $10^{-2}$ & eV  \\ [5pt]
\cline{1-4} 
linear Stark effect  & $(n^*)^2$ & $10^3$ & MHz/(V/cm) \\ [5pt] \cline{1-4}
ionizing field       & $(n^*)^{-4}$     & $10^3$    &V/cm \\ [5pt]
\cline{1-4} 
transition frequency & $(n^*)^{-3}$     & $10^{11}$ & Hz  \\ [5pt]
$\Delta n^* = 1$     &                  &           &     \\ \cline{1-4}
radiative            & low-$\ell$ $(n^*)^3$ & $10^{-5}$ & sec \\ [5pt]
lifetime             & high-$\ell$ $(n^*)^5$ & $10^{-2}$ & sec \\ [5pt] \hline
\end{tabular}
\ncap{Properties of Rydberg atoms.}{Properties of Rydberg atoms.
\label{table1}}
\end{center}
\end{table}

This would be reason enough to warrant investigation of such universal atomic
systems.  However, Rydberg atoms have many other interesting properties which
make them exciting to study.  To characterize some of these properties, we
shall use the effective principal quantum number, $n^* = n -
\delta_{\ell}$, where $n$ is the principal quantum number and $\delta_{\ell}$
is the quantum defect.  In general, $\delta_{\ell}$ is dependent both on $n$
and $\ell$.  However, for the majority of Rydberg systems, the dependence on
$n$ is very small and the dependence on $\ell$ very large.  The effective
principal quantum number  enables us to write the properties of Rydberg atoms in
a form that encompasses all Rydberg atoms.  Some of these properties are listed
in Table~\ref{table1} \cite{Gallagher_book,Stebbings_book,Kleppner:86}.  The
outer turning point of the electron's radial motion scales as $(n^*)^2$.  The
electrons are weakly bound with binding energies scaling as $(n^*)^{-2}$. 
Because of this small binding energy, these electrons are nearly free, and hence
are easily perturbed by dc fields (electric and magnetic).  When weak electric
fields are applied to Rydberg atoms, the linear Stark effect is the dominant
effect and it scales as
$(n^*)^2$.  Then for larger electric field strengths, dc field ionization can
take place.  The electric field required to field ionize a $n^*$ Rydberg states
scales as
$(n^*)^{-4}$. The transition frequencies between the $n^*$ and the $n^*+1$
Rydberg states scales as $(n^*)^{-3}$.  And finally, the radiative lifetimes of
Rydberg states are relatively long compared to lower-lying states.  For
low-$\ell$ states, these lifetimes scale as $(n^*)^3$ and for high-$\ell$ states
as $(n^*)^5$.

A very important consequence of these properties that will play a large role in
subsequent chapters is that dipole moments between Rydberg states scale as
$(n^*)^2$.  Since the energy difference between neighboring Rydberg states is
very small, their wavefunctions are very similar throughout most of space. 
Consequently, the dipole matrix element, which involves the overlap of these
two states, will be very large and scale with the size of the wavefunction. 
These large dipole moments cause Rydberg atoms to be very susceptible to
dc and low-frequency fields.  For instance, blackbody radiation, which has a
negligible impact on ground state atoms at room temperature, can shift the
energies of Rydberg atoms \cite{Gallagher:79} as well as induce transitions
even to the extent of photoionizing Rydberg states \cite{Spencer:82}. 

The variety of research that has been performed in Rydberg atoms is large, and
ranges from spectroscopic studies of the Rydberg states themselves to
fundamental tests of light-matter interaction.  We shall briefly describe some
of the most vibrant areas that are receiving current attention.  The
interaction of a Rydberg atom with a high-$Q$ microwave cavity has given rise to
a field known as cavity quantum electrodynamics.  Since the microwave frequency
resonant in the cavity can be varied by changing the length of the cavity, the
coupling between two Rydberg states that are near resonant with this microwave
field can be easily controlled.  This has led to observations of
cavity-enhanced \cite{Goy:83} and cavity-inhibited \cite{Hulet:85} spontaneous
emission.  These studies led to the development of the one-atom maser
\cite{Meschede:85} where maser action was achieved with less than one atom in
the cavity.  The microwave field generated by this maser was shown to exhibit
quantum jumps for certain cavity parameters \cite{Benson:94}.  Direct evidence
for field quantization in these microwave cavities has been observed by Brune
\etal \cite{Brune:96a}, and recently,  the same group reported exciting results
of an investigation of the role of decoherence in the quantum measurement
process \cite{Brune:96b}.

When the coherent bandwidth of an ultrashort pulse is large enough to overlap
several Rydberg states, then a coherent superposition of Rydberg states is
formed. These Rydberg wave packet states have been the source of much research
since their original proposal \cite{Parker:86,Alber:86}.  While these works
originally proposed a wave packet localized in the radial variable,
the first experimental work entailed observation of a wave packet localized in
the angular variables \cite{Yeazell:87,Yeazell:88}.  However, further work
primarily focused on radial wave packets, with observations of the
classical-like motion of these wave packets at the Kepler period
\cite{Wolde:88,Yeazell:89}.  Because the discrete Rydberg energy levels are not
equally spaced, this periodic motion begins to decay and ultimately revives
\cite{Yeazell:90}.  In between the decay and revival there are so-called
fractional revivals, where multiple versions of the original wave packet orbit
the nucleus at the same time
\cite{Averbukh:89,Yeazell:91,Meacher:91}.  A suggestion for the excitation of a
classical limit state localized in all three dimensions and traveling on a
Kepler ellipse was made by Gaeta \etal \cite{Gaeta:94}. 
Three dimensionally localized Rydberg wave packets excited by microwave
fields with frequencies given by the Kepler frequency have been analytically
studied \cite{Birula:94} and numerically demonstrated \cite{Kalinski:95}. Radial
wave packets have been used to demonstrate electron interference
within an atom \cite{Noel:95} and to create a Schr\"{o}dinger cat state of an
atom \cite{Noel:96}.

When a strong optical field interacts with an electron in a Rydberg state, the
ionization of this state takes place predominantly near the core.  This can be
seen by examining the energy and momentum conservation laws and realizing that
the nucleus is required to satisfy both \cite{Eberly:91}.  A ``dark wave
packet'' is then formed, where a hole in the electron probability distribution,
due to photoionization near the core, oscillates at the Kepler period
\cite{Noordam:92a,Jones:93b}.  The fact that the pulse duration can be shorter
than the Kepler period of the Rydberg states can give rise to ionization
suppression in these strong optical fields
\cite{Parker:90,Stapelfeldt:91,Jones:91}.

The development of high peak intensity $\sim 1$~ps half-cycle electromagnetic
field pulses \cite{You:93} has given Rydberg atom researchers a new tool to
study Rydberg states.  Investigations in this area have involved studies of the
ionization due to these short electromagnetic field pulses \cite{Jones:93a}
and the classical correspondence of this process \cite{Frey:96}.  These pulses
have been shown to generate wave packet states and their dynamics have been
observed \cite{Raman:96,Jones:96}.

\section{Excitation Techniques}
\hspace{\parindent}  There are three main techniques that have been used to
excite Rydberg atoms in the laboratory.  They are charge exchange, electron
impact, and optical excitation.  These three processes can be represented
by the reaction formulas
\begin{eqnarray}
A^+  + B & \rightarrow & A(n\ell) + B^+ \\
e^-  + A & \rightarrow & A(n\ell) + e^- \\
h\nu + A & \rightarrow & A(n\ell).
\end{eqnarray}
In these formulas, $A$ is the species to be excited to the Rydberg
state, $A(n\ell)$.  All three of these techniques have excitation cross-sections
that have the same dependencies on the principal quantum number.  This scaling is
$(n^*)^{-3}$ and can be understood as follows.  In each process, a localized
electron (which may be attached to a different atom) is projected onto a
spatially extended Rydberg wavefunction.  The amount of electron probability near
the nucleus for a $n^*$ Rydberg state scales as $(n^*)^{-3}$
\cite{Bethe_Salpeter}.  The overlap between the localized and the extended
state, which determines the cross-section, then scales as $(n^*)^{-3}$. 
Efficient excitation to high Rydberg states, therefore, becomes increasingly
more difficult for higher $n^*$ states.

In the case of the charge exchange reaction, an ion beam passes through an
ensemble of neutral atoms and in the course of colliding with these atoms, the
neutral atom gives one of its electrons to the ion.  The recapture of this
electron occurs at various levels of excitation.  Consequently, a whole range
of Rydberg levels are excited by this process.  This technique has been the
main method of producing hydrogen atoms in Rydberg states \cite{Bayfield:74}. 
Besides being non-selective, this technique is also very inefficient.  If the
density of neutral atoms is made large, the probability of a charge exchange
reaction can also be made large.  However, the probability that the Rydberg
atom will experience a collision with the neutrals then also increases, and the
Rydberg atoms may either be de-excited or scattered out of their original
direction by this second collision.  Thus, the beam of ions becomes a beam of
Rydberg atoms that will  not make it out of the charge exchange cell if the
probability of creating a Rydberg state is large.

Electron impact involves the rather conceptually straightforward interaction of
colliding an accelerated beam of electrons with a target neutral
atom \cite{Schiavone:79}.  The collision process serves to redistribute some of
the incident electron's energy to the energy of the bound electron in the target
atom.  Again, this process produces a large range of $n^*$ levels.  The
technique also has the possibility of exciting high-$\ell$ states.  The
impinging electrons excite bound electrons to Rydberg states with
low-$\ell$ because the overlap between the ground state and high-$\ell$
Rydberg states is small.  Subsequent collisions, however, redistribute
low-$\ell$ states among high-$\ell$ states.  Evidence for these high-$\ell$
states arises from the observation of $n^*$ states whose low-$\ell$ components
spontaneously decayed long before the $n^*$ state was observed.  Since
high-$\ell$ states have much longer lifetimes (see Table~\ref{table1}), $n^*$
states that have large-$\ell$ should survive longer than low-$\ell$ and hence be
detected.

The final technique we consider is optical excitation.  This is the method that
will be studied in this thesis.  It has the advantage of selectivity and the
potential for high efficiency.  We will actually show that one may lose
selectivity in an attempt to improve efficiency.  There are various coupling
schemes possible to excite Rydberg states.  The most direct approach is to
couple the ground state of the atom directly to the desired Rydberg state. A
recent example is the excitation of $n \sim 520$ states of K using CW
ultraviolet laser light \cite{Frey:96}.  This technique is generally limited
to the elements with smaller binding energies ($\sim 3$ eV), namely the alkalis
and alkaline-earths.  For these atoms, a single UV photon can resonantly couple
the ground to the Rydberg state.  For elements such as the noble gases, the
binding energy is so large that the required deep-UV laser pulse is not
conveniently available.  Sometimes this excitation is multiphoton in nature
\cite{Harper:77}.  Then, two or more photons are used to excite the Rydberg
state with lower-lying intermediate states being far from resonance.

Even though direct excitation is the most straightforward optical technique,
the most common is step-wise excitation \cite{Fabre:75,Zimmerman:79}. 
In this case, a laser pulse is tuned to resonance between the ground state
and some low-lying excited state.  A second laser pulse, delayed in time
with respect to the first, resonantly couples the low-lying excited state to
the Rydberg state.  This is the two-step method, but can be extended
to more steps.  This approach has the possibility of achieving higher transfer
efficiencies than the direct excitation technique, because strong optical
transitions are used that can be more easily saturated.  The final step will
have a larger dipole moment coupling to the Rydberg state than the ground state
does and hence, this transition too may be more easily saturated.  The delay
between the two pulses must be shorter than the lifetime of the intermediate
state.

\section{Overview of Thesis}
\hspace{\parindent} In this thesis, we investigate the optical excitation of
Rydberg states considering both direct excitation and two-photon excitation.
We take the goal of the interaction to be efficient population transfer from
the ground state to a target Rydberg state.  In \rChapter{direct}, we examine
theoretically the direct excitation process and find that under very general
conditions, the interaction between nearly degenerate Rydberg states with the
same $n^*$ but differing $\ell$ can dominate the dynamics in a strong optical
field.  We examine this interaction and predict many phenomena due to the
mixing of these Rydberg states.  These include population transfer to higher
angular momentum states giving rise to angular distributions perpendicular to
the laser polarization direction, emission of high-harmonics of the laser
driving field, and laser induced stabilization of the Rydberg population.  We
develop a model of the dynamics based on Landau-Zener level crossing theory,
which has been successful in describing the behavior of Rydberg atoms in dc
and low-frequency fields.

In \rChapter{step}, we describe an experimental investigation of Rydberg atoms
in strong optical fields, for which the optical mixing of Rydberg states is
expected to be an important effect.  We measure the three-photon ionization
signal tuning through two-photon resonance with the potassium Rydberg series
for varying intensities.  We find a general tendency in the shape of the
ionization curve as the laser intensity is increased.  We model the interaction
using a traditional approach from multiphoton physics that neglects the
Rydberg-Rydberg coupling and find very poor agreement.  We then modify our
analytic results from \rChapter{direct} to include ionization and find very
good agreement, demonstrating that the optical mixing of Rydberg states is the
dominant interaction in this system.

We begin our study of the step-wise optical excitation of Rydberg states in
\rChapter{citheory}.  A theoretical investigation of the population transfer
in a doubly resonant cascade system in sodium is considered as a function of
the time delay between the two resonant laser pulses.  This leads us into the
field of counterintuitive pulse sequences, and we examine the affects of
various experimental perturbations on the transfer efficiency.  We model
Doppler broadening, the transverse profile of the laser beam, and amplitude
fluctuations, and find that even in the presence of these effects, the
counterintuitive pulse ordering is the most efficient technique to excite the
Rydberg final state.

An experimental investigation of this counterintuitive excitation is made in
\rChapter{ciexpt}.  We utilize a system of atomic sodium vapor interacting
with nanosecond dye laser pulses.  This system is inherently susceptible to
Doppler broadening due to the thermal motion of the sodium atoms in the
absorption cell.  Also, the geometry we use measures the population transfer
averaged over the transverse profile of our laser beams.  We measure the
population transfer as a function of time delay in this system, and also
investigate the noise properties of this population transfer process in the
presence of amplitude fluctuations in our dye lasers.  We find that the
population transfer is more efficient and less noisy when the exciting lasers
are applied in the counterintuitive order.

